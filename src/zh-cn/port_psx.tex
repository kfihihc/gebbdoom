\begin{wrapfigure}[11]{r}{0.3\textwidth}{
\centering \scaledimage{0.3}{psx_logo.png}}
\end{wrapfigure}
PlayStation 的历史始于 1988 年,当时 Nintendo 与 Sony 合作,为 SNES 开发 CD-ROM 读盘器扩展。在合同条款中,Sony 可以独立为该平台开发,并保留 “Super Disc” 格式控制权——这对 Nintendo 来说是两项不寻常的让步。\\
\par
项目一直推进到 1991 年 CES,此时 Sony 宣布了名为 “Play Station” 的合资计划。第二天,同一场活动上,Nintendo 又宣布改与 Philips 合作(条款更有利),令 Sony 大感震惊与羞辱。被背叛的 Sony 试图转向 Sega 董事会,但迅速被否决。2013 年采访中,时任 Sega CEO Tom Kalinske 回忆了董事会的结论:\\
\par
\rawfq{这是个愚蠢的想法,Sony 不懂硬件,也不懂软件。我们为什么要这么做?}\\
\par
他们并非没有道理。Sony 几乎没有游戏经验,也几乎没有兴趣涉足这个领域,因为此前一切都系于一人:Ken Kutaragi。自从看到女儿玩 Nintendo Famicom 后,Ken 一直鼓吹 Sony 进军游戏。他甚至违背 Sony 高管的建议,为 SNES 设计了音频芯片(SPC700)。\\
\par
尽管其他 Sony 高管认为这是高风险赌局,Kutaragi 仍得到 CEO Norio Ohga 的支持。1992 年 6 月,Ken 得到绿灯,从零开始打造游戏系统\footnote{《Playstation: Anthology》by GREEKS-LINE。}。这位后来被称为“PlayStation 之父”的人,被转岗到财务独立的 Sony Music 以安抚董事会,从而得以推进“PlayStation”(无空格)的研发。\\
\par
起初架构方向不确定:要么专注 2D 精灵,要么专注 3D 多边形。1993 年 10 月 Sega 的《Virtua Fighter》在日本街机上的成功消除了所有疑虑\footnote{来源:“How Virtua Fighter Saved PlayStation's Bacon”. WIRED. 2012 年 9 月。}:PSX 走向 3D。\\
\par
两年后项目 culminate 为 Sony Computer Entertainment,并于 1994 年 12 月 3 日在日本发布。它一举成功:首日卖出 10 万台,六个月售出 200 万台,生命周期销量达 1.02 亿台。\\
\par

\cfullimage{consoles/PSX.png}{Sony PlayStation}

\vspace{-20pt}
\subsubsection{成功要素}
\par
在诸多正确选择中,Sony 听取开发者反馈,将 RAM 从 1 MiB 提升到 2 MiB。它采用以开发者为中心的态度,使开发周期简单、工具频繁更新且可在线下载,并提供第三方技术支持。CD 格式让游戏定价更低,开发者也无需从 Sony 购买卡带。\\
 \par
更重要的是,Sony 不像 Nintendo 与 Sega 那样审查开发者。与此同时,版税更低,提高了盈利能力\footnote{Nintendo 有时会抽取高达 20\%。}。Kalisto Gaming 的 CEO 后来说:“PlayStation 解放了我们。”\\
\par
Sony 的好运在于收购 Psy-Q,使 PSX 的 SDK 成为程序员的梦想。Psygnosis 是一家英国游戏公司,曾在 Atari ST、Amiga 与 SNES 上开发。Sony 在 1993 年初收购 Psygnosis,并指派其开发当时仍保密的 PlayStation 游戏 \textit{Wipeout} 与 \textit{Destruction Derby},用于首发展示。\\
\par
在此之前,Sony 设想 PlayStation 的开发要依赖专用的 Sony NEWS MW.2 工作站\footnote{“The development system”, Next Generation 1995 年 6 月。}。这些机器基于 MIPS R4000,体积庞大且价格昂贵。Psygnosis 不喜欢这套方案,尤其是在比较过他们常用的工具(由 SN Systems 出品的 Psy-Q)后。\\

\vspace{-10pt}
\begin{wrapfigure}[10]{r}{0.21\textwidth}{
\centering \scaledimage{0.21}{psyq_logo.png}}
\end{wrapfigure}
1993 年圣诞节前后,SN Systems 的合伙人 Andy Beveridge 与 Martin Day 收到 Psygnosis 给的一台 MW.2,要求是:让 Psy-Q 能跑起来!两人昼夜奋战,把 GNU 工具链(\cw{cc} 编译器、\cw{ld} 链接器、库构建器 \cw{ar} 与 \cw{gdb} 调试器)移植到 PC 上,并通过 PC 连接 Sony 的 MW.2 盒子,在 1994 年初的 CES 拉斯维加斯展上演示。\\
\par
Sony 很喜欢,立刻订购了 700 套开发套件。1994 年春末,开发套件硬件缩小为两张 ISA 卡(DTL-H2000),连接 SCSI 硬盘,因此无需刻录光盘测试。\\
\vspace{5pt}
\cscaledimage{0.95}{DTL-H2000-CPU1and2.png}{DTL-H2000 双 ISA 卡开发套件。SCSI 硬盘与 CD 刻录机未显示}
\vspace{-10pt}
\par
利用 PC 不仅大幅降低开发成本,也降低了进入门槛,因为多数开发者已经熟悉 Windows。\\
\par
从 1993 年 9 月到 1995 年 6 月,全球 500 家授权开发商涌入 Sony 的梦想主机\footnote{“Sony's PlayStation game plan”, Next Generation 1995 年 6 月。}。开发者购买了 PSX 及其开发套件。\\
PlayStation 的编程体验令人愉快。大多数编程以 C 语言完成,必要时可手工写汇编。Psy-Q 提供的编译器驱动能把一组 \cw{.c}/\cw{.obj} 文件一键输出为 PlayStation 可执行文件。\\
\par
PSX 的编程哲学是不要让开发者在多个系统间周旋。举例来说,1MiB 的视频帧缓冲不能直接访问,必须交由 GPU 处理。下面这个来自 PSX 开发手册的例子很好地总结了他们减负的用心。\\
\par
\fq{
CPU 只负责向专用硬件提供非常少量的数据,例如显示位置与数据传输起始地址。数据通过 DMA 控制器传输,由 GPU 消耗。这种并行处理的结果是 CPU 几乎可以把所有时间用于生成绘制命令列表。}{PlayStation, Net Yaroze Manual}\\
\par
\rawdrawing{psx_arch}
\par







\fullimage{Sony-PlayStation-SCPH-1000-Motherboard}
\par
\vspace{15pt}
打开 PlayStation 查看主板,可以看到不少于 15 颗芯片\footnote{来源:NEXT Generation Issue \#6 1995 年 6 月,“Inside the Playstation”。}。\\
\par
\circled{1} 32 位 33MHz R3000 CPU(30 Mips),带 4KiB i-cache 与 1KiB d-cache,并包含 88 Mips 的几何变换引擎(GTE)、DMA 控制器与 Sony 的 80 Mips MDEC 视频解压硬件。
\circled{2} 操作系统 ROM。
\circled{3} GPU。
\circled{4} 2 MiB RAM。
\circled{5} 1 MiB VRAM。
\circled{6} DSP。
\circled{7} 512 KiB DSP RAM。
\circled{8} CD 控制器:包含 CD ROM-XA 转换器(允许多达 8 路音频与 CD 数据混合流)与少量缓冲 RAM。
\circled{9} CD 驱动 DSP。
\circled{A} 16 位视频数字转换器。
\circled{B} 视频解码与编码(NTSC 或 PAL)输出到电视。


\rawdrawing{psx_motherboard}
\par
\begin{wrapfigure}[10]{r}{0.4\textwidth}{
\centering \scaledimage{0.4}{trex.png}}
\end{wrapfigure}
最初说服开发者加入 PlayStation 并不容易。1993 年 10 月 27 日,Sony 聚集了 300 名开发者、代表 60 家游戏发行商进行了一次强势展示。他们看到了著名的“恐龙演示”\footnote{来源:“PlayStation: Anthology, p30”。},其中展示了可实时控制的 T-rex 恐龙。\\
\par 演示以 512x256 分辨率、50 FPS 运行,每帧处理约 1800 个多边形并绘制多达 1300 个多边形。1993 年的《侏罗纪公园》仍在玩家记忆中,令人惊叹的演示迅速在游戏圈传播,SDK 订单随之暴涨。
 
 






\subsection{\doom{} 在 PlayStation 上}
\doom{} 的 PSX 移植由 Williams Entertainment 完成。一个 5 人团队\footnote{三名设计/美术与两名程序员。}耗时不到一年,将引擎、资产与系统适配到仅 3 MiB RAM。最终成果被普遍认为是最好的主机版之一,某些方面甚至超越 PC 版。\\
\par
工作并非从零开始。团队利用了 Jaguar 版的成果,尤其是使用了更简化、纹理更少的地图。\\
\par
\fq{画面被削减:纹理缩小、精灵、怪物与武器都缩小。……有时动画帧也被删减。}{Harry Teasley}\\
\par
这些限制不必像 Atari 主机那样苛刻。得益于 CD-ROM 容量,最终包含 59 张地图(\doom{} 的 33 张与 \doomii{} 的 26 张)。为补偿慢访问时间与 RAM 限制,每张地图被存成独立的 \cw{WAD} 文件。另一方面,除了 Arch-Vile 之外几乎所有怪物都保留。\\
\par
\fq{Arch-vile 的动画帧数是其他怪物的两倍,我们无法在 PSX 上做到它的表现。不能删掉它的攻击,也不能删掉它的复活能力。它太大了,无法包含进去。}{Harry Teasley(设计师)接受 \cw{doomworld.com} 采访}\\
\par
令人惊讶的是相对 PC 版的改进。\\
\par
声音方面,SPU 处理器实现了小房间的混响效果。Spectre(PC 上以 Predator 式“闪烁”模拟透明)改为减法混合。音乐提升到 CD 品质(44KHz、16 位、立体声)。\\
\par
最惊艳的新增特性是 16 位彩色光照:为扇区添加颜色并将纹理与之 50/50 混合。在某些场景中它改善了玩法,例如下一页红光指示需要红钥匙的门。为完善效果,玩家手也会随之上色。\\
\par


\fullimage{psx_screen_door.png}\\

\fullimage{psx_screen_wall.png}\\
\par


还有许多细节改动,例如动态天空。在下面的截图中,士兵身处地狱,天空中有华丽的火焰效果。\\
\par
\fullimage{psx_screen_sky.png}\\
\par
PSX 的 GPU 体现了硬件加速的取舍:生成图形更像是管理一件工具,而不是逐像素手工绘制。\\
\par
在一系列改动中,开发者无法直接访问 VRAM。纹理传输通过 DMA 进入一个自由布局的 1024x512 16 位区域,该区域也存放帧缓冲(称为 “Display Area”)。与其写纹理映射器,开发者更像是在写 VRAM 分配器。另一个怪癖是精灵与纹理必须是 2 的幂尺寸,以优化纹理采样(64x64 纹理在坐标 \cw{(u,v)} 处的查找可以优化为 \cw{(u $\ll{}$ 6) + v})。\\
\par
得益于 8 位调色板(CLUT)纹理支持,GPU 很适合复刻 VGA 子系统:使用一个 CLUT 作为全局调色板。光照衰减通过每顶点颜色混合到 Display Area 中,实现类似前面提到的扇区着色。\\
\par
尽管具备诸多优势,GPU 的一个特性几乎让移植夭折。


\fq{我和 Aaron Seeler 一起做过 Nintendo 64(完全不同的游戏)和 PlayStation 的版本。这是第一次不是“贴着硬件写”,因为 Sony 和 Nintendo(至少对第三方)强制开发者通过 API 而不是直接访问硬件寄存器文档。SGI 的文化一开始让开发者很不适应,但 Nintendo 最终有所放宽。\\
\par
说个 PlayStation 开发的趣事:Aaron 和我起初采用一种不同的引擎架构,用三角形渲染世界,因为那是硬件加速。这样在 N64 上运行很好,它有亚像素精度、透视正确的渲染(SGI 影响)。但 PlayStation 使用整数坐标、仿射纹理映射,大墙和地板三角形看起来糟糕透顶。}{John Carmack}\\
\par
仿射纹理映射是在屏幕空间进行纹理插值而不考虑透视的过程。感谢 \cw{doomworld.com} 的用户 \cw{Lollie},我们可以看看 \doom{} 若使用错误纹理映射会变成什么样。\\
\par
\fullimage{affine_texture_mapping/hallway.png}\\
\par
问题在左侧墙面尤为明显:黑色条带不再与地面平行,而是上下锯齿起伏。

% To understand the issue better, let's take the example of the projection and rasterization of a quad. It is first cut into two triangles. For each triangle, all three vertices \cw{(x,y,z)} are projected into screenspace and only \cw{X=x/z,Y=y/z} remain. For each line of pixels (called scanlines), texture coordinates are interpolated linearly between line start and line end.\\
% \par
% This algorithm gives the weird-looking result featured in figure \ref{affine_texture_mapping/affine2.png}. Notice how all lines are parallel and there is no perspective (which is the main issue).\\
从前面的截图可以看出有问题,但不清楚具体是什么。显然有扭曲,却又不完全像第 \pageref{affine_texture_examples} 页的例子。不同之处在于我们之前可直接画四边形,而 PlayStation GPU 只能处理三角形。\\
\par
没有四边形支持,开发者必须把一切拆成三角形。绘制一面墙就得放置两个相邻三角形。\\
\par
\cfullimage{perspective_psx_explained.png}{左:仿射纹理。右:透视正确纹理}
\par
在图 \ref{perspective_psx_explained.png} 中,左侧墙面显示 PSX 接收到两个三角形,在屏幕空间执行仿射纹理,未考虑观察点距离。结果与应有的效果差异明显,右侧墙面展示了正确透视。\\
\par
为了光栅化这些三角形,GPU 只能使用扫描线算法。该过程保持线条平行,并且两个三角形之间似乎没有“协议”,导致令人不悦的“锯齿缝”。\\
\par
% To get texturing right and interpolate a vertex attribute correctly, it first has to be divided by its z-coordinate. Texturing can then happen via linear interpolation. Before sampling the coordinate must be multiplied by Z again. The implementation is expensive since linear interpolation has to be done between \cw{(u/z,v/z)} and the final result has to be multiplied by \cw{z}.\\
% \par
在图 \ref{texture_anmgle} 中,随着墙面角度增大,视觉伪影更加明显。右列里每个格子的宽度始终相同,这是仿射纹理的明显特征,与左侧透视正确的逐渐变窄形成对比。\\ 
\pagebreak




\par
\begin{figure}[H] \centering
\begin{minipage}{\textwidth}
\scaledimage{0.4}{affine_texture_mapping/tex_perp_pc_repeat_nearest_replace.png} \hspace{2cm}
\scaledimage{0.4}{affine_texture_mapping/tex_perp_no_pc_repeat_nearest_replace.png}\\
\par

\scaledimage{0.4}{affine_texture_mapping/tex_pc_repeat_nearest_replace.png} \hspace{2cm}
\scaledimage{0.4}{affine_texture_mapping/tex_no_pc_repeat_nearest_replace.png}\\
\par

\scaledimage{0.4}{affine_texture_mapping/tex_pc_repeat_nearest_replace_sharp.png} \hspace{2cm}
\scaledimage{0.4}{affine_texture_mapping/tex_no_pc_repeat_nearest_replace_sharp.png}\\
\par
\end{minipage}
\caption{透视正确纹理(左) vs 仿射纹理映射(右)}
\label{texture_anmgle}
\end{figure}
\par

\pagebreak


Sony 很清楚该问题,但制造成本阻碍了为 PlayStation 配备透视正确硬件(他们也缺少像 SGI 这样具备深厚图形经验的合作伙伴,这一点深刻影响了 Nintendo 64)。PSX 开发者手册建议的缓解方式是将三角形再细分成更多三角形。\\
\par
这听上去像是在逃避问题,但 PSX 在当时能处理相当高的三角形数量,因此并不糟糕。然而要获得满意的视觉效果,三角形数量必须足够高。\\
\par
\scaledimage{0.95}{affine_texture_mapping/subdivision_sample.png}\\
\par
由于透视问题,Williams Entertainment、John 与 Harry 面临巨大麻烦,有人非常焦虑。

\fq{Aaron 在我们合作的项目中一直压力极大,而这个计划彻底失败让他对项目失败充满恐慌。我只是耸耸肩,说:“先把所有东西备份(那时还没有源码管理!),我们要做完全不同的方案”。\\
\par
我们最终使用硬件来渲染一像素宽的列或行三角形,就像 PC 的汇编代码一样,效果很好。PlayStation 更常见的做法是对几何进行双轴细分,但我一直对 Doom 在 PlayStation 上不那么“晃动”的感觉很满意。}{John Carmack}\\
