\vspace{-20pt}
\section{漏洞}
\doom{} 因为开发过程使用了七种编译器和系统而以稳定著称,但它仍然带着一些 bug 发货。







\subsection{有缺陷的碰撞检测}
一个罕见的碰撞检测 bug 首先由 Colin “cph” Phipps 在他的文章《Shooting Through Things》中揭示并深入解释。\\
\par
\fq{一个怪物离你太近了。你朝它开枪,却没打中。如果你运气不好,它就会杀了你。但你非常确定准星正对着它,离得这么近你不可能打空。也许是链锯枪在捣乱,把子弹都偏到了一边。也许这游戏是群在高中几何课上挂科的菜鸟写的。或者也许在千钧一发的时刻,你真的打偏了——人无完人。\\
\par
好消息是:你可以怪罪工具。}{Colin Phipps}\\
\par
有些敌人非常大,几乎是玩家(16 单位)的十倍,例如 Spider Mastermind(128 单位)。如第 \pageref{E1M1_blockmap} 页所示,\doom{} 使用 blockmap 来加速与物体和墙体的相交检测。如果一个物体恰好在 block 的边缘,而玩家又稍不走运,本该命中的攻击可能会变成未命中。

\fullimage{beasts_sizes.png}
\vspace{-1cm}
\fullimage{beasts_sizes2.png}


\rawdrawing{collision_miss}
\par
在走廊里确实可能打空一个 SpiderDemon。上面的单房间地图中,左侧绿色玩家向右侧一个宽 128 单位的红色敌人开火(可能是 SpiderDemon)。蓝色网格显示 blockmap 的对齐方式。\\
\par
子弹的轨迹线和怪物的半径明显重叠,这本应命中。但实际只会检查 blockmap 的 \cw{0}、\cw{1} 和 \cw{2},因此只会与墙 \cw{D}、\cw{A} 和 \cw{B} 做测试。由于敌人在 block \cw{5} 中,子弹并未穿过该 block,因此命中不会被记录。\\
\par
这个 bug 并不局限于非常大的怪物。它可能发生在任何敌人身上,取决于敌人与玩家的距离以及 blockmap 的对齐方式。






\subsection{黏液拖尾}
当两面墙之间在屏幕空间出现水平缝隙时,就会出现“黏液拖尾”。Visplane 会在缝隙中“泄漏”,导致图形瑕疵。开发期间这是已知问题,但由于截止日期和出现频率较低而从未修复。John Carmack 在 \cw{doombsp} 源码发布时提到过这一点。\\
\par
\fq{这里确实有个 bug,会导致最多 4 像素宽的一列被错序绘制,使得更远处的地板和天花板平面向前“流”得比应该的更远。有时你能在 E1M1 里看到它:看向入口处锯齿房间里平台上的小恶魔。你会看到人行道右侧有一条几像素宽的黏液往下流。要用鼠标稍微抖动一下才能找到位置。如果有人能追查到它,告诉我……
}{John Carmack}\\\par
\fullimage{slime_trail.png}\\
\par
如果知道去哪看,这个问题实际上可以远宽于 4 像素。\\
\par
\fullimage{big_slime_trail.png}
\par
\vspace{10pt}
这一特定的黏液拖尾是由引擎的 visplane 推断系统与 \cw{doombsp} 对地图切分时整数精度有限共同导致的。\\
\par
再次来看一个例子。前两张截图来自游戏第一关 E1M1 的一个锯齿区域,四周是毒绿液体。乍看没有异常,但当我们查看二叉划分时的切分情况,就会出现一个有趣的特殊情形。\\
\par
线段 \cw{A} 被选作分割线。当它与 \cw{B} 与 \cw{G} 相交时,产生 \cw{B1}、\cw{B2}、\cw{G1}、\cw{G2} 四段。新生成的顶点为红色。对于 C 和 E(或者是 F?)就没那么清晰了,我们需要更仔细地看。\\
\par
当我们放大到顶点 \cw{3} 时,会看到分割线在整数坐标之间穿过了 \cw{F}。由于地图顶点坐标以整数存储,这种分割是不可能的。误差很小,因此 \cw{doombsp} 将该顶点当作正好落在分割线上处理。

\begin{minipage}{0.47\textwidth}
\rawscaleddrawing{1}{E1M1_slimetrail}
\end{minipage}
\hspace{4mm}
\begin{minipage}{0.47\textwidth}
\rawscaleddrawing{1}{E1M1_slimetrail_split}
\end{minipage} 
\par
\vspace{1mm}
\rawdrawing{E1M1_slimetrail_zoom}
\par



\fullimage{leak_opposite.png}
\par
\begin{wrapfigure}[20]{r}{0.41\textwidth}
\centering
\includegraphics[width=.41\textwidth]{drawings/E1M1_slimetrail_split2.pdf}
\end{wrapfigure}
\par
\vspace{10pt}
让玩家站在刚才研究的同一条分割线附近,但朝相反方向看(场景更简单、墙更少)。玩家非常靠近 \cw{E} 和 \cw{F} 段(渲染错误发生处)。更远处是 \cw{G1} 和 \cw{G2} 段,它们来自 \cw{G} 被切分后的结果。\par
\vspace{10pt}
该帧从空白画布开始。首先渲染 portal \cw{E}。它没有上部或中部贴图,但有下部贴图要绘制(在 \pageref{leak_opposite_explained.png} 页用绿色覆盖标注)。渲染下部贴图时,其下方的屏幕空间会被推断为地板 visplane \cw{E-VP}(用粉色覆盖表示)。引擎随后继续向下遍历 BSP 并渲染 \cw{G1}。\par
\vspace{10pt}

\cw{G1} 的下部贴图被渲染(红色)。其下方的一切被推断为地板,于是生成 visplane \cw{G1-VP}。注意该 visplane 一直延伸到底部,这是渲染错误。若 BSP 被正确切分,\cw{F1} 会在 \cw{G1} 之前渲染,阻止 visplane 在屏幕空间中错误泛滥。\\
\par
 

\fullimage{leak_opposite_explained.png}\\
\label{leak_opposite_explained.png}
\par
从这里开始,问题就已经发生了。引擎渲染 BSP 的另一侧,遇到 \cw{F},并在 \cw{G1} 设定的垂直边界上进行裁剪后绘制(蓝色)。\cw{F} 下方被推断为地板,生成 visplane \cw{F-VP}(青绿色)\footnote{引擎实际上会合并 \cw{E-VP} 与 \cw{F-VP},为简化说明此处忽略。}。


\section{油桶自杀}
\doom{} 的怪物可以互相厮杀。在战斗中若发生误伤,受伤的怪物会自动转而反击伤害来源,而不是继续攻击玩家。这被称为怪物内讧(monster infighting)。\\
\par
有一个有趣的特殊情况与爆炸桶有关。当桶受到伤害时,引擎会“记住”是谁造成了伤害。桶真正爆炸时,会把伤害来源传递给所有受伤实体,以便它们反击。\\
\par
可能发生的情况是:怪物引爆桶并被爆炸波及受伤。在这种情况下,带近战攻击的怪物(如 Cacodemon、Imp、Hellknight)会“把自己撕碎”,而只有远程攻击能力的怪物(人类僵尸)则会发疯,盲目乱射,可能引发进一步的怪物内讧。









\section{“始终奔跑”溢出}
令玩家沮丧的是,\doom{} 引擎并没有“始终奔跑”的切换项。不过玩家发现了一个技巧:打开 \doom{} 配置文件,把 \cw{joyb\_speed} 的值手动设为 \cw{29},就能始终奔跑。\\
\par
如果这还没让你大喊“缓冲区越界读”,那 \cw{29} 仅在 DOOM v1.9 有效,而 Final DOOM 需要 \cw{31} 的事实应该足以说服你。\\
\par 
幕后发生的事情是:检查摇杆按钮数组时没有边界检查。将值设为一个不存在的按钮编号后,引擎会读取摇杆按钮数组之外的内存。\\
\par
\ccode{speed.c}
\par
要理解 \cw{29}/\cw{31} 这两个值落在哪里,我们得看变量是如何声明的。\\
\par
\ccode{joybuttons.c}
\par
它到底读到了哪里?由于 \cw{joybuttons} 初始化为 \cw{joyarray} 的第二个条目,我们还需要知道 \doom{} 中 \cw{bool} 的大小。\\
\par

\ccode{typedef_bool.c}
\par
由于代码写于 C/C++ 标准引入 \cw{bool} 之前,\doom{} 使用 enum,其中每个元素都是 4 字节整数。这意味着 \cw{29}/\cw{31} 会落在 \cw{savedescription} 内部,它存放游戏存档时的名称。更可能的是,在 \cw{savegameslot} 之前曾经有别的东西,使得 \cw{29}/\cw{31} 落在 \cw{savedescription} 的字符指针上。它必定非零,从而持续启用“奔跑模式”。\\
\par








\trivia{为了让游戏更有趣,John Romero 把 DOOM 角色的移动速度调得很高。现实中他们可以走到 21.5 英里/小时,跑到 43 英里/小时\footnote{来源:https://www.doomworld.com/vb/doom-general/5909-so-just-how-fast-does-the-marine-run/}!}
