\vspace{-20pt}
\section{缺陷}
\doom{} 以稳定性著称,这部分归功于使用七种编译器与系统的开发流程。尽管如此,它仍带有一些 bug。







\subsection{有缺陷的碰撞检测}
一个罕见的碰撞检测 bug 最早由 Colin "cph" Phipps 在文章《Shooting Through Things》中揭示并深入解释。\\
\par
\fq{怪物靠得太近让你不安。你朝它开火,却打偏了。如果你运气不好,它会杀掉你。但你明明确信自己正对着怪物,而且距离这么近怎么可能打偏?也许是机枪在捣乱,把子弹全打向一侧。也许这游戏是一些高中几何不及格的笨蛋写的。或者,也许在那一瞬间你真的打偏了——没人是完美的。\\
\par
好消息是:你可以怪你的工具。}{Colin Phipps}\\
\par
有些敌人非常巨大,几乎是玩家(16 单位)的十倍,比如 Spider Mastermind(128 单位)。正如我们在第 \pageref{E1M1_blockmap} 页看到的,\doom{} 使用 blockmap 加速对物体与墙体的相交检测。如果一个物体处在 block 边缘,而玩家又有点“倒霉”,本该命中的射击可能会判定为未命中。

\fullimage{beasts_sizes.png}
\vspace{-1cm}
\fullimage{beasts_sizes2.png}


\rawdrawing{collision_miss}
\par
在走廊里甚至可以打偏一只 SpiderDemon。在上面的单房间示意图中,左侧的绿色玩家朝右侧宽 128 单位的红色敌人开火(可能是 SpiderDemon)。蓝色网格显示 blockmap 的对齐。\\
\par
子弹轨迹与怪物半径明显重叠,这本应命中。但只会检查 blockmap 中的 \cw{0}、\cw{1} 和 \cw{2},因此只与墙 \cw{D}、\cw{A} 与 \cw{B} 做相交测试。由于敌人在 block \cw{5} 中且子弹没有穿过该 block,命中就没有被记录。\\
 \par
该 bug 并不限于巨型怪物,任何敌人都可能触发,这取决于它们与玩家的距离以及 blockmap 的对齐方式。






\subsection{史莱姆拖尾}
当两面墙之间存在水平屏幕空间缝隙时,会出现“史莱姆拖尾”。Visplane 会在缝隙间“泄漏”,导致图形瑕疵。这是开发期间已知的问题,但因期限与其罕见性而未修复。John Carmack 在 \cw{doombsp} 源码发布时提到过。\\
\par
\fq{这里确实有个 bug,会导致最多四个像素宽的列以错误顺序绘制,让更远处的地板与天花板平面向前“流”得太多。你有时可以在 E1M1 中朝着之字形房间入口平台上的 Imp 看去时看到。几像素宽的史莱姆列会沿着走道右侧流下来。需要稍微摆动鼠标才能找到位置。如果有人能追踪到它,请告诉我……
}{John Carmack}\\\par
\fullimage{slime_trail.png}\\
\par
实际上,如果知道观察点,这个问题会远远超过四个像素宽。\\
\par
\fullimage{big_slime_trail.png}
\par
\vspace{10pt}
这个特定的史莱姆拖尾,是引擎的 visplane 推断系统与 \cw{doombsp} 在切割地图时整数精度有限共同作用的结果。\\
\par
再次来看一个例子。前两张截图来自游戏第一张地图 E1M1,在一段被毒绿色液体包围的之字形区域。乍看没有异常,但当我们查看它在二叉分割时的切割方式,会出现一个有趣的特殊情况。\\
\par
线段 \cw{A} 被选为分割线。当它穿过 \cw{B} 与 \cw{G} 时,产生了 \cw{B1}、\cw{B2}、\cw{G1} 和 \cw{G2} 四段。新生成的顶点以红色标出。对于线段 C 与 E(或是 F?)的情况就没那么明确,我们需要更仔细地看。\\
\par
放大顶点 \cw{3} 可见,分割线在整数坐标之间穿过 \cw{F}。由于地图顶点坐标以整数存储,无法进行精确切分。误差很小,因此 \cw{doombsp} 认为该顶点正好在分割线上。

\begin{minipage}{0.47\textwidth}
\rawscaleddrawing{1}{E1M1_slimetrail}
\end{minipage}
\hspace{4mm}
\begin{minipage}{0.47\textwidth}
\rawscaleddrawing{1}{E1M1_slimetrail_split}
\end{minipage} 
\par
\vspace{1mm}
\rawdrawing{E1M1_slimetrail_zoom}
\par



\fullimage{leak_opposite.png}
\par
\begin{wrapfigure}[20]{r}{0.41\textwidth}
\centering
\includegraphics[width=.41\textwidth]{drawings/E1M1_slimetrail_split2.pdf}
\end{wrapfigure}
\par
\vspace{10pt}
让玩家沿我们刚研究的分割线站位,但面朝相反方向(场景更简单、墙更少)。玩家非常接近 \cw{E} 与 \cw{F} 段(也就是渲染误差发生处)。更远处是 \cw{G1} 与 \cw{G2},它们来自线段 \cw{G} 的切分。\par
\vspace{10pt}
画面从空白开始。首先渲染 \cw{E} 门洞。它没有上部或中部贴图,但有下部贴图,并被绘制(在第 \pageref{leak_opposite_explained.png} 页用绿色叠加标注)。绘制下部贴图时,屏幕下方区域被推断为地板 visplane \cw{E-VP}(粉色叠加)。引擎随后继续遍历 BSP 并渲染 \cw{G1}。\par
\vspace{10pt}

\cw{G1} 的下部贴图被绘制(红色)。其下方全部区域被推断为地板,于是创建 visplane \cw{G1-VP}。注意它一直延伸到屏幕底部,这是渲染 bug。若 BSP 分割正确,\cw{F1} 会在 \cw{G1} 之前绘制,从而阻止 visplane 在屏幕空间中错误“泛滥”。\\
\par
 

\fullimage{leak_opposite_explained.png}\\
\label{leak_opposite_explained.png}
\par
从这里开始,伤害已经造成。引擎渲染 BSP 的另一侧并碰到 \cw{F},它被 \cw{G1} 设定的垂直边界裁剪后绘制(蓝色)。\cw{F} 下方的空间被推断为地板并生成 visplane \cw{F-VP}(青绿色)\footnote{引擎实际上会合并 \cw{E-VP} 与 \cw{F-VP},但为简化说明暂不考虑。}。


\section{油桶自杀}
\doom{} 的怪物会互相攻击。当战斗中发生误伤时,被击中的怪物会自动反击对方而非继续攻击玩家,这被称为“怪物内斗”。\\
\par
有一个涉及爆炸桶的有趣特殊情况。当桶受到伤害时,引擎会“记住”责任实体。当桶真正爆炸时,伤害来源会传递给所有受伤实体,使其进行报复。\\
\par
可能发生的情况是:某怪物触发桶爆炸并被爆炸波及受伤。此时有近战攻击的怪物(Cacodemon、Imp、Hellknight)会“撕裂自己”,而只能远程攻击的怪物(前人类)会发疯乱射,可能进一步引发怪物内斗。








\section{Always Run 溢出}
让玩家遗憾的是,\doom{} 引擎里没有“始终奔跑”的开关。但玩家发现了一个技巧:在 \doom{} 配置文件中手动把 \cw{joyb\_speed} 设为 \cw{29},就能一直奔跑。\\
\par
如果这还不够像“缓冲区越界读取”,那么 \cw{29} 仅对 DOOM v1.9 有效,而 Final DOOM 需要 \cw{31} 的事实应该足以让你信服。\\
\par 
其内部原因是检查摇杆按钮数组时没有边界检查。把值设为不存在的按钮编号时,引擎会读到数组之外。\\
\par
\ccode{speed.c}
\par
要理解 \cw{29}/\cw{31} 对应的位置,需要看看变量声明方式。\\
\par
\ccode{joybuttons.c}
\par
它到底读到了哪里?\cw{joybuttons} 从 \cw{joyarray} 的第二项开始初始化,因此还需要知道 DOOM 中 \cw{bool} 的大小。\\
\par

\ccode{typedef_bool.c}
\par
由于代码编写时 C 和 C++ 标准都没有 \cw{bool},\doom{} 使用 enum,每个元素是 4 字节整数。这意味着 \cw{29}/\cw{31} 会落到 \cw{savedescription} 内部,它保存存档时的名称。很可能在 \cw{savegameslot} 之前曾有其他字段,使 \cw{29}/\cw{31} 落在 \cw{savedescription} 的字符指针上。该指针必定非零,因此持续启用“奔跑模式”。\\
\par







\trivia{为了让游戏更有趣,John Romero 提高了 DOOM 角色的移动速度。在现实中他们能以 21.5 英里/小时行走、以 43 英里/小时奔跑\footnote{来源:https://www.doomworld.com/vb/doom-general/5909-so-just-how-fast-does-the-marine-run/}!}
