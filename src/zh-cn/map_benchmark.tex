截至 2018 年,距离 \NeXT{} 在 1993 年位于 Redwood City 的工厂产出最后一台机器已经过去了 25 年。如今要找到一台仍能工作的黑色硬件已十分罕见。\\
\par
由于本书力求历史准确,找到一台真正的 NeXTstation TurboColor 变得至关重要——首先用于记录当时的开发条件,其次也为了亲眼见证完整的游戏流水线运行。尽管热心而敬业的程序员们制作了名为 “Previous” 的精美模拟器,但其性能数据并不准确。\\
\par
我在 eBay 上走运,找到了所需的配置。机器能正常工作,但 SCSI 硬盘发出咔哒声,这是濒死迹象。此外,MegaDisplay 的颜色也褪色了\footnote{这可以通过更换显示器控制板上的电容轻松修复,但无法解决重量问题。},而它 50 磅(23 kg)的重量使移动十分困难。\\
\par
多亏 Black Hole Inc. 的创始人兼老板 Rob Blessin,我将 HDD 更换为 SCSI2SD 的 SD 卡,提供类似的访问时间。\NeXTns{} 那套“绿同步”的奇特视频规格使得很难找到兼容显示器,但得益于 \cw{www.nextcomputers.org} 的热心人士,我找到了 NEC MultiSync 1980SX,运行完美。\\
\par
言语难以形容听到机器风扇嗡鸣的感觉。看到 \cw{Doom.app}、\cw{DoomED} 和 \cw{doombsp} 无误编译。看到这台 NeXTstation Turbo Color(序列号 \cw{\#ABC0053943})重获新生。这台机器没有像 Jobs 希望的那样治愈癌症,但它确实给无数开发者带来快乐。





\section{开发游戏}
这一跨页重现了典型的开发者桌面。注意 “Interceptor VGA Console”,它暴露了 \cw{libinterceptor.a}——\NeXTns{} 工程师提供的私有库,用于在 Display Postscript 中打开一个“洞”,绕过“缓慢”的合成器。\\
\par
\cscaledimage{0.9}{doom_on_next.png}{NeXTSTEP 开发环境(屏幕左侧)}

MegaDisplay 的分辨率 1120x832 非常高,id Software 不得不为游戏窗口实现 2 倍软件缩放。否则 \doom{} 窗口看起来像一枚小邮票,几乎看不到像素。\\
\par
\vspace{18.5pt}
\cscaledimage{0.92}{doom_on_next2.png}{NeXTSTEP 开发环境(屏幕右侧)}
\pagebreak

\section{编译地图}
\doom~与 \doomii 中每一关的 \cw{doombsp} 运行时间基准\footnote{基于 John Romero 于 2015-04-22 发布的 \cw{.map} 文件。}。\\
\par
 \begin{minipage}[t]{0.45\textwidth}
 \begin{figure}[H]
\centering  
\begin{tabularx}{\textwidth}{ L{0.3} | R{0.7} }
  \specialrule{1pt}{0pt}{0pt}
  \textbf{Map} & \textbf{\cw{doombsp} 运行时间 (s)} \\
  \specialrule{1pt}{0pt}{0pt}
E1M1 &     8.2 \\ 
E1M2 &       32.0 \\
E1M3 &       26.2\\
E1M4 &       18.4\\  
E1M5 &       19.9\\
E1M6 &       44.0\\
E1M7 &       22.3\\
E1M8 &        6.9\\
E1M9 &       15.4\\
E2M1 &        6.0\\
E2M2 &        55.4\\
E2M3 &        19.6\\
E2M4 &        36.0\\
E2M5 &        46.8\\
E2M6 &        32.5\\
E2M7 &        60.8\\
E2M8 &         2.5\\
E2M9 &         1.5\\
E3M1 &        2.5\\
E3M2 &        9.2\\
E3M3 &       38.1\\
E3M4 &       23.7\\
E3M5 &       34.5\\
E3M6 &       22.5\\
E3M7 &       23.4\\
E3M8 &        1.9\\
E3M9 &        8.9\\
   \specialrule{1pt}{0pt}{0pt}
\end{tabularx}
%\caption{Video system interface}
\end{figure}
\end{minipage}
\hspace{1cm}
\begin{minipage}[t]{0.45\textwidth}
 \begin{figure}[H]
\centering  
\begin{tabularx}{\textwidth}{ L{0.3} | R{0.7} }
  \specialrule{1pt}{0pt}{0pt}
  \textbf{Map} & \textbf{\cw{doombsp} 运行时间 (s)} \\
  \specialrule{1pt}{0pt}{0pt}
MAP01 &       6.1  \\
MAP02 &       6.6 \\
MAP03 &       8.7 \\
MAP04 &       8.5 \\
MAP05 &       17.6\\
MAP06 &       25.0\\
MAP07 &       1.9 \\
MAP08 &       15.2\\
MAP09 &       16.3\\
MAP10 &       34.0\\
MAP11 &        15.7 \\
MAP12 &        15.2\\
MAP13 &        31.5\\
MAP14 &        44.7\\
MAP15 &        66.0\\
MAP16 &        16.2\\
MAP17 &        36.2\\
MAP18 &        17.2\\
MAP19 &        45.8\\
MAP20 &        29.2\\
MAP21 &        5.7 \\
MAP22 &        9.4 \\
MAP23 &        7.5 \\
MAP24 &       30.5 \\
MAP25 &       21.1 \\
MAP26 &       18.8 \\
MAP27 &       26.2 \\
MAP28 &       19.6 \\
MAP29 &       45.8 \\
MAP30 &        1.0 \\
MAP31 &       16.4 \\
MAP32 &        2.7 \\
MAP33 &        6.6 \\
MAP34 &        9.3 \\
MAP35 &        0.3 \\
   \specialrule{1pt}{0pt}{0pt}
\end{tabularx}
%\caption{Video system interface}
\end{figure}

\end{minipage}
\\

\section{运行游戏}
在 NeXTstation TurboColor 上运行 \doom{} 的帧率出奇地差。\\
\par
 \begin{figure}[H]

\centering  
\begin{tabularx}{\textwidth}{ L{0.08} | C{0.42} | R{0.25} | R{0.25} }
  \specialrule{1pt}{0pt}{0pt}
  \textbf{模式} & \textbf{分辨率} & \textbf{高细节 FPS} & \textbf{低细节 FPS} \\
  \specialrule{1pt}{0pt}{0pt}
B & 320x200 & 9 & 13 \\  
A & 320x168&  9 & 14 \\
9 & 288x144& 11 & 15 \\
8 & 256x128& 12 & 16 \\
7 & 224x112& 13 & 17\\
6 & 192x096& 15 & 19 \\
5 & 160x080& 17 & 20 \\
4 & 128x064& 19 & 22 \\
3 & 096x048& 21 & 23 \\
   \specialrule{1pt}{0pt}{0pt}
\end{tabularx}
\caption{\protect\doom{} 在 NeXTstation TurboColor 上的帧率}
\end{figure}
\par
在开发期间使用的 2 倍缩放模式下,情况更糟。在该模式中,写入核心帧缓冲的像素数相同,但需要通过总线传输的数据量增加了四倍。\\
\par
 \begin{figure}[H]
\centering  
\begin{tabularx}{\textwidth}{ L{0.08} | C{0.42} | R{0.25} | R{0.25} }
  \specialrule{1pt}{0pt}{0pt}
  \textbf{模式} & \textbf{分辨率} & \textbf{高细节 FPS} & \textbf{低细节 FPS} \\
  \specialrule{1pt}{0pt}{0pt}
B & 640x400 & 6 & 8 \\  
A & 640x336 & 6 & 8 \\
9 & 576x288 & 7 & 9 \\
8 & 512x256 & 8 & 9 \\
7 & 448x224 & 8& 10\\
6 & 384x192 & 9 & 10 \\
5 & 320x160& 9 & 10 \\
4 & 256x128& 10 & 11 \\
3 & 192x096 & 11  & 11 \\
   \specialrule{1pt}{0pt}{0pt}
\end{tabularx}
\caption{\protect\doom{} 在 NeXTstation TurboColor 上的 2 倍缩放帧率}
\end{figure}
\par
降低分辨率或细节级别略有帮助,但不像 DOS 版本那样明显。因为在 \NeXTns{} 上视频系统的实现方式不同。\\
\par
实现忽略来自 \cw{I\_UpdateNoBlit} 的更新信号,而把所有工作延后到 \cw{I\_FinishUpdate},在那里把帧缓冲 \#0 的完整内容 blit 到 \cw{NSWindow}。没有脏矩形优化,也没有像 DOS 那样直接访问硬件。\\
\par
\trivia{“低细节”模式在 NeXTSTEP 上从未正确实现。引擎只写一半的列,但没有像 VGA bank mask 那样的复制机制。结果是只更新了 \cw{NSWindow} 的左侧部分。}

\section{帧缓冲无畸变}
NeXTstation 拥有“干净”的视频系统:色彩空间线性,像素为“正方形”(帧缓冲与 MegaDisplay 监视器具有相同的纵横比)。因此 \doom{} 的帧缓冲 \#0 在窗口系统中显示时没有畸变。\\
\par
320x200\footnote{320x200 是活动区域的尺寸,不含标题栏。} 的 “Interceptor VGA Console” \cw{NSWindow} 不会被拉伸到 320x240,因此看起来纵向被压扁。特别是在 NeXTSTEP 的启动画面(图 \ref{doom_crushed})与 DOS 版本(图 \ref{doom_title_4_3})并排显示时尤为明显。\\
\par
\cscaledimage{1}{doom_crushed}{\doom{} 在 NeXTSTEP 上显示。内容垂直方向被压扁}
\par
\trivia{许多移植版搞错了纵横比。\doom{} 95 也是如此,它本用于展示微软 Windows 95 图形驱动的低开销。在默认设置下,320x200 的宿主窗口直接映射 \doom{} 的核心帧缓冲。结果敌人看起来更矮,火箭爆炸呈椭圆,其他一切都变形。}
\par
\cscaledimage{1}{doom_title_4_3}{\doom{} 在 1993 年 PC 显示器上呈现的 4:3 纵横比}
