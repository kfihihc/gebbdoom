1994 年,Jonathan Mendoza 出版《Doom Survivor's Strategies \& Secrets》时,id Software 的三名成员撰写了各自领域的文章:John Carmack 写引擎,Sandy Petersen 写关卡设计,Kevin Cloud 写美术。经 Jonathan Mendoza 授权,这些内容在此重现。\\
\par
\section{John Carmack}
\subsection{目标}
\b{高速} Doom 是交互式游戏,因此应以每秒 10 帧或更快的速度运行。我们的目标用户是 386/33 及更快的机器。通过可选的细节与屏幕大小,较慢的电脑可以用画面精度换取速度。在高端,一台快速的 Pentium 在大多数情况下可以以 35 fps(最高速度)运行 Doom。\\
\par

\b{自由形态的世界几何} 我们以前的游戏都是“基于格子”的,也就是说,世界由固定大小的方块组成,从预先创建的数据调色板中选择。基于格子的世界好处是可以快速、轻松地创建,简单方块重复使用即可。但如果关卡包含大量独特区域或斜角走廊,就需要成千上万的小几何格子。我们希望能在不受块状世界限制的情况下设计关卡。\\
\par 
\b{无限视距} 多数 3D 游戏遵循原则:只绘制一定距离内的对象。这简化了渲染算法,但迫使视距要么迅速淡出(Ultima Underworld/Shadowcaster),要么出现物体突然弹出的突兀效果(大多数飞行模拟和驾驶游戏),而不是先以远处小点出现再逐渐变大。
\par
\subsection{实现}
Doom 的编程工作可以大致分为四个同等部分:
\begin{itemize}
\item 开发渲染引擎以绘制世界环境画面。
\item 开发用于为游戏创建数据的工具。
\item 开发世界模型,用于管理游戏世界中各事物的交互。
\item 根据新情况不断调优和修改代码。
\end{itemize}

主游戏代码不到 30,000 行 C 代码。DOS 版有三个汇编函数:水平纹理映射、垂直拉伸、读取摇杆。声音代码由外部承包商开发。我们开发策略的一个关键点是:几乎所有编程工作都在 NEXTSTEP 系统上完成。强大稳定的开发环境让我们能做出比受限于 DOS 时更丰富的工作。\\
\par
游戏结构允许它在 NEXTSTEP 下以窗口运行便于调试,或重新编译为 DOS 下全屏运行。渲染引擎实际上主要是在我家的一台黑白 NeXTStation 上开发的。它被设计为可在灰度、8 位颜色或 12 位双色(彩色 NeXTstation 原生)下绘制图形。刷新也可在任意分辨率下使用,不限于 PC 的屏幕尺寸。对可移植代码的纪律性要求,让我对更好的游戏架构有了一些洞见。\\
\par

我通常用三个轴来分类游戏渲染引擎:速度、能力和图像保真度。速度是视窗大小与帧率的关系。能力涵盖世界模型的限制,如是否只能 90 度墙、是否支持斜坡、可变光照、可变视高等。保真度包括纹理映射的准确性、为了速度所做的取巧,以及诸如抗锯齿之类的因素。\\
\par

我们的游戏设计从为目标平台选择速度开始,然后尽可能获得更强能力和更高保真度。Doom 的世界几何仅限于二维线段的墙体与可变高度的平坦地板和天花板。Doom 不能绘制斜坡地面、重叠的走道或倾斜的墙。视角有四个自由度:前/后、左/右、上/下、顺时针/逆时针。这对建筑漫游程序来说是明显限制,但对游戏设计而言给了我们大量自由。在 Doom II 的开发中,我们仍在发现利用这些能力的新方式。我为 Doom 引擎的保真度感到自豪。纹理映射是亚像素级精度,距离上没有妥协。\\
\par

由于 Doom 的几何限制,隐藏面消除问题可以被简化为仅处理墙体的二维问题。地板与天花板在墙体绘制完成后填充剩余空间。这比任意三维渲染方案快得多。Doom 用于隐藏面消除的核心算法是二维二叉空间分割树遍历。地图绘制完成后,会交给一个独立工具,把线段分组成扇区,并递归地将整个地图划分为凸区域。这项任务耗时,但提前完成它能让游戏运行时做更少工作。代价是构成世界的线段端点在游戏过程中无法调整。这就是为什么 Doom 没有摇摆门。\\
\par

我们的地图编辑器被设计师几乎连续使用了一整年,因此投入时间让它高效易用是非常值得的。DoomEd 是我们在 NEXTSTEP 上创建的应用,用于构建和修改世界。它让我们以俯视角设计世界几何,并选择贴图应用到墙、地板和天花板。\\
\par

游戏世界模型从一开始就为联网而设计。世界中的每个对象都会经过相同的例程处理,不论它是奖励物品、怪物还是联网玩家。一些世界工具例程,比如子弹目标与追踪调用,实际上比 3D 渲染例程更复杂。\\
\par

整个项目的调优是打造可玩游戏最重要的阶段。动画的时机、音效的音高、爆炸尸体的运动等细微元素都会影响玩家的印象。正确的调优需要大量时间与测试,但细节至关重要。在 Doom 中,我们经历了一次有趣的协同效应:多个与移动、战斗和环境相关的元素互相补强,使最终游戏比我们最初设想更好。通常的游戏设计过程是从一个辉煌愿景开始,然后随着项目推进逐渐被现实削减。虽然我们原计划被大幅改动,也丢失了一些功能,但最终产品超出了我们早期的期待。\\

\subsubsection{后记}
Doom 是我第一个在完成时仍为之自豪的项目。Wolfenstein 发布时我已经不满意了,而我对 Shadowcaster 引擎的实现也很失望。我看到了一些瑕疵,但对 Doom 的工作仍感满意。刷新代码里还有几个不太可能修复的 bug。有时你会看到一列一像素宽的柱子从屏幕顶端拉到顶端。原因是绘制了一条线,其两个端点被变换到几乎相同的极角。用于计算该柱缩放的定点运算会偶尔溢出,使柱子被拉到最大可能的 64 倍高度。地板或天花板上几乎垂直的切口有时也会出现错误。\\
\par

地图分区器还有一个舍入误差,可能导致一条线的像素宽线段在地板/天花板纹理的狭窄条带中被错误顺序绘制,从而越过本应阻止它的线。有些美术资源绘制得比游戏物体的实际宽度更宽,因此在某些情况下,精灵的部分可能会在墙后被看到。如果我更关注 Intel 底层架构,我本可以让 Doom 快 15\%。我们在 Doom 开发中学到了很多,还有许多新东西可以带到下个项目。敬请期待!




\section{Sandy Petersen}


Doom 的关卡并非由单一设计者完成。John Romero 从零开始创建了第一章《Knee Deep in the Dead》的全部关卡,除了第 1.8 关。其余所有关卡都由我完成,有时独立完成,有时把别人的早期作品打磨成更完善的版本。以下段落将给出其余关卡的完整归属说明。\\
\par
尽管对 Tom Hall 和 Shawn Green 的关卡做了大量修改(前者曾在 id Software 工作,后者仍在 id Software),包括放置怪物、修复墙面纹理以及无数细节调整,但基本架构保持不变。\\
\par
我认为,敏锐的 Doom 玩家会感受到每位设计师关卡之间明显的个性差异。对于 Tom Hall 和 Shawn Green,这种效果可能会稍弱,因为他们原本鲜明的风格在大量修改中与我的风格有所融合。\\
\par
John Romero 的特别疯狂之处在于:用看似不可阻挡的怪物潮淹没玩家,并穿插长时间紧张的寂静,让玩家思考下一波恐怖会从何而来。他常把怪物放在远处的制高点,使其能在相对安全的位置狙击玩家。John 的关卡遍布特殊制高点、狡猾的秘密区域,以及多层次的动作设计。\\
\par
John 几乎总以一场噩梦般的血战开场,让玩家肾上腺素飙升。只有撑过这一波攻势,你才能喘息并决定下一步方向。John 的另一个倾向是线性化关卡。如果不算那些秘密通道,你几乎必须按他规定的顺序推进。\\
\par
我的关卡则倾向于持续不断的小规模怪物压力,而非 John 那种阶段性的恐怖爆发。此外,与 John 的恶魔式秘密隧道和平台不同,我更喜欢用陷阱和机关袭击玩家。典型例子是 E2M6 的假出口:看起来像出口,闻起来像出口,但它并不是真正的出口。\\
\par
我的关卡开局往往相对安静,玩家只能靠自己。通常拐角处就有一两个怪物,但不是你从 John 那里习惯的嗜血大群。有些关卡非常线性(如 E3M1 或 E3M4),但另一些,如 E3M2、E2M5 和 E3M6,则几乎完全开放,玩家可以随意探索。我发现一些玩家很喜欢这种自由形态体验,而另一些会感到迷失和困惑,直到找到正确方向(这通常因人而异)。\\
\par
设计关卡时必须始终记住三件事:
\begin{enumerate}
\item 看起来如何?
\item 好玩吗?
\item 记得清理了吗?
\end{enumerate}
\par

\subsection{看起来如何?}

这是我最难学会的关卡设计部分。对 Romero 来说似乎很自然,而我必须不断努力。根本问题是:要为 Doom 设计好看的房间,你必须以建筑视角思考——把房间视为空间,而不是地图上的一堆线。赋予房间动画与色彩的具体墙面纹理往往是次要的,真正重要的是房间的结构组件。有些房间确实看起来非常棒,而另一些即便有色彩和结构也没那么出色。例如,我们对 E1M4 的大入口大厅从未完全满意。它功能上没问题,也很好玩,但就是缺少“劲”。屋顶开洞和对大厅装饰的多次修改也没彻底修复。最终我们决定:它玩起来没问题,就留下它,继续做别的。\\
\par
在 Doom 的早期设计中,我们倾向于做很多曲折的小迷宫。开始试玩后我们发现这通常不太好玩,所以大多被丢弃了(少数例外,主要在 Tom Hall 的旧关卡里)。即便留下的那些,大多数也被改造简化,或者以幽闭恐怖的目的存在(例如 E1M4 的优秀最终迷宫、E3M3 的二层迷宫、或 E3M7 的熔岩迷宫)。

\subsection{好玩吗?}

这当然比“看起来好看”更重要——如果不好玩,那画面再好也没意义。对我来说,做出好玩的关卡需要一个初始总体规划和持续的试玩。\\
\par
当我开始一个关卡时,会长时间思考它的总体主题——玩家应从中获得什么体验。例如,在 E3M5 中,我希望给玩家一种宏大殿堂或神庙的幻觉,拥有对称且易于理解的建筑结构。开始时玩家会因传送门、放出的怪物等感到困惑,但很快就能理解关卡整体结构,并轻松绕场奔跑。一旦玩家理解布局,他们就能以科学、理性的方式攻略 E3M5,这会与该关卡情感上的沉重(巨大大教堂)形成有趣对比。\\
\par
另一方面,在 E3M1 中,目标只是用地狱中等待他们的奇观震撼玩家。关卡从一开始就充满不祥恐怖的画面:你在红色阴沉天空下的室外,被小恶魔追逐。当你打开看似能逃离的门,会放出一只 Cacodemon。通往霰弹枪的桥会坍塌,等等。你被迫不断奔跑,目睹越来越多的诡异与恐怖,很快就会意识到地狱与第一、二章的理性结构关卡截然不同。\\
\par
一旦主题确定,我通常先完成关卡中的一个小区域,然后快速试玩。如果效果好且看起来没问题,就完成下一个区域,再把已完成的区域一起测试,如此循环直至整个关卡完成。\\
\par
\subsection{记得清理了吗?} 

仅仅因为关卡看起来不错、玩起来也好,并不代表完工。现在还必须确认所有细节:弹药和武器是否足够?奖励物品呢?玩家期待它们,而且很容易遗漏。记得标记秘密区域了吗?陷阱和机关够不够让玩家保持兴趣?\\
\par
在覆盖完这些可怜的小细节后,我还要更深入地检查:是否存在聪明的玩家可以绕过整个关卡的大部分内容?如果有,这可以接受吗?(有时可以——如果你很聪明,几乎可以跳过 E3M6 的大部分内容,我并不介意,但你会错过很多武器和有趣的战斗。)该关卡与前一关、后一关的衔接如何?如果 B 关的起始房间只有一只孤零零的 Cacodemon 从长廊向你冲来,而你在 A 关出口房里刚捡到火箭筒,那基本没什么难度。相反,如果火箭筒在 A 关一开始就给你,那么只有你仔细保留火箭弹,到了 B 关才有它可用——这可能是好的,你应当因节俭而得到奖励。\\
\par
当最终关卡完成后,我会再玩几遍,寻找错误和瑕疵(我总能找到不少),然后把它交给那些臭名昭著的“人类失败样本”——id Software 里的其他 id-iots。他们很快会找出我可怜关卡里各种糟糕的问题,我会尽快修复,以免被他们揭露我在某处埋下的瑕疵后遭受尖刻评论和嘲笑。如果我听起来像个怨气十足的人,事出有因。



\section{Kevin Cloud}

在游戏中,优秀的电脑美术常被称为“渲染精美或细节丰富”,因为大多数好看的游戏美术都像精心手绘。然而,精美的渲染世界往往显得布景化。对于 Doom,我们想要一个现实、阴暗、比起漂亮更“脏”的世界。Doom 没有什么“美”,我们希望它的世界传达这一概念——可怕、阴暗。\\
\par

为达到预期效果,我们使用了扫描与手工渲染图像的组合。John Carmack 编写了 Fuzzy Pumper Palette Shop 程序,用于捕捉实时视频图像并转换为 PC 图形格式。然后我们把图像直接导入 PC 美术应用中,进行编辑、缩放、上色、合成——不管需要什么来制作有趣的图形。整体效果略有失真,但这就是 Doom。\\ 
\par
Doom 中的角色通过多种方法制作:手绘、扫描黏土模型,以及最终的乳胶和金属模型。做 Wolfenstein 时,我们就知道为每个角色动画制作旋转视角有多沮丧。大多数角色正面好画,但旋转 45 度就会复杂得多。我们用小木偶和几磅黏土来制作自己的模型。这种技术不完美,但它让我们能把怪物摆出平常不会画的姿态。\\
\par
项目接近尾声时,我们想做一个非双足怪物。我们想象了一只巨大的“脑子怪”,脸上嵌着链锯枪,身体被几根大金属钩子挂在一个四足金属机器上。我们无法用黏土制作这个家伙,于是联系了 Gregor Punchatz。\\
\par
Gregor 在模型制作方面经验丰富。他参与过《猛鬼街》《机械战警》等经典电影的制作,具备制作模型所需的工具和才能。几周内,Gregor 就把我们的草图变成了一个可完全活动的怪物。流程很顺利。虽然这个版本的 Doom 只用了他的一件作品,但零售版 Doom 将会充分利用 Gregor 的才能。

\subsection{HAPPY DOOMING}

显然,Doom 远不止屏幕上所见,而 Doom 的创作者本可以分享更多。我希望以上讨论至少能让你了解 Doom 创作者的思维,并对 Doom 的艺术与科学有所欣赏。想要了解更多关于如何游玩与享受 Doom 的技术信息,请在附录中查找你关心的主题。\\
\par
