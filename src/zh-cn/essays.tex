1994 年,Jonathan Mendoza 出版《Doom Survivor's Strategies \& Secrets》时,id Software 有三位成员撰写了各自领域的文章:John Carmack 写引擎,Sandy Petersen 写关卡设计,Kevin Cloud 写美术。经 Jonathan Mendoza 授权,这些文章在此转载。\\
\par
\section{John Carmack}
\subsection{目标}
\b{高速} Doom 是交互式游戏,因此应以每秒 10 帧或更高的速度运行。我们的目标用户是 386/33 或更快的机器。通过可选的细节与画面尺寸,较慢的电脑可以用画质换取可用的速度。在高端,快速的 Pentium 机器在多数情况下可跑到 Doom 的最高速度 35 fps。\\
\par

\b{自由的世界几何} 我们以前的游戏都是“瓦片式”的,即世界被划分为固定尺寸的块,从预制数据的调色板中选择。瓦片式世界的优势是能快速构建,通过重复简单瓦片即可。但若关卡包含大量独特区域或斜向走廊,就需要成千上万小块几何。我们希望能设计关卡而不被块状世界所束缚。\\
\par 
\b{无限视距} 多数 3D 游戏遵循原则:只绘制一定距离内的物体。这简化了渲染算法,但迫使地平线要么快速淡出(Ultima Underworld/Shadowcaster),要么出现令人不适的“突然弹出”(多数飞行模拟与驾驶游戏),而不是先作为远处小点出现再逐渐变大。
\par
\subsection{实现}
Doom 的编程工作大致可以分成四个相等部分:
\begin{itemize}
\item 开发渲染引擎,绘制世界环境。
\item 开发用于为游戏创建数据的工具。
\item 开发掌控世界中物体交互的世界模型。
\item 随着新情况出现,对代码进行调优与修改。
\end{itemize}

主游戏代码不到 30,000 行 C。DOS 版本中有三个汇编函数:水平纹理映射、垂直拉伸与读取摇杆。音频代码由外包承包商开发。我们的开发策略核心是:几乎所有编程工作都在 NEXTSTEP 系统上进行。这个强大且稳定的开发环境让我们做出远比在 DOS 下更丰富的工作。\\
\par
游戏结构允许在 NEXTSTEP 下窗口运行以便调试,或重新编译为 DOS 全屏运行。渲染引擎大多是在我家的一台黑白 NeXTStation 上完成的。结构上允许以灰度、8 位彩色或 12 位双色(彩色 NeXTstation 原生)绘制图形。刷新也可用于任意分辨率,不局限于 PC 屏幕大小。强迫自己写可移植代码,让我对更好的游戏架构有了一些洞察。\\
\par

我通常从三个维度分类游戏渲染引擎:速度、能力与画面保真度。速度是视窗大小与帧率的关系。能力涵盖世界模型的限制,如仅 90 度墙、斜坡地面、可变光照、视角高度变化等。保真度包括纹理映射精度、为提速所做的取舍以及抗锯齿等。\\
\par

我们的游戏设计从目标平台上的速度开始,然后尽可能争取更多能力与更高保真度。Doom 的世界几何被限制为二维墙线排列,以及不同高度的平坦地板与天花板。Doom 无法绘制倾斜地面、重叠走道或倾斜墙面。视角有四个自由度:前/后、左/右、上/下、顺时针/逆时针。对建筑漫游程序而言这是显著限制,但它为游戏设计提供了巨大的自由度。我们在 Doom II 的开发中仍在寻找利用这些能力的新方式。我为 Doom 引擎的保真度感到自豪:纹理映射具有亚像素精度,距离上没有妥协。\\
\par

由于 Doom 的几何限制,隐藏面消除问题可简化为仅处理墙面的二维问题。地板与天花板在墙面绘制完成后填充剩余空间。这比任意三维渲染方案快得多。Doom 使用的核心隐藏面算法是二维 BSP 树遍历。绘制完地图后,会交给独立工具将线条分组为 sector 并递归将地图分割为凸区域。这很耗时,但预处理后运行时就能做更少工作。代价是世界中构成线条的端点无法在游戏中调整,这就是 Doom 没有摆动门的原因。\\
\par

我们的地图编辑器几乎被设计师日复一日使用了近一年,为了提高其效率付出的努力非常值得。DoomEd 是我们创建的 NEXTSTEP 应用,用于构建与修改世界。它允许从俯视角设计几何,并选择映射到墙、地板与天花板的图形。\\
\par

世界模型从一开始就为了联网而设计。世界中的每个对象都通过相同的例程处理,无论它是奖励物、怪物还是联网玩家。有些世界工具例程,如子弹目标与射线追踪,甚至比 3D 渲染例程更复杂。\\
\par

项目调优是打造优秀游戏最重要的阶段。动画时机、音效音高或爆炸尸体的运动等细节都会影响玩家体验。恰当的调优需要大量时间与测试,但细节很关键。Doom 中存在一种有趣的协同作用:关于移动、战斗与环境的多个元素彼此配合得很好,使游戏比我们最初设想更好。游戏设计的常规流程是从一个辉煌愿景开始,随着项目推进逐步被现实磨平。尽管原计划变化很大、一些特性被舍弃,最终成品仍超出早期预期。\\

\subsubsection{后记}
Doom 是我参与过并在完成时仍感到自豪的第一个项目。Wolfenstein 发布时我已不满意,我对 Shadowcaster 引擎的实现也很失望。我能看到一些瑕疵,但仍为 Doom 的成果感到满意。刷新代码里仍有些 bug 不太可能修复。有时你会看到一条从屏幕顶到底的一像素宽列,这是因为一条线的两个端点被变换到几乎相同的极角,计算该列缩放的定点数溢出,导致列被放大到最大 64 倍高度。接近垂直的地板或天花板切割有时也会显示错误。\\
\par

地图分割器存在舍入误差,会导致一条线的一像素宽片段在地板与天花板纹理的狭窄条带中被以错误顺序绘制,本该阻挡它的线段没有生效。有些美术绘制得比实际对象宽,因此在特定情况下可能看到精灵的一小部分“穿墙”。如果我更关注底层 Intel 架构,本可让 Doom 快三分之一左右。我们在 Doom 的开发中学到了很多,也有许多新东西要带到下个项目里。小心!





\section{Sandy Petersen}


Doom 的关卡并非由单一人设计。John Romero 从零制作了第一章 Knee Deep in the Dead 的所有关卡,除了 1.8。其余关卡由我完成,或独立完成,或将他人早期作品打磨成更精致的版本。下面段落对这些关卡的归属做完整说明。\\
\par
尽管对 Tom Hall 与 Shawn Green 的关卡做了大量改动(前者曾是 id Software 成员,后者仍在 id),包括放置怪物、修复墙面贴图与调整无数细节,基本结构仍保持不变。\\
\par
我相信敏锐的玩家能感受到不同设计者关卡之间的个性差异。在 Tom Hall 与 Shawn Green 的情况下,这种差异可能被淡化,因为他们独特风格在大量编辑后与我的风格有所融合。\\
\par
John Romero 的疯狂之处在于将玩家淹没在看似无穷无尽的怪物潮中,中间夹杂着漫长而紧张的寂静期,让玩家思考下一波恐怖会是什么。他经常把怪物放在远处制高点,让它们相对安全地狙击玩家。John 的关卡充满特殊视角点、巧妙的秘密区域与多层动作。\\
 \par
John 几乎总是以噩梦般的血洗开场来点燃玩家的肾上腺素。只有熬过这一波,你才能喘口气并决定下一步去哪儿。John 的另一个倾向是让关卡线性化。如果不算秘密通道,你几乎必须按他规定的顺序通关。\\
  \par
在我的关卡里,我倾向于让怪物持续不断地出现,而不是 John 那样的阶段性恐怖爆发。并且我更喜欢用陷阱与圈套来攻击玩家,而不是 John 的恶魔秘密通道与平台。经典例子是 E2M6 的假出口:看起来像出口、闻起来也像出口,但其实不是出口。\\
   \par
我的关卡开头通常比较安静,让玩家独自行动。通常拐角就有一两只怪物,但不是你可能从 John 那里学会期待的怪物大军。我的一些关卡很线性(比如 E3M1 或 E3M4),但另一些如 E3M2、E2M5 与 E3M6 则几乎完全开放,玩家可以随意探索。我发现有些玩家很喜欢这种自由体验,而另一些会感到迷失,直到找到正确路线(通常因人而异)。\\
    \par
在为玩家设计关卡时必须时刻记住三件事:
\begin{enumerate}
\item 看起来怎么样?
\item 好玩吗?
\item 你记得清理了吗?
\end{enumerate}
\par

\subsection{看起来怎么样?}

这是关卡设计中最难学的部分,至少对我而言。Romero 似乎天生就会,而我必须反复练习。基本问题是:要设计出好看的 Doom 房间,你必须以建筑思维来思考,也就是把房间看作空间,而不仅是地图上的线条。用于赋予房间动画与颜色的墙面纹理常常次于房间的结构组件。有些房间看起来非常出色,另一些尽管颜色与结构不差,却不够震撼。例如,我们一直不满意 E1M4 的大入口厅。它完成了任务,也玩起来很有趣,但总觉得缺少那股劲。屋顶的巨大开口和大量装饰改动也没能彻底解决。最终我们决定它玩起来没问题,就放过它继续做别的。\\
\par
Doom 早期设计中有很多蜿蜒迷宫。开始试玩后我们发现这通常不太好玩,于是大多被删掉(少数保留下来,主要在 Tom Hall 的旧关卡里)。即便保留的部分也大多被改动简化,或作为狭窄恐怖的目的存在(比如 E1M4 末尾的优秀迷宫、E3M3 的楼上迷宫、或 E3M7 的熔岩迷宫)。

\subsection{好玩吗?}

当然,这比“看起来好”更重要——如果不好玩,那再好看也没意义。对我来说,做出好玩的关卡是整体规划与不断试玩的结合。\\
\par
当我开始一个关卡时,会反复思考它的总体主题——玩家应从中获得什么。例如在 E3M5,我想给玩家一种宏大的圣堂或神殿错觉,具备对称且可理解的建筑。玩家起初会被传送器、放出的怪物等弄得困惑,但很快就理解整体结构,随后以从容的方式跑遍全图。一旦玩家理解布局,他们便能以更科学、理性的方式处理 E3M5,这让他们在情感性极强的关卡(巨大教堂)与自身行为之间形成有趣对比。\\
 \par
另一方面,E3M1 的目标就是用地狱中等待他们的奇观震慑玩家。关卡从一开始就充满阴森可怖的景象:你在愤怒的红色天空下被 Imp 追逐,打开看似有希望的门后却放出 Cacodemon。通往霰弹枪的桥坍塌等等。你不断奔跑,看到越来越多诡异恐怖的景象,这些快速教会你地狱的不同性质,相比前两章更理性构建的关卡。\\
  \par
一旦主题明确,我通常会先完成关卡中的一小块区域,然后快速试玩。如果感觉可行且看起来不错,就完成下一块,并把已完成区域一起测试,如此反复直到完成整个关卡。\\
   \par
    \subsection{你记得清理了吗?} 

    关卡看起来好、玩起来好并不代表完成。我还得确保考虑到一切:弹药和武器是否足够?奖励物品呢?玩家会期待它们,很容易遗漏。秘密区域是否标记?陷阱和小把戏够多以保持玩家兴致吗?\\
    \par
     覆盖了这些琐碎细节后,我还得更深入思考:聪明的玩家是否能绕过整个关卡的战斗?如果可以,这是否可以接受?(有时可以——如果你够聪明就能跳过几乎整个 E3M6,我不介意;但你会错过许多武器与有趣战斗。)关卡与前一关、后一关的衔接如何?如果 B 关的起始房间里只有一只 Cacodemon 沿长廊冲来,而 A 关出口房间给了你火箭筒,那就没什么难度。相反,如果火箭筒在 A 关一开始就给你,那么只有你小心保留火箭,到了 B 关才有它,这就合理——你应当为节省而得到回报。\\
     \par
      当最后关卡完成后,我会再玩几遍,找出缺陷与错误(我总能找到不少),然后把关卡交给那些“烂到不配做人”的家伙——id Software 的其他 id-iots。他们很快就能找到我这可怜宝贝关卡里各种糟糕问题,我得尽快修复,以避免他们揭露我埋下的缺陷后出现的尖刻评论与冷笑。如果我听起来像个苦涩的人,那是有原因的。



\section{Kevin Cloud}

在游戏中,优秀的电脑美术通常被称为“渲染精美或细节丰富”,因为大多数优秀游戏美术看起来都像手工精心绘制。不幸的是,精美渲染的世界往往显得像布景。对于 Doom,我们想要一个真实、黑暗、看起来更脏而不是更美的世界。Doom 没有什么漂亮可言,我们希望它的世界传达这一点——恐怖而黑暗。\\
\par

为了达到预期效果,我们使用了扫描与手绘渲染图像的组合。John Carmack 编写了 Fuzzy Pumper Palette Shop 程序,能够捕获实时视频并转换为 PC 图形格式。然后我们把图像直接导入 PC 美术应用中,进行编辑、缩放、上色与合成——做一切需要做的事情以创造有趣图形。整体效果有些失真,但这就是 Doom。\\ 
\par
Doom 的角色通过多种方式制作——手绘、扫描黏土模型,以及乳胶与金属模型。做 Wolfenstein 时,我们就知道为每个动作制作旋转视图有多令人沮丧。大多数角色从正面好画,但旋转 45 度后就复杂得多。我们用小木偶与几磅黏土开始制作自己的模型。这种方法并不完美,但能让我们摆出平常不会画出的姿势。\\
\par
随着项目接近尾声,我们想做一个不是双足行走的怪物。我们设想了一个大脑型生物,脸上嵌着链枪,身体用几根大型金属钩连接到四足金属机器上。我们无法用黏土做出它,于是联系了 Gregor Punchatz。\\
\par
Gregor 有丰富的模型制作背景。他参与过《猛鬼街》《机械战警》等经典电影的布景制作,拥有必要的工具与才能。几周内,Gregor 就把我们的草图变成了一只可动怪物。过程很顺利。尽管本版 Doom 只使用了他的一个创作,但零售版 Doom 将充分发挥 Gregor 的才能。

\subsection{愉快地 Doom 吧}

显然,Doom 远不止屏幕上所见,Doom 的创造者们还能分享更多。我希望上述讨论至少能让你窥见 Doom 创作者的思维,并欣赏 Doom 的艺术与科学。关于如何游玩与享受 Doom 的更多技术信息,请在附录中查找相应主题。\\
\par
