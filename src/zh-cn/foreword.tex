在很多方面,\doom{} 几乎是一款“完美”的游戏。\\
\par
回头看并经过二十多年的技能积累后,我能想到几乎所有方面更好的实现方式,但即便我能坐上时光机回去做完所有改动,也不会真正改变什么。\doom{} 的成功达到了饱和水平,即使它快了 25\% 且多了几个功能,它的遗产也不会有什么不同。\\
\par
巨大的锯齿像素让它从现代视角看起来不太舒服,但 \doom{} 的“扎实感”是当时少数 3D 游戏所拥有的,这主要得益于透视校正、亚像素精度的纹理映射,以及整体很高的稳健性。\\
\par
走向一个完全贴图并带光照、且具有任意 2D 几何的世界,让设计师可以在关卡里做更有意义的事情。Wolfenstein 3D 仍可被视为一款“迷宫游戏”,但 \doom{} 有了建筑感,一些构图里也透露出恢弘的气势。\\
\par
音效经过了实际的处理,包括衰减与空间化,而不是简单播放,其中许多音效足够标志性,以至于几十年后人们仍能认出。\\
\par
引擎从一开始就为用户修改而打造,试玩版分发、公开工具源代码发布与早期在线社区之间的协同,使得原始游戏只是由玩家创造的海量内容中的极小一部分。许多游戏行业的职业生涯起步于对 \doom{} 的改造。\\
\par
与朋友在合作模式中一路轰杀非常有趣,但竞争性的 FPS 死亡竞赛才是这款游戏最伟大的遗产之一。看到另一名玩家穿过屏幕,与你刚发射的火箭路径汇合,这样的瞬间至今仍让数以百万计的玩家会心一笑。\\
\par
为了让 \doom{} 看起来与感觉上都如此出色,背后用了许多巧妙的障眼法。这证明了那些决策的高质量——如此多人以为它做得比实际更多。这个关键教训直到今天仍然适用:经常存在一些折中,可以用可被设计巧妙掩盖的限制,换取显著的优势。\\
\par
—— John Carmack\\
\par
\thispagestyle{plain} % Trick to remove header
