\section{声音系统}
PC 配备了“PC Speaker(内置喇叭)”,只能发出单调刺耳的“哔哔”声。它的本意是用于开机自检(短促的一声表示系统正常)。但严肃的玩家总会投资声卡。凭借强势营销、更优技术和更便宜的产品,Creative Labs 统治了市场。为了生存,任何后来者都必须标榜“SoundBlaster-compatible”。这个非官方标准意味着:基于 OPL2 的 FM 合成器负责音乐,以及一个可在 22KHz、8 位、立体声下回放数字音效的 DSP。\\
\par
 90 年代初是游戏音频最后一波创新的舞台,也见证了曾经重要的厂商 AdLib 的消亡\footnote{讽刺的是,它曾确立了 OPL2 芯片组的必要性。}。尽管如此,仍有两张卡带来了新东西:Creative Labs 的 Sound Blaster 16,以及 Advanced Gravis Computer Technology Ltd. 的 Gravis Ultrasound。\\

\vspace{-2mm}
\subsection{Sound Blaster 16}
 1992 年 6 月,Sound Blaster 16 发布,Creative Labs 以一张可实现 CD 级播放的声卡——44KHz、16 位、立体声采样——彻底解决了 PC 游戏的音频问题。它一经推出便大受欢迎。\\
\par
\fullimage{SoundBlaster_16_ASP_CT1740.png}
\par
然而,解决音频问题反而成了 Creative Labs 商业上的新问题。即便后续推出新产品,也主要面向音频专业人士。ASP 与 EAX 等技术上也曾尝试创新,但消费者对这些改进并不买账。随着音频芯片价格下降且技术需求停滞,厂商开始在主板上集成音频功能。\\
\par
一段时间内,允许连接光驱的 Panasonic/Matsushita 额外接口让声卡还能在捆绑产品中存活。但这仍不足以拯救它们。十年内,声卡市场便消失了。\\
\par
\rawdrawing{sb16}
\par
上图为 1994 年的 Sound Blaster 16 CT1740 型号。
\circled{1} Panasonic/Matsushita 接口(用于 CD-ROM),\circled{2} C1741 DSP 芯片,\circled{3} C1748 ASP 芯片,\circled{4} CT1746B 总线接口,\circled{5} 46.61512 MHz 振荡器,\circled{6} CT1745A 混音器,\circled{7} 用于 MIDI“波表合成器”子板的 WaveBlaster 接口,\circled{8}(自上而下)线性输入、麦克风输入、音量滚轮、线/扬声器输出以及 MIDI/手柄端口。






\subsection{Gravis UltraSound}
Gravis Computer Technology 起初制造了被公认为最佳 PC 手柄的 Gravis PC GamePad。凭借稳健的现金流,他们决定进军声卡市场,推出一张大胆且创新的卡。Gravis UltraSound(昵称 GUS)于 1992 年发布。\\
\par
GUS 宣称通过 TSR 软件仿真实现 Sound Blaster 2.0 的音乐播放。除此之外,这张卡还拥有市面上独一无二的能力:它不是用 FM 合成,而是用数字化乐器采样来回放音乐。这项名为“波表合成(wavetable synthesis)”的技术提供了远超竞争对手的音质。


\fullimage{Gravis_UltraSound_PnP_Pro_V1.png}\\
\par
Gravis UltraSound Pro,\circled{1} 两个 SIMM 插槽,最多支持 8 MiB RAM,\circled{2} IDE/ATAPI 接口,\circled{3} CD 音频接口,\circled{4} IW78C21M1 芯片(1 MiB Flash ROM),\circled{5} HM514260ALJ7 70ns DRAM,\circled{6} 主 CPU InterWave AM78C201KC,以及 \circled{7}(自上而下):麦克风输入、线性输入、线性输出、MIDI/手柄端口。\\
\rawdrawing{gravis}

这个理念很激进,Gravis 工厂里产出的硬件也同样激进。他们使用的红色树脂让其声卡一眼可辨。\\
\par
这项技术的成本是双重的。首先,声卡需要音色采样。“解决方案”是安装 Gravis 驱动,它会装入超过 12 MiB 的声音采样\footnote{当时这是巨大的体积,作为对比,完整版 \doom{} 也才 12 MiB。}。其次,声卡在运行时必须访问这些采样,这意味着它必须自带 RAM。因为采样比正弦方程占空间得多,最初的 GUS 只有 256 KiB,可升级到 1 MiB。\\
\par
它很快在 demo 制作者中形成了追随者,因为其音乐质量极高。但对玩家市场而言,事情更复杂。GUS 的 GF1 主芯片难以仿真 OPL2,配置也很复杂(用户必须手动加载一个平庸的 TSR 仿真器)。GUS 还不巧撞上 1993/1994 年的内存短缺。玩家不愿意掏出 169 美元,比 Sound Blaster 16 贵 40 美元。虽一开始卖得不错,但 1995 年前后销量下滑,1996 年停产。\\
\par
id Software 是少数支持 Gravis UltraSound 的公司之一。\doom{} 包含一个映射文件,将 MIDI 乐器 ID 转换为 Gravis 的 \cw{.PAT} 乐器文件\footnote{它也控制在不同卡配置下加载哪些采样到 RAM 中。}。聆听 \doom{} 原声中 “At Doom's Gate” 的电吉他与鼓点,会让 SoundBlaster 版本相形见绌。但任何成功都需要正确的时机,遗憾的是 GUS 走在了时代前面。\vspace{-5pt}

 \subsection{Roland}
若不提 Roland 的硬件就结束,本节会留下重大缺憾。Roland Corporation 成立于 1972 年,不仅制造音频播放设备,也提供最顶级的作曲与录音硬件。DOS 游戏的突破是 1991 年发布的 Roland SC-55(又名 SoundCanvas)。它不仅是第一款通用 MIDI 标准设备(定义了 128 种乐器,所有遵循标准的设备都可采用),还使用 Roland 专有的“预录采样 + 减法合成”组合进行合成,效果远胜于 Yamaha 的 OPL。\\
 \par
 \fullimage{roland_cr55.png}\\
 \par

Roland 的设备完全围绕 MIDI 协议构建,使用一种特殊的 5 针圆形接口线缆传输。\\
\par
 SC-55 的前身是 MT-32 合成器,可通过 MPU-401 ISA MIDI 转接卡连接到 PC。还有一张 LAPC-I 组合卡,将适配器和 MT-32 的变体 CM-32L 集成在同一张 ISA 卡上。\\
\par 
\cfullimage{roland_lapc1.png}{Roland LAPC-I}
\par
Roland 还发布了 SCC-1,将 SC-55 与 MPU-401 集成在一张 ISA 卡上。\\
\par 
\cfullimage{roland_scc1.png}{Roland SCC-1}
\par
\pagebreak
录制时,音乐人会把键盘连接到名为 MIDI sequencer 的“音符录制程序”。一旦捕获到电脑中,基于 MIDI 的音乐就可以像任何媒体一样被调整与编辑。\\

\par
回放时就复杂一些。当 Sierra On-Line 在 1988 年率先支持 Roland 声卡时,游戏利用硬件播放动听音乐。音效要么来自 PC Speaker,要么后来使用 General MIDI 的预置音效\footnote{1991 年的 Another World 使用了预置音效。}。随着游戏越来越多地采用 PCM 数字化音效(而 Roland 卡无法播放),玩家面临两难:最好的音乐需要 Roland,最好的音效需要 SoundBlaster 或 GUS。昂贵的解决方案(1991 年售价 499 美元)是同时购买两张卡并在外部混音。\\
\par
\rawdrawing{ultimate_audio}
