\section{Watcom}

\begin{wrapfigure}[9]{r}{0.25\textwidth}
\centering
\includegraphics[width=.25\textwidth]{drawings/watcom.pdf}
\end{wrapfigure}


DOS 扩展器很神奇,但在独立产品中设置却很困难。需要一个引导程序去定位 \cw{DOS4GW.EXE} 和要运行的程序,并设置两者,这一流程需要多步并接近 100 行 C 代码\footnote{来源:Watcom C/C++ Programmer's Guide,“7.1.1 The Stub Program”。}。上手时间很长,门槛很高。真正需要的是一个集成环境,让编译器与链接器负责将扩展器和应用打包为一个可执行文件。解决方案再次来自大白北国。\\ 
\par

Watcom 编译器项目始于 1979 年,加拿大安大略省滑铁卢大学。最初只支持 BASIC,随后在学生的改进与支持下,逐步加入对新操作系统和语言的支持。1988 年,三位博士(Fred Crigger、Ian McPhee 和 Jack Schueler)拥有了能在 DOS 上运行并支持 C 的版本。\\
\par
看到商业潜力后,他们成立了 Watcom International Corporation,并选择闪电作为标志以突出性能优势。五年后的 1993 年,Watcom C 已大幅改进。最新版本(9.0)零售价“仅” 639 美元\footnote{按通胀调整,2018 年为 USD\$1,116。如今编译器“免费”。},被认为是 MS-DOS 上最好的选择\footnote{Editor's choice -- PC Magazine, April 1995.}。 \\
\par
他们不仅编程出色,营销也很厉害。90 年代初,翻开任意一本电脑杂志几乎都能看到 Watcom 编译器的整页广告。每条广告都强调其 DOS 扩展器能让程序员摆脱讨厌的 16 位模式,释放 32 位能力。
\par
\label{watcomad}
\fullimage{watcom_ad.png}



\vspace{-4mm}
他们不仅在平面媒体,也在 BBS 和 Usenet 等线上渠道投放广告。\\
\par
\fq{WATCOM C/C++ 生成的代码至少比你当前的 16 位编译器快两倍,通常会快五倍左右。}{rec.games.programmer}\\
\par
\trivia{Watcom 的营销策略之一是从未发布过 v1.0 或 v2.0,而是直接从“版本 6”开始。它至少领先竞争对手(Borland 与 Microsoft)一代。更高的版本号潜意识里传递“比竞争对手更先进”的感觉。版本 1 往往意味着大量 bug,而第六代更可能已经久经战阵。}\\


\subsubsection{受欢迎程度}
id Software 并不是唯一看重 Watcom 方案的团队。许多工作室也将代码托付给它,因此 90 年代大量知名软件都由 Watcom 技术构建:\\
\begin{enumerate}
\item id Software 
       \begin{enumerate}
       \item \doom{} (1993)
       \item \doomii{} (1994)
       \end{enumerate} 
\item Blizzard Entertainment 
       \begin{enumerate}
       \item Warcraft (1994)
       \item Warcraft II (1995)
       \end{enumerate}
\item Ken Silverman 的 BUILD 引擎游戏
      \begin{enumerate}
       \item Duke Nukem 3D (1996)
       \item Shadow Warrior (1997)
       \item Blood (1997)
       \end{enumerate}
\item LucasArts Entertainment Company
      \begin{enumerate}
       \item Full Throttle (1995)
       \item The Dig (1995)
       \item Dark Forces  (1995)
       \item Rebel Assault II  (1995)    
      \end{enumerate}
\end{enumerate}
\par


\subsection{ANSI C}
Watcom/扩展器组合让编程更简单、程序更快,但最重要的还在后面。保护模式编程还有第三个方面——不那么明显但非常关键——对 \doom{} 影响巨大。\\
\par
为了让 C 适配 PC/DOS 的实模式并处理段寄存器,语言曾被“增强”。Wolfenstein 3D 的内存管理器示例展示了 “DOS 版 C” 的样子。\\
\par
\ccode{real_mode_c.c}\\
\par
注意那些疣状关键词,如 \cw{near}、\cw{far},以及 \cw{FP\_OFF}、\cw{FP\_SEG} 等宏,还有 \cw{DOS.H} 库函数,如 \cw{farmalloc}、\cw{coreleft} 与 \cw{farcoreleft}。无论是 “DOS 版 C” 还是 I/O 函数都不具可移植性,因此不可能将 UNIX 程序直接在 DOS 上编译。\\
使用 Watcom 编译器,C 可以按 ANSI 标准编写,从而打开了在不同机器、不同操作系统上编写程序的大门。\\
\par
其中一个系统最终吸引了 id Software 的注意,它的名字叫 NeXTSTEP,运行在 \NeXTns, Inc. 制造的硬件上。
