\vspace{-15pt}
\section{网络}
90 年代初期互联网与 Wi-Fi 尚未普及。把电脑连在一起既困难又昂贵\footnote{计算机对计算机的游戏自 80 年代初就已存在,一些如《Battle Chess》甚至实现了跨平台。}。即便条件允许,带宽和延迟也糟糕透顶。大多数时候,和朋友一起玩意味着把所有电脑搬到同一间房里(局域网聚会)。在自己房间里舒舒服服地玩联机极其罕见。讽刺的是,如今没有联网的电脑反而被视为无用之物。与其他机器通信已被视作理所当然,是一台机器“有用”的最低要求。\\
\par
但在 90 年代初,把 50 磅重的机器(连同 CRT 显示器)装上自行车,活着骑到朋友家,接好线缆,启动 \doom{},终于在另一台电脑屏幕上看到自己的角色动起来,那种感觉难以言表。亲眼看到机器真正互相通信,几乎像魔法一般。\\
\par
为了实现这看似不可能的事,玩家有三种技术可选:零调制解调器(Null-Modem)线缆、电话调制解调器,以及基于网卡的局域网(LAN)。








\subsection{零调制解调器线缆(Null-Modem Cable)}
最便宜、也最常用的方法,是一根约 20 美元的“零调制解调器”线缆,直接插在每台 PC 的 COM 口上。线缆并不进行调制(因此得名)。显而易见,这只能让两位玩家参与。\\
\par
\cscaledimage{0.5}{nullmodemcable.png}{零调制解调器线缆}
\par
两人对战按今天的标准可能很寒酸,但在当时新鲜又酷,简直像是世界上最棒的事情。







\subsection{BNC 10Base2 局域网(Local Area Network)}
想要对战超过一位对手难度陡增。除了购买设备的相对轻松的经济负担外,你还要说服家长让四个青少年来家里通宵尖叫。著名的格言“骗我一次是你的错,骗我两次是我的错”据说源自那些被 \textit{doom} 通宵折磨的家长们。\\
\par



\begin{wrapfigure}[6]{r}{0.25\textwidth}
\centering
\scaledimage{.25}{BNC_T-piece.png}
\end{wrapfigure}

抛开“如何求原谅”的创意方式不谈,从技术角度看,玩家需要通过 ISA 总线插入一块 10Base2 网卡。\\
\par 该卡带有 BNC 接头,需要插上一个称为 T 形接头(T-piece)的连接器。每台 PC 通过 10Base2 同轴电缆与最多两个其他节点相连。这种网络没有中心节点;所有机器组成一条链。链条两端必须连接信号终端器,以防射频信号从末端反射回来造成干扰或功率损失。


\begin{wrapfigure}[8]{r}{0.4\textwidth}
\centering
\scaledimage{.4}{BNC_connector_50_ohm_male}
\end{wrapfigure}
同轴电缆很粗,连接器也不小。将电缆端接到 T 形接头上时,只需优雅地旋转四分之一圈的耦合螺母即可完成连接。\\
\par
% 一旦物理连接完成,无需配置(IPX 是类似以太网的网络层协议)。网卡 MAC 地址即可运行大多数游戏使用的 IPX 协议。\\
 
一旦物理连接完成,游戏依赖于 IPX(Internetwork Packet Exchange)协议,这是一种类似 IP 的网络层协议。无需配置主机或网络,因为与 IP 不同,IPX 能够使用以太网 MAC 地址作为机器的 IPX 地址。\\
 \par
\drawing{10base5bncConnector}{基于 10Base2 BNC 的网络。}
\par
图 \ref{10base5bncConnector} 展示了 1994 年局域网的四个要素:\circled{1} T 形接头连接两根 \circled{2} 同轴电缆形成链路。链条两端必须连接两只负载终端器 \circled{3}。网卡通过 ISA 插槽 \circled{4} 与 PC 相连,并连接到 LAN 的 T 形端口。\\
\par
\trivia{向网络中添加一台新机器意味着要么拔下一个 T 形接头,要么拔下链路终端器。无论哪种方式,主干都被断开,所有其他机器都会失去连接。大家都记得那个总是迟到、导致所有人都得断线让他/她加入的朋友。带宽是共享的,理论 10Mb/s 往往接近 5Mb/s。还不包括那些想交换 30MiB \cw{.wav} 音乐(当时没有 MP3)的朋友。}\\
\par
\trivia{真正有钱的人会用 10baseT 网络,需要一个“集线器”作为中心设备,从而形成星型网络。}








\subsection{电话调制解调器(Modem)}
最幸运的玩家能够负担在家联网的奢侈体验。这非常昂贵,因为不仅要买调制解调器,还得为在线的每一分钟付费。宽带之前,调制解调器通过电话线连接互联网服务提供商。这意味着连接期间没人能打电话,家里任何人拿起电话都会造成足够的干扰以终止连接。\\
\par
当时的互联网并不吸引人,因为游戏和访问 BBS 都是通过直接拨打电话号码完成的。寻找一个酷的 BBS 或游戏伙伴的号码本身就是一场冒险。如果真想看看为数不多的 HTML 页面,AOL(America OnLine)提供的套餐是 9.95 美元包含 5 小时,额外每小时 3.50 美元。若用户平均每天 2 小时,一个月费用为 $9.95 + 55 * 3.5 = \$202$\footnote{按通胀调整:2018 年约为 \$352。},别忘了还要支付一次性 399 美元\footnote{按通胀调整:2018 年约为 \$696。}的 9600 波特率调制解调器费用。\\
\par
\cscaledimage{0.9}{robotic28-8.png}{US Robotics 28.8k 波特率调制解调器。1994 年的顶级机型。}

在建立初始握手时,调制解调器会打开扬声器。耳朵灵的人可以轻松辨别 V.X bis 交易的不同阶段:速率协商、回声抵消器关闭、调制模式选择等,这些共同构成了“死亡竞技即将开始”的难忘旋律。\\
 \par 
\cfullimage{spectrogram2.png}{18 秒的 V.34 握手频谱图\protect\footnotemark }
\par
  \footnotetext{来源:Oona R"{a}is"{a}nen 的文章《The sound of the dialup, pictured》。}
 
 \begin{figure}[H]
\centering  
\begin{tabularx}{\textwidth}{ R{0.1} L{1.9} }
  \toprule
  \textbf{阶段} &  \textbf{描述} \\
  \toprule 
   
   1 & 调制解调器摘机。\\
   2 & 电话交换机发出拨号音。\\
   3 & DTMF:调制解调器拨打 1-(570)-234-0003(美国宾夕法尼亚的一位 \doom{} 玩家)。\\
   4 & 应答端调制解调器启动 V.8 bis 交易。\\
   5 & 应答端请求对方的能力列表。\\
   6 & 呼叫方响应 V8 bis,确认将提供能力列表并请求从电话语音切换到信息传输模式。\\
   7 & FSK 数据 @ 300 bps:我支持完整 V.8。我能发送 ACK。我所在国家是美国,我由 Net2phone Inc 制造。\\
   8 & FSK 数据 @ 300 bps:那我们就用 V8 吧。\\
   9 & 好,模式确认。终止 V.8 bis 交易。\\
   \toprule 
   A & 应答端关闭 PSTN 的回声抑制与回声消除器。\\
   B & FSK 数据 @ 300 bps:重复 6 次:以下是我的调制模式:V.34、V.32、v.23 duplex ...\\
   C & FSK 数据 @ 300 bps:重复 3 次:这些我都支持。\\
   D & 双方调制解调器互相发送宽频探测信号以测量线路。\\
   E & 双方切换到扰码数据模式。\\
   \toprule
\end{tabularx}
\caption{}
\end{figure}
\par




90 年代期间带宽稳步提升。\doom{} 发布时多数调制解调器只有 14.4 Kbit/s。那些在 1993 年 12 月下载共享版的人,要等待 25 分钟才能取回 2,166,955 字节的 ZIP 压缩包。

 \begin{figure}[H]
\centering  
\begin{tabularx}{\textwidth}{ L{0.2} L{0.3} L{0.5}}
  \toprule
  \textbf{年份} & \textbf{版本} & \textbf{带宽} \\
  \toprule 
   
    1990 & V.32 & 9.6 kbit/s \\
    1991 & V.32bis &  14.4 kbit/s \\
    1994 & V.34 & 28.8 kbit/s \\
    1995 & V.34 & 33.6 kbit/s \\
    1996 & V.90 & 56.0/33.6 kbit/s\\
    1999 & V.92 & 56.0/48.0 kbit/s\\
   
   \toprule
\end{tabularx}
\caption{90 年代调制解调器速率。\protect\footnotemark}
\end{figure}
\footnotetext{比特率提升以牺牲延迟为代价。9600 波特调制解调器玩 \doom{} 的体验甚至好过 56k 调制解调器的默认配置。Quake 需要比 \doom{} 的控制器复制更多带宽,因此成为另一种权衡。}

\vspace{-10pt}
\par
在 V.XX 硬件通信层之上,调制解调器使用 Hayes 命令驱动\footnote{这是一个不错的抽象层,但 \doom{} 仍包含一份为 49 种调制解调器准备的初始化参数长文件。}。注意命令 \cw{ATDT} 在前面的频谱图中对应了 DTMF。\\% in figure \ref{spectrogram2.png} on page \pageref{spectrogram2.png}.\\

\par
{
% \setlength{\belowcaptionskip}{-30pt}
 \begin{figure}[H]

\centering  
\begin{tabularx}{\textwidth}{ L{0.2} L{0.15} L{0.65}}
  \toprule
  \textbf{调制解调器 A} & \textbf{调制解调器 B} & \textbf{备注} \\
  \toprule 
   
    ATDT15551234 &\t&\t调制解调器 A 发出拨号命令:AT-引起调制解调器注意(ATtention);D-拨号(Dial);T-双音拨号(Touch-Tone);15551234-拨打这个号码\\
    \toprule 
    RING  & \t& 调制解调器 A 开始拨号。调制解调器 B 的电话线响铃,并报告来电。\\
      \toprule 
    & ATA\t& 调制解调器 B 发出应答命令。\\
    \toprule 
    CONNECT\t& CONNECT\t& 调制解调器建立连接,双方报告“connect”。\\
    abcdef\t& abcdef\t& 连接建立后,任一方输入的字符都会出现在另一方。\\
    \toprule 
    & +++\t& 调制解调器 B 发出转义命令。\\
    \toprule 
     OK &\t& 调制解调器确认。\\
    \toprule 
    & ATH\t& 调制解调器 B 发出挂断命令。\\
    \toprule 
    NO CARRIER &\tOK\t& 双方报告连接结束。调制解调器 B 以“OK”确认命令正常执行;调制解调器 A 以“NO CARRIER”报告远端中断连接。\\
   \toprule
\end{tabularx}
\caption{呼叫方与被叫方的 AT 层对话。}
\end{figure}
}

\trivia{连接的脆弱性催生了幽默的结尾方式。人们常在论坛帖子末尾写上“嘿!等等!别拿起电话 ph\{\#`\$\{\%\&`+'\%NO CARRIER”。}
