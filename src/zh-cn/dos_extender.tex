\section{RAM}
随着 RAM 价格下降,游戏开发者终于可以指望 4 MiB 内存。这本该是好消息,意味着更丰富的世界、更好的资源、更多角色和更大的地图。但由于臭名昭著的内存管理方式,它反而带来了更多复杂性与头痛。\\
\par
责任一部分在 Intel,更多在 Microsoft。1981 年,IBM 发布首台 PC:5150,采用 Intel 8088。该 CPU 只有 16 位寄存器,但 Intel 希望它能够访问 20 位地址空间。为兼顾两者,Intel 设计师想出了一个怪招:分段寻址,把两个 16 位寄存器组合成一个 20 位地址。\\
\par

\drawing{register_combination_20_bits_address}{}
\vspace{-10pt}
\par
指针操作很容易出错,因为不同的段/偏移组合可能指向同一 RAM 位置。指针运算也有问题:一旦偏移回绕,段不会自动更新。\\
\par
到了 Intel 286 和 386SX(24 位寻址),再到 386DX 与 486(32 位地址总线),内存系统变得更加混乱。地址空间已经超出 20 位分段技巧的能力。解决方案只能借助内存管理器,如 \cw{EMM386.EXE} 和 \cw{HIMEM.SYS}\footnote{16 位编程和内存管理器详见 \textit{Game Engine Black Book: Wolfenstein 3D}。},它们提供了访问 1 MiB 屏障之外不可寻址 RAM 的手段。\\

其实还有更简单的方案。Intel 允许 CPU 以两种模式工作:向后兼容的实模式(让 CPU 像一颗非常快的 8088)和释放 CPU 全部能力的保护模式。在保护模式下,32 位寄存器足以寻址板上所有 RAM(称为平坦寻址)。\\
\par
如果操作系统能在保护模式下运行,一切会更顺利。然而出于兼容性,Microsoft 的 DOS 只能处理实模式,从而把开发者锁死在 16 位编程中。\\ 
\par




\par
在 DOS 的阵痛与挫败中,有人看到了机会。\newpage 

虽然有很多产品可以解决问题,但其中两家公司凭借组合脱颖而出:Watcom International Corporation 的 C 编译器与 Rational Systems 的 DOS/4GW “DOS 扩展器”。它们让程序能在保护模式下运行,同时仍可访问 16 位 DOS 功能。\\
\par


\subsection{DOS/4GW 扩展器}
在 DOS 下执行“系统调用”的常规方式是使用软件中断指令,参数为 \cw{21h}。在 C 编程中,这由头文件 \cw{DOS.H} 抽象处理,它在幕后完成所有底层工作。\\
\par
\drawing{realmode_app_limk}{}
为了让应用在一个模式运行、OS 在另一个模式运行,两者需要桥接。于是出现了“DOS 扩展器”这一中间层——它可在两种模式下运行,插在程序和操作系统之间。\\

\par
\rawdrawing{realmode_app_limk_extender}
扩展器启动时会在 OS 的中断向量表中插入钩子,把自己的例程放进去。对应用而言一切透明,开发者无需修改代码。若需要执行扩展器未钩住的系统调用(例如用 \cw{int 33h} 读取鼠标输入),扩展器提供了一个特殊接口 DPMI,通过 \cw{31h} 中断,把 32 位寄存器请求转换为 16 位,使 IVT 例程能够理解。\\
\par
\trivia{DPMI(DOS Protected Mode Interface)最初是为了让 Windows 3.0 运行 32 位应用,并兼容与 IBM 的联合操作系统项目 OS/2。}\\
\par
当扩展器拦截操作系统调用时,需要完成大量工作:
\begin{enumerate}
\item 完成所需的所有转换(例如将 32 位地址表达为 16 位偏移加 16 位段)。
\item 将 CPU 切换到实模式。
\item 将调用转发给 DOS。
\item 取回结果并把 16 位寄存器值转换回 32 位。
\item 将 CPU 切回保护模式。
\end{enumerate} 

性能敏感之处在于实模式与保护模式之间的切换。286 上这曾是大问题,因为 Intel 从未设想程序需要从保护模式切回实模式。必须采用各种技巧\footnote{详见 \textit{Game Engine Black Book: Wolfenstein 3D}。},其中之一是伪造键盘 Ctrl-Alt-Del 重启,以在不真正重启的情况下重置 CPU。\\
\par
另一方面,从实模式切到保护模式很简单。控制寄存器从第 0 位设为 1 只需六条指令。\\
\par
\acode{switch_to_protected_mode.asm}
\par

\doom{} 使用了 Rational Systems 的 DOS/4GW 扩展器。它的存在会在启动时短暂显示。执行 \cw{DOOM.EXE} 后,DOS 会先加载这个小扩展器。加载完成后,DOS/4GW 将 CPU 切换到保护模式,加载 \doom{} 的代码到内存并跳转到 \cw{main} 函数。\\
\par
\fakedosoutput{dos4gw.txt}
