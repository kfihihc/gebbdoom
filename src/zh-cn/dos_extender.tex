\section{内存}
随着内存价格下降,游戏开发者终于可以指望 4 MiB。按理说这意味着更丰富的世界、更好的资产、更多角色与更大的地图。但由于臭名昭著的内存管理方式,这反而带来更多复杂度与头痛。\\
\par
责任一部分在 Intel,但更多在微软。1981 年 IBM 发布了第一台 PC——5150,基于 Intel 8088。CPU 寄存器只有 16 位,但 Intel 希望它能访问 20 位地址空间。为调和二者,Intel 设计师创造了名为“分段寻址”的怪物:将两个 16 位寄存器组合成一个 20 位地址。\\
\par

\drawing{register_combination_20_bits_address}{}
\vspace{-10pt}
\par
指针操作很容易出错,因为不同的段/偏移组合可能指向同一块 RAM。指针算术也有问题:当偏移量回绕时,段并不会自动更新。\\
\par
在 Intel 286 与 386SX(支持 24 位寻址)之后,RAM 系统已混乱不堪;到了拥有 32 位地址总线的 386DX 与 486,更是雪上加霜。20 位分段技巧无法应对如此大的地址空间。解决方案是使用 \cw{EMM386.EXE} 与 \cw{HIMEM.SYS} 等内存管理器\footnote{16 位编程与内存管理器在 \textit{Game Engine Black Book: Wolfenstein 3D} 中有详细介绍。},它们提供了访问 1 MiB 以上不可寻址内存的方法。\\

如果操作系统能运行在保护模式,会有更简单的方案。Intel 允许 CPU 以两种模式运行:向后兼容的实模式让 CPU 像极快的 8088;保护模式则释放全部能力。在保护模式下,32 位寄存器足以寻址全部内存(即平坦寻址)。\\
\par
如果操作系统能够运行在保护模式,这一切就顺利了。然而为了向后兼容,微软的 DOS 只能处理实模式,等同于把开发者锁在 16 位编程中。\\ 
\par



\par
随着 DOS 的阵痛与挫败感加剧,一些人看到了机会。\newpage 

虽然可以解决问题的产品很多,但有两家公司组合起来脱颖而出:Watcom International Corporation 的 C 编译器与 Rational Systems 的 DOS/4GW “DOS 扩展器”。它们让程序在保护模式下运行,同时仍可访问 16 位 DOS 函数。\\
\par


\subsection{DOS/4GW 扩展器}
在 DOS 下,执行“系统调用”的常规方式是使用带参数 \cw{21h} 的软件中断指令。在 C 语言中,这通过头文件 \cw{DOS.H} 抽象出来,由其在幕后完成低层工作。\\
\par
\drawing{realmode_app_limk}{}
要让应用在一种模式运行、操作系统在另一种模式运行,两者必须桥接。一个名为 “DOS extender” 的中间层——能在两种模式下运行——插在程序与操作系统之间。\\

\par
\rawdrawing{realmode_app_limk_extender}
启动时,DOS 扩展器会在操作系统的中断向量表(IVT)中挂钩,并放入自己的例程。从应用程序的角度看一切透明,开发者无需修改代码。对于扩展器未挂钩的系统调用(例如用 \cw{int 33h} 读取鼠标输入),扩展器提供一个名为 DPMI 的特殊接口(中断 \cw{31h}),将 32 位寄存器请求转换为 16 位,以便 IVT 例程理解。\\
\par
\trivia{DPMI(DOS Protected Mode Interface)最初用于让 Windows 3.0 运行 32 位应用,并与 IBM 的 OS/2 联合操作系统项目兼容。}\\
\par
当扩展器拦截操作系统调用时,需要完成大量工作:
\begin{enumerate}
\item 执行所有必要的转换(例如把 32 位地址表示为 16 位段+偏移)。
\item 将 CPU 切换到实模式。
\item 转发调用给 DOS。
\item 取回结果并将 16 位寄存器值转换回 32 位。
\item 将 CPU 切回保护模式。
\end{enumerate} 

性能敏感之处在于实模式与保护模式之间的切换。最初这在 286 上是个问题,因为 Intel 从未设想程序会从保护模式切回实模式。必须使用各种技巧\footnote{详见 \textit{Game Engine Black Book: Wolfenstein 3D}。},其中包括伪造键盘 Ctrl-Alt-Del 重启,以在不真正重启的情况下重置 CPU。\\
\par
相反,从实模式切回保护模式则很简单:将控制寄存器第 0 位从 0 置为 1 只需六条指令。\\
\par
\acode{switch_to_protected_mode.asm}
\par

\doom{} 使用了 Rational Systems 的 DOS/4GW 扩展器,其存在会在启动时短暂显示。执行 \cw{DOOM.EXE} 时,DOS 先加载这个小型扩展器。加载后,DOS/4GW 将 CPU 切换到保护模式,把 \doom{} 代码载入内存,并跳转到 \cw{main} 函数。\\
\par
\fakedosoutput{dos4gw.txt}

