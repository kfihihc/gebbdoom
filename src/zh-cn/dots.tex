\section{等待小点}
\label{dots_explained}
对 1994 年在 PC 上玩过 \doom{} 的人来说,最令人抓狂的部分就是等待加载。尤其是那个神秘的 \cw{R\_Init},似乎永远也跑不完\footnote{事实上取决于硬盘速度,耗时约 15 到 30 秒。}。临时做的进度条由一串小点组成,告诉玩家游戏在加载,请耐心等待。无数小时都花在看这些规整的小点往右推进。大概还有更多小时被用来猜测 \cw{R\_Init} 在后台到底做了什么。\\
\par
\fakedosoutput{dots.txt}\\
\par
通过访问源码,可以修改引擎,让它在当前阶段输出“标签”而不是小点。结果表明它共有 11 个“阶段”。\\
\par
\fakedosoutput{dots_explained.txt}\\
\par

阶段 \cw{OOOOO} 对应 \cw{R\_InitTextures}。书中的 3D 渲染器章节(为简化起见)没有提到的一点是:纹理由补丁(patch)组成。为了节省空间,纹理通过标识符引用补丁。对于由重复图案组成的纹理,这一点尤其有用。该阶段会把纹理的所有属性(如尺寸与补丁定义)加载到 RAM(但 texel 仍在 \cw{.WAD} 中)。由于数据量大且涉及大量 \cw{Z\_Malloc} 内存分配,这是最耗时的阶段。\\
\par
该阶段第一步是打开名为 \cw{PNAMES} 的 lump,其中包含所有补丁名称。根据出现顺序建立补丁名到 ID 的映射。\\
第二步(占大部分处理时间)是打开 \cw{TEXTURE1} 与 \cw{TEXTURE2}。其中包含所有纹理条目。每个条目包含名称与补丁 ID 列表,以及补丁坐标与偏移。\\
\par
阶段 \cw{1} 只是一个标记,表示 \cw{R\_InitTextures} 已返回。\\
\par

阶段 \cw{2} 对应 \cw{R\_InitFlats}。它会查找 \cw{F\_START} 与 \cw{F\_END} 这两个环绕平面纹理(flat)的标记。此处只获取 flat 的数量,以便对 flat 数组进行正确的 \cw{malloc}。\\
\par

阶段 \cw{333333333333} 与阶段 \cw{0} 类似,但这次处理的是 sprite lump。函数 \cw{R\_InitSpriteLumps} 查找 \cw{S\_START} 与 \cw{S\_END},获取 WAD 中所有 sprite 的宽度与水平偏移,并保存到 sprite 数组。处理过程中每 64 个 sprite 会打印一个点。由于需要大量 I/O,这是 \cw{R\_Init} 中第二慢的阶段(仅次于 \cw{R\_InitTextures})。\\
\par

第五阶段(\cw{4})只是一个标记,表示 \cw{R\_InitSpriteLumps} 的结束。\\
\par

第六阶段(\cw{5})也是一个标记,表示 \cw{R\_InitData} 的结束,后者包括 \cw{OOOOO}、\cw{1}、\cw{2}、\cw{333333333333} 与 \cw{4} 这些阶段。\\
\par

阶段 \cw{6} 对应 \cw{R\_InitPointToAngle},用于(当启用时)构建正切查找表。现在它已成为空函数,因为该表已预计算并存放在 \cw{tables.c} 中。\\
\par

阶段 \cw{7} 对应 \cw{R\_InitTables},原本用于构建 \cw{finetangent} 与 \cw{finesine} 查找表。与正切查找表一样,这些也已预计算并烘入 \cw{tables.c} 与可执行文件。\\
\par

阶段 \cw{8} 对应 \cw{R\_InitPlanes},什么也不做。真是浪费了一个点。\\
\par

阶段 \cw{9} 对应 \cw{R\_InitLightTables},它初始化用于光照贴图的 zlight 表。\\
\par

阶段 \cw{A} 对应 \cw{R\_InitSkyMap},它初始化静态的 \cw{skyflatnum}。\\
% \par
% As you will have guessed, the "dot thermometer" was not very accurate. It started fast with next to instant \cw{OOOOO12}, slowed down to a crawl for \cw{333333333333} which accessed a lot of data, and then sped up again at the end with \cw{456789A} when calling mostly empty functions.\\
\par
\section{重新加载 Hack}
在游戏开发期间,启动时间过长是个问题。即便设计师没有改动地图几何(因此无需运行 \cw{doombsp}),仍要等 30 秒才能看到结果。为避免打断创作流程,引擎被加入了一个“重新加载 hack”。\\
\par
只要在 WAD 路径前加上波浪号(\textasciitilde),引擎就会在每次关卡开始时重新加载 WAD。\\
\par
\fakedosoutput{reload_hack.txt}
\par
这让美术与设计人员几乎可以即时看到自己的成果。
