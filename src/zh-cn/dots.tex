\section{等待点点}
\label{dots_explained}
对于 1994 年在 PC 上玩 \doom{} 的玩家来说,最令人抓狂的部分就是等待游戏加载。尤其是某一步——神秘的 \cw{R\_Init}——似乎永远也结束不了\footnote{事实上,根据硬盘速度不同,这一步需要 15 到 30 秒。}。一个由点点组成的“进度条”提示玩家游戏正在加载,耐心等待即可。无数小时都耗在看这些整齐的小点向右推进上。大概也有不少时间用来猜测 \cw{R\_Init} 在后台到底做了什么。\\
\par
\fakedosoutput{dots.txt}\\
\par
如果能访问源码,就可以修改引擎:不再输出点点,而是输出当前阶段的“标签”。结果发现共有 11 个“阶段”。\\
\par
\fakedosoutput{dots_explained.txt}\\
\par

阶段 \cw{OOOOO} 对应 \cw{R\_InitTextures}。书中在 3D 渲染器部分为简洁起见没有提到的是:纹理由 patch 组成。为了节省空间,纹理通过标识符引用所有 patch。对于由重复图案构成的纹理,这尤其强大。该阶段会把纹理的所有属性(如尺寸与 patch 定义)加载到 RAM(但 texel 仍留在 \cw{.WAD} 中)。由于数据量大且有大量 \cw{Z\_Malloc} 内存分配,这是最昂贵的阶段。\\
\par
该阶段第一步是打开名为 \cw{PNAMES} 的 lump,其中包含所有 patch 名称。根据它们出现的顺序,建立 patch 名称到 ID 的映射。\\
第二步(占据绝大部分处理时间)是打开 \cw{TEXTURE1} 与 \cw{TEXTURE2} 两个 lump。它们包含所有纹理条目,每个条目包含纹理名与 patch ID 列表,以及 patch 坐标与偏移。\\
\par
阶段 \cw{1} 只是一个标记,显示 \cw{R\_InitTextures} 返回的时刻。\\
\par

阶段 \cw{2} 对应 \cw{R\_InitFlats}。它查找 \cw{F\_START} 与 \cw{F\_END} 两个 lump,它们是 flat 纹理的边界标记。这里只读取 flat 的数量,以便对 flat 数组进行正确的 \cw{malloc}。\\
\par

阶段 \cw{333333333333} 类似于阶段 \cw{0},但这次是 sprite lump。函数 \cw{R\_InitSpriteLumps} 查找 \cw{S\_START} 与 \cw{S\_END} 来获取 WAD 中所有 sprite 的宽度与水平偏移,并将数据保存到 sprite 数组中。过程中每 64 个 sprite 打印一个点。这是 \cw{R\_Init} 中第二慢的阶段(仅次于 \cw{R\_InitTextures}),因为需要大量 I/O。\\
\par

第五阶段(\cw{4})只是一个标记,表示 \cw{R\_InitSpriteLumps} 结束。\\
\par

第六阶段(\cw{5})也是标记,用于表示 \cw{R\_InitData} 结束,它涵盖了 \cw{OOOOO}、\cw{1}、\cw{2}、\cw{333333333333} 与 \cw{4}。\\
\par

阶段 \cw{6} 对应函数 \cw{R\_InitPointToAngle} ——如果使用——用于构建切线查找表。现在它是空函数,因为该表已预计算并存储在 \cw{tables.c} 中。\\
\par

阶段 \cw{7} 对应函数 \cw{R\_InitTables},过去用于构建查找表 \cw{finetangent} 与 \cw{finesine}。像切线表一样,这些表已预计算并烘焙进 \cw{tables.c} 的可执行文件中。\\
\par

阶段 \cw{8} 对应函数 \cw{R\_InitPlanes},它什么也不做。真是浪费了一个点。\\
\par

阶段 \cw{9} 对应函数 \cw{R\_InitLightTables},初始化用于光照贴图的 zlight 表。\\
\par

阶段 \cw{A} 对应函数 \cw{R\_InitSkyMap},初始化静态 \cw{skyflatnum}。\\
% \par
% As you will have guessed, the "dot thermometer" was not very accurate. It started fast with next to instant \cw{OOOOO12}, slowed down to a crawl for \cw{333333333333} which accessed a lot of data, and then sped up again at the end with \cw{456789A} when calling mostly empty functions.\\
\par
\section{重载 Hack}
在开发期间,漫长的启动时间是个问题。即便设计师没有改动地图几何(因此无需运行 \cw{doombsp}),仍要等待 30 秒才能看到结果。为了不打断创作流程,引擎被加入了一个“重载 hack”。\\
\par
只要在 WAD 路径前加上波浪号(\textasciitilde),引擎就会在每次关卡开始时重新加载 WAD。\\
\par
\fakedosoutput{reload_hack.txt}
\par
这让美术与设计人员几乎能立即看到工作成果。
