\begin{wrapfigure}[10]{r}{0.30\textwidth}{
\centering \scaledimage{0.30}{saturn_logo.png}}
\end{wrapfigure}
Sega Saturn 的开发始于 1992 年 6 月\footnote{来源:“Console Wars: Sega, Nintendo, and the Battle That Defined a Generation”。},目的是取代风靡一时却渐显老态的 Genesis。当时 Genesis 已卖出超过 3000 万台,在 15-25 岁群体中拥有“酷”的形象——得益于大量好游戏与巨额电视广告。对 Sega 来说,这是一次巨大且不得不做的工程,至少要做到不输前代。\\
\label{saturn_port}
\par
两年辛苦工作后,Sega 在 1994 年 6 月东京玩具展展示了 Saturn 原型。其结果连他们自己都不知道:它造成的伤害比 32X 更甚,损害 Sega International 的形象,并最终销量惨淡。\\
\par
在开发过程中,Sega 与 Hitachi 合作开发一颗满足其需求的新 CPU。该合作最终在 1993 年底产出 “SuperH RISC Engine”(亦称 SH-2),Sega 以双 SH-2 作为 Saturn 的基础。\\
\par
图形方面,原计划由一个视频显示处理器(VDP)完成大部分工作。然而 PlayStation 的能力报告促使 Sega 增加第二颗 VDP,以提升系统的 2D 性能与纹理映射能力。\\
\par
\fullimage{consoles/Saturn.png}\\

\vspace{-10pt}
Sega 设法在 PSX 之前发布主机,日本初期销量不错,凭借《Daytona USA》与尤其是《Virtua Fighter》的口碑。初期成功很大程度上归功于《Virtua Fighter》,它是当时日本最火的街机游戏\footnote{来源:“Virtua Fighter Mania”. GamePro. No. 89. 1996 年 2 月。p28。}。\\
\par
抢先 Sony 付出了高昂代价,结果显得仓促。1995 年 E3 洛杉矶展上,Sega CEO Tom Kalinske 意外宣布 Saturn 当天即可购买。即便供应商都不知情,主机很快断货。仓促带来的另一个后果是首发仅有 6 款游戏。《Panzer Dragoon》本可作为旗舰,但未能赶上截止日期\footnote{来源:“The Making Of... Panzer Dragoon Saga”,\cw{nowgamer.com}。}。\\
\par 机器也难以编程。3D 的实现方式类似 3DO:程序员需要处理 2D 四边形,并在屏幕空间扭曲来粗糙模拟透视。这并非努力不足,而是硬件复杂。\\
\par
\fq{目前我们只使用主 SH2,从机 SH2 将在我们弄清楚怎么用时再启用。}{Mick West 1995 年 Saturn 开发日志}\\
\par
全球范围内,平台反响平平。然后野兽登场了。\\
\par
两周后,PSX 携《Ridge Racer》上市,席卷全球\footnote{PlayStation 销量是 Saturn 的三倍。}。不仅 Saturn 更贵(\$399)而 PSX 仅 \$299,而且像《Daytona USA》这样曾经看起来不错的游戏在与《Ridge Racer》对比后暴露问题:更低帧率、弹出式多边形、信箱式画面令人绝望。雪上加霜的是 1996 年 6 月又迎来 Nintendo 64。\\
\par
% Ironically Sega's next console, the DreamCast addressed all issues.  It was easy to program, ran on Windows and was very well liked by developers.
作为失败的象征,《Sonic X-treme》——本可成为 Saturn 上首款全 3D 索尼克与原创索尼克游戏——在经历三年的末日式开发后取消,几乎耗尽制作人 Mike Wallis 的生命。\\
\par
1998 年 Sega 发布 Dreamcast。它受到开发者好评(除拒绝支持的 EA),但早期破解的拷贝保护与 Xbox/PS2 竞争让其受挫。2001 年末,Sega 濒临破产,退出硬件业务,转而专注于 “Virtua” 系列产品。


\subsection{Saturn 编程}
Saturn 编程困难重重。程序员手册分为八本厚重的手册,需要反复阅读才能建立对数据流的心智模型。图 \ref{arch_saturn} 的示意图可以让人感受到协调八颗芯片的艰巨。\\
\par
主程序运行在连接到 2.0MiB 共享 RAM 的两颗 SH-2 上。一颗作为主处理器,另一颗作为从处理器。在常规配置中,从处理器被设计为并行任务的辅助。芯片之间的通信(指示要执行的内容)必须通过繁琐的中断系统完成。处理静态与全局变量时必须使用互斥与信号量,这对主机程序员而言并不常见。由于共享总线,如果两颗处理器同时访问 RAM 或系统外设,就必须互相等待。为缓解这一问题,加入了共享的 4KiB 统一缓存,但对瓶颈改善有限。\\
\par
音频由 SCSP(Saturn Custom Sound Processor)完成,它驱动一颗声音处理器(Motorola 68000)。SCSP 需配置在专用 512 KiB RAM 中进行混音,再由声音处理器取走。组合的结果是强大的系统,可合成乐器、播放 PCM 音频并实现 3D 效果/失真。该芯片还轮询玩家输入并存储在内部寄存器中,供 SH-2 读取。\\
\par
图形编程通过两颗芯片 VDP1 与 VDP2 实现。VDP1 是硬件加速的四边形渲染器,特点是使用前向纹理映射:在渲染精灵(2D 游戏)时非常高效,但在放大或缩小纹理(3D 游戏)时效率不佳。渲染先写入图层,完成后由 VDP2 按优先级与透明度进行合成并发送至电视。注意这两颗芯片并行工作:VDP1 处理下一帧时,VDP2 结束上一帧。\\

\par
CD-ROM 访问由驱动控制 SH-1 处理器完成。双倍速光驱读取速度为 150 KiB/s,但平均访问时间 300 ms。为补偿这一点,SH-1 将数据存入 512KiB 缓冲区。基于糟糕的访问时间,程序员被要求提前很久请求数据。\\ 
\par
为了控制这些组件并在系统间传输数据,第七颗芯片 SCU(System Control Unit)作为 DMA 控制器、DSP 与总线控制器。DSP 能进行矩阵变换并将结果直接写入 VDP RAM。\\
\par


\trivia{仔细阅读编程手册会发现,每个组件似乎都能与其他组件的 RAM 交互。这让调试非常困难。}
\pagebreak

\scaleddrawing{1}{arch_saturn}{Sega 手册:“Introduction to Saturn Game Development”,1994 年 4 月 }




\cfullimage{Sega-Saturn-Motherboard.png}{Sega Saturn 主板}
\par
打开 Sega Saturn 查看主板,可以看到接近 20 颗芯片。\\
\par
\circled{1} 32 位 28.6 MHz SH-2,
\circled{2} 32 位 28.6 MHz SH-2,
\circled{3} VDP2,
\circled{4} YMF292,又名 SCSP(Saturn Custom Sound Processor),
\circled{5} SCU DSP 数学协处理器 @ 14.31818 MHz,
\circled{6} BIOS,
\circled{7} SMPC(System Management \& Peripheral Control),
\circled{8} Motorola 68CE00,
\circled{9} 32 KiB 电池备份 SRAM,
\circled{A} 4 MiB RAM(2MiB RAM + 1.5MiB VRAM + 540KiB 音频 RAM),
\circled{B} VDP1,
\circled{C} Hitachi CD-ROM I/O 数据控制器,
\circled{D} 32 位 20 MHz SH1 微控制器(内置 64k ROM),
\circled{E} 两个手柄接口,
\circled{F} A/V OUT 插座,
\circled{G} Sega 通信接口,
\circled{H} 卡槽(“X-Men vs Street Fighter”所需的 RAM 扩展),
\circled{I} CD-ROM 连接器。





\rawdrawing{saturn_motherboard}
尽管存在诸多问题且发布时机不佳,看到 Saturn 的结局仍令人苦涩,它被视作失败。四年寿命里,平台仍孕育了出色技术与娱乐作品,如 Radiant Silvergun、Grandia、Sega Rally Championship、Virtua Fighter 2、Panzer Dragoon Saga、Guardian Heroes、NiGHTS into Dreams、Panzer Dragoon II Zwei 与 Virtua Cop。不幸的是,\doom{} 不在其列。\\
\par
\fq{多年等待之后,Doom 终于登陆 Saturn。不幸的是,这是经典游戏令人震撼地糟糕的移植。}{Sega Saturn Magazine \#16, 1997 年 2 月}

















\subsection{\doom{} 在 Saturn 上}
% The port to the Saturn was done by Rage Software on a very tight schedule. Like most ports, it is based on the Jaguar assets. The console had the same technical shortcomings when it comes to texture sampling, as Jim Bagley remembers\footnote{RetroGamer \#134.}:\\

Saturn 版移植由 Rage Software 在极其紧迫的时间内完成。引擎图形部分通过 VDP1 将四边形写入三个独立图层(地板/天花板/墙壁层、物体层、状态栏层),再由 VDP2 合成输出到电视。与其他移植相比帧率出色,但缺乏透视正确纹理映射让 Jim Bagley 的计划受挫\footnote{RetroGamer \#134。}:\\
\vspace{10pt}
\par
\fq{我刚开始项目时,需要做一个 demo 给 id Software 审批。我先把 WAD 文件里的关卡、音频、纹理提取出来,做了自己的 Saturn 版本,然后用 3D 硬件跑起了早期渲染器。提交后几天,我收到 John Carmack 的电话,他明确规定:无论如何都不能用 3D 硬件来绘制屏幕。我必须像 PC 那样用处理器渲染。幸运的是我喜欢挑战,所以我们用两颗 SH2 像 PC 那样渲染显示,并用 68000 来协调。\\
\par
但这等于给游戏戴上手铐,速度和帧率都大幅下降。}{Jim Bagley for RetroGamer \#134}\\
\par
多年后,到了 2014 年,Carmack 改变了看法。\\
\par
\fq{我讨厌仿射纹理游移与整数四边形顶点,但事后看来,我可能应该让他们试试。}{John Carmack}\\
\par
最终,VDP1 的 60 FPS 硬件加速引擎被抛弃。\\
\par
由于时间紧迫,Jim 无法改成像 PlayStation 那样的像素宽三角形渲染。发售时,游戏帧率最高可达 20 FPS,但多数时候在全屏 281x235 分辨率下跌至个位数。为补偿低帧率,Jim Bagley 决定降低所有移动速度,这一决定激怒了玩家。\\
\par
\trivia{匆忙的另一影响是音频系统的 bug,导致所有音效都偏向左声道。玩家不得不切换为单声道才能听到两个声道\footnote{Digital Foundry: “Every Console Port Tested and Analysed!”。}}\\
\par




\fullimage{doom_saturn1.png}\\
\par
上图中,E1M1 与其他主机版相同(都基于 Jaguar 的初始版本)。但状态栏做了改造。\\
\par
真正“绊倒”项目的是墙体、天花板与地板渲染(占大部分计算成本),它们最终由 SH-2 软件渲染,而状态栏与物体(如带透明部分的怪物或墙体)由 VDP1 硬件加速\footnote{Digital Foundry: “Every Console Port Tested and Analysed!”。}。三层就绪后,由 VDP2 合成输出,VDP1 同时开始渲染下一帧。\\
\par 
透明效果以一种怪异方式实现,细节证明了这台机器的复杂性。VDP1 与 VDP2 都支持“半透明”,即源与目标等比例混合。但 VDP1 只支持 15 位色透明,而 VDP2 只支持索引模式透明。结果是,你可以让精灵彼此半透明,或让精灵对 VDP2 背景层半透明,但无法同时满足两者。\\
\par
这对渲染 “Spectre” 敌人是个大问题。如果 VDP1 将 Spectre 像素标记为“半透明”,它能正确显示在背景层之上,但也会“吞掉”后方其他精灵,导致场景错误\footnote{来源:“The Sega Saturn and Transparency” by Matt Greer。}。\\
\par
Sega 设计者清楚 VDP 的限制,于是引入了“网格”精灵概念:由 VDP1 以不透明渲染,但只绘制隔行像素。\\
\par
\cfullimage{doom_saturn4.png}{以“网格”方式渲染的 Spectre 敌人。}
\par
像素级截图,尤其是下一页放大图,乍看粗糙。但要记住复合交错显示系统最终会混合一切。尽管视觉效果相当可信,这个魔术技巧在 “HD” 像素级时代并未幸存。\\
\par
\cfullimage{doom_saturn41.png}{同一场景,放大后显示 Spectre 的跳跃像素。}
