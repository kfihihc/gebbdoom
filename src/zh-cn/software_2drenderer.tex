
\section{2D 渲染器(Drawers)}
\doom{} 中所有“drawer”都由 Dave Taylor 完成。这位自称“补缝码农”\footnote{来源:“Dave Taylor Interview”,blankmaninc.com。}在游戏发售前 3 个月才加入,但他仍然完成了许多系统。他的代码风格与变量命名约定不同于 John Carmack,因此一眼就能看出归属。
\begin{itemize}
\item 关卡间画面(\cw{WI\_Drawer()})
\item 状态栏(\cw{ST\_Drawer()})
\item 菜单(\cw{M\_Drawer()})
\item HUD(\cw{HU\_Drawer()})
\item 自动地图(\cw{AM\_Drawer()})
\item 过渡屏幕(\cw{wipe\_StartScreen()})
\end{itemize} 
\par




\subsection{关卡间画面}
关卡间画面很简单,全部在 \cw{wi\_stuff.c} 中。它首先把一个原始背景地图从 WAD 档案加载到 RAM,并放入帧缓冲 \#1。\\
\par
\fullimage{intermissions/WIMAP0.png}\\
\par
当需要刷新屏幕时,关卡间代码会把帧缓冲 \#1 的内容用 \cw{memcpy}(代码里称为 “slam”)复制到帧缓冲 \#0,然后在其上绘制精灵和文本。\\
\par
\fullimage{intermissions/stats.png}\\

关卡间画面结束后,\cw{WI\_unloadData} 不会释放 RAM,只会把所需元素标记为 PU\_CACHE。\\

%\fullimage{intermissions/Intermission.png}\\



\par

\subsection{状态栏}
状态栏的设计与关卡间 drawer 相似。它在 \cw{st\_lib.c} 和 \cw{st\_stuff.c} 中实现,代码把其元素称为“控件(widgets)”。共有 7 个控件,从左到右依次是:当前武器弹药、生命百分比、武器栏、表情、护甲百分比、钥匙盒、以及全部四种弹药数量。\\
\par
\fullimage{stbar.png}
\par
%\trivia{Head shows where damage came from.}\\

\par
\vspace{20pt}
\fullimage{faces.png}\\ \par \vspace{10pt}
大多数时候,表情控件显示的是“正常”动画,海军陆战队员的眼睛左右看。注意它为五个生命区间各提供了一组状态。\\
\par
也许是因为战斗时压力很大,许多玩家从未注意到这个控件会显示伤害来自何处(左/右)。拾取武器时会显示“Evil”。当玩家正面受伤、持续按住开火按钮,或“因为自己该死的愚蠢而受伤”(\cw{st\_stuff.c} 中的原文注释)时,会显示“Kill”。\\
\par
即便是玩了很久 \doom 的玩家,也很可能从未见过“ouch”表情。它原本用于在玩家受到极大伤害(超过 20 点生命)时显示,足以让生命区间下降两档。但代码中的一个 bug 阻止了它发生:\\
\par
\ccode{ouch_face.c}\\
\par
测试条件与预期相反。ST 模块在增加 20 点生命时显示 ouch,而这几乎不会发生。测试应当反转。\\
\par
\ccode{ouch_face_fixed.c}\\
\par
在与视频系统的交互方面,状态栏与关卡间模块类似。启动时,它会把一个“干净”的状态栏绘制到帧缓冲 \#4。当状态栏需要刷新时,它会从帧缓冲 \#4 \cw{memcpy} 到帧缓冲 \#0,然后在其上绘制所有控件。\\
\par
\fullimage{stbar_virgin.png}
\subsection{菜单}

\begin{wrapfigure}[5]{r}{0.1\textwidth}
\centering
\scaledimage{0.1}{M_SKULL1.png}
\end{wrapfigure}
共有 10 个菜单,全部硬编码在 \cw{m\_menu.c} 与 \cw{m\_misc.c} 中。设计很简单,\cw{menu\_t} 包含 \cw{menuitem\_t} 列表。每个 menuitem 的 \cw{name} 字段用来获取相应的精灵。\\
\par

\ccode{menu_item.c}

\par
一个 \cw{menu\_t*} 会被菜单绘制例程消费,并在当前选中的 \cw{menuitem\_t} 左侧绘制一个骷髅。\\
\par
\begin{minipage}{0.55\textwidth}
\ccode{main_menu.c}
\end{minipage}
\begin{minipage}{0.45\textwidth}
\centering
\scaledimage{0.9}{menu.png}
\end{minipage}
\par
\ccode{M_DrawMainMenu.c}
\par
在上面的主菜单代码示例中,注意字符串 \cw{M\_NGAME}、\cw{M\_OPTION}、\cw{M\_DOOM} 都是 WAD 档案中的 lump 名称。\\
\par
%\trivia{用于选择屏幕大小、鼠标灵敏度、音效音量与音乐音量的滑块被称为“Thermostats”。}\\
%\par








\subsection{HUD(抬头显示)}
在早期版本中,\doom{} 的 HUD 模仿了 Doomguy 的头盔。\\
\par
\fullimage{alpha_hud.png}
\par
随着时间推移,设计改变,HUD 缩小为几行文字。相关的小段代码位于 \cw{hu\_lib.c} 与 \cw{hu\_stuff.c}。\\
\par
\fullimage{new_hud.png}\\
\par
%\trivia{尽管市场很小,本模块仍特别照顾了法语字符。}



\subsection{自动地图}
自动地图是一个小而简单的组件,位于 \cw{am\_map.c} 中。随着玩家探索关卡,地图会记录已看到的线条。红线表示实墙。黄线表示天花板高度变化(例如门)。棕线表示地板高度变化。\\
\par
遗憾的是,无法在这个模式下玩游戏(我们都试过,别装了),因为已访问线条的标记由 3D 渲染器完成,而自动地图启用时 3D 渲染器被禁用。\\
\par
\fullimage{automap.png}
\par
\trivia{自动地图几乎包含一个彩蛋。文件 \cw{am\_oids.h/c} 旨在让玩家玩一个《Asteroids》的复刻版。不幸的是,这个彩蛋未完成。}\\
\par
\fq{我记不清是谁的主意了,但大概是我的。自动地图的矢量美术风格很吸引我,所以《Asteroids》会很合适,但 Doom 落后于进度,而 id 的开发节奏又快得惊人。}{Dave Taylor}\\
\par
%\trivia{允许玩家在自动地图上看到一切的作弊码 \cw{iddt} 以 Dave Taylor 命名。}
