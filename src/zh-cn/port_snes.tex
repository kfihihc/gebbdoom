\begin{wrapfigure}[3]{r}{0.4\textwidth}{
\centering \scaledimage{0.4}{snes_logo.png}}
\end{wrapfigure}
Super Nintendo Entertainment System 于 1990 年在日本发布,次年在美国与欧洲上市。它是 8 位 NES 的 16 位继任者。在日本,Super Fami-Com(“FAMIly COMputer”)一炮而红,首批 30 万台在数小时内售罄。狂热程度之高,以至于政府要求 Nintendo 在周末发布新主机以避免骚乱。\\
\par
Nintendo 建立了严苛的机制来确保游戏质量。发行商每年只能发布五款游戏。为确保规则执行,只有 Nintendo 可以生产卡带;发行商必须向 Nintendo 购买。为了确保所有人守规矩(也为了防拷贝),SNES 在启动游戏前会查找 CIC 锁定芯片。这一强大机制直到 SNES 生命周期末期才被破解。\\
\par
在九年生命周期内\footnote{Super Nintendo 于 1999 年停产。},共发布 721 款游戏,其中包括多款商业与口碑双成功作,如 Super Mario World、Zelda III、Mario Kart、F-Zero、Super Metroid 与 Donkey Kong Country。总销量接近 5000 万台,使其在销量与游戏阵容上都堪称史上最受欢迎的主机之一\footnote{来源:“The SNES is the greatest console of all time” by Don Reisinger。}。\\
\par
\cfullimage{consoles/SNES.png}{Nintendo 的 Super Famicom(即 SNES)}
\par
从技术角度看,SNES 在 2D 方面表现卓越。它拥有 16 位 65C816 3.58 MHz CPU,配备 128 KiB RAM。其 PPU(图像处理单元)拥有 64 KiB RAM,可处理大尺寸精灵,分辨率 256x240,最多 256 色。音频方面,8 位 Sony SPC700 与 16 位 DSP 组合,并配备 64 KiB 专用 SRAM。\\
\par
尽管拥有出色的 2D 精灵引擎,尤其是“Mode 7”能力,SNES 在 3D 计算等高负载任务上仍然吃力。Nintendo 清楚 3D 将是游戏的下一个风口,但苦于无力推进。命运使然,一家英国小公司提供了解决方案。\\

\vspace{-10pt}
\subsection{Argonaut Games}
早在 1982 年,Jez San 还是一名单打独斗的游戏开发者,专注于 C64、Atari ST 与 Amiga。为了销售作品,他需要一家公司。看到自己的名字(J.San)与希腊神话中的 Jason and the Argonauts 相似,他将公司命名为 Argonaut Games plc。\\
\par
他的创业很快不再是单人项目。到 1990 年,他在伦敦组建了团队,并对 Nintendo 的 1989 掌机 Game Boy 产生兴趣。团队完成了两项几乎不可能的壮举:实现 3D 线框引擎,并破解 CIC 保护,使其能运行在 Game Boy 上。\\
\par
\fq{他们让 Nintendo 的标志从屏幕顶部落下,当它落到中间位置时,引导程序会检查是否对齐。\\
\par
只有标志在正确位置时游戏才会启动。如果有人想在没有 Nintendo 许可的情况下制作游戏,他们就等于未经授权使用“Nintendo”商标,Nintendo 便可起诉侵权。我们发现只需一个电阻和电容——约 1 美分的元件——就能绕过保护。系统会读取 “Nintendo” 两次:第一次用于开机显示,第二次用于启动前校验。这是致命错误,因为第一次读取时我们让它返回 “Argonaut”,所以下落的字样就是它;第二次校验时,我们的电阻与电容已经上电,正确的 “Nintendo” 还在,于是游戏就能正常启动。}{Jez San\protect\footnotemark}

\footnotetext{来源:Jez San 采访 Damien McFerran 的文章 “Born slippy: the making of Star Fox”。}






在 1990 年 CES 上,他在 Nintendo 展台演示了这段引擎(运行在破解的卡带上),消息一路传到京都总部。Jez 当时不知道的是,他的时机好得惊人。此时 Nintendo 正在为 Super Famicom 的首发准备展示其技术实力的游戏。Super Mario World 还在早期阶段,而飞行模拟游戏 “Pilotwings” 已更成熟。\\ 
\par
\begin{wrapfigure}[14]{r}{0.5\textwidth}{
\centering \scaledimage{0.5}{pilotwings.png}}
\end{wrapfigure}
SNES PPU 的 Mode 7(可进行旋转、缩放与剪切等仿射变换)配合 HDMA 模式,用于模拟 Pilotwings 的地形。但飞机仍是 2D 手绘精灵。这让制作人 Shigeru Miyamoto 感到不满,因为无法让摄像机平滑绕机旋转(量化精灵很锯齿)。\\ \par
当时 Nintendo 不习惯与外部人士,更别说外国人合作。但这次他们破例,将 Jez 与负责 3D 的 Dylan Cuthbert 一同请到京都总部。\\
\par
这对年轻人\footnote{Jez 23 岁,Dylan 18 岁。}见到了 Nintendo 的高管:Miyamoto、Gunpei Yokoi、Takehiro Izushi、Yasuhiro Minagawa 与 Genyo Takeda。他们看到了从秘密 SNES 到秘密 Mario/Pilotwings 的一切,随后被问到是否有办法把飞机绘制成完整面片的多边形。\\
\par
\fq{我告诉他们,除非让我们设计一些硬件来增强 SNES 的 3D 能力,否则这已经是极限了。令人惊讶的是,尽管我从未做过硬件,他们竟然说“好”,并给我 100 万美元让它实现。}{Jez San}。\\
\par
Jez 大胆承诺“10 倍”性能提升,Nintendo 接受了为其游戏设计专用硬件的提议。“Pilotwings”将以精灵飞机发售以赶上 Super Famicom 首发,但后来被称为 “Super FX” 的芯片要用于 Nintendo 另一个项目。\\
\par
它的名字是 “Star Fox”。

\subsubsection{Star Fox}
合作协议规定 Nintendo 负责游戏设计并出资,Argonaut Games 负责生产硬件与 3D 引擎。Jez-san 立刻招募并外包英国最优秀的人才。\\
\par
硬件方面,他们联系了 Flare Technology(同样设计过 Atari Jaguar)。Ben Cheese、Rob Macaulay 与 James Hakewill 的项目代号为 Mathematical Argonaut Rotation I/O,即 “MARIO”。最终设计的芯片强大到他们戏称 Super NES “只是用来放芯片的盒子”。由于无法改造主机,芯片被焊在每张新卡带上,显著提高了 MSRP。\\
\par
\fq{我们设计 Super FX 芯片的方式与以往硬件设计截然不同——先做软件,再为软件设计指令集以便最优执行。以前没人这么做!我们并非设计 3D 芯片,而是设计了一颗完整的 RISC 微处理器,拥有数学与像素渲染功能,其余由软件完成。这是世界上第一款图形处理器(GPU),我们还有专利。}{Jez San}\\
\par
引擎方面,Carl Graham 与 Pete Warnes 在伦敦总部工作,Dylan Cuthbert、Krister Wombell 与 Giles Goddard(后来还有 Colin Reed)常驻京都,与 Miyamoto 团队紧密合作。\\ 
\par
最终该项目在口碑、商业与工程上都取得成功。《Star Fox》于 1993 年 2 月 21 日发售,全球销量达 400 万份\\
 \par

这段两家公司合作的美好故事后来却变得苦涩。Argonaut 已完成《Star Fox 2》,计划于 1996 年发布,但 Nintendo 担心其影响 Nintendo 64 的发布而突然取消。Argonaut 不满,双方关系破裂。之后 Nintendo 挖走 Goddard 与 Wombell。Dylan Cuthbert 也想加入,但因合同中的竞业限制被阻止。他离开 Argonaut,转投 Sony 参与 PlayStation 的开发。\\
\par
最终破裂发生在 Nintendo 拒绝让 Argonaut 在 PS1 平台游戏中使用 Yoshi。Argonaut 转而用鳄鱼替代,推出《Croc: Legend of the Gobbos》。Nintendo 后来发布的《Mario 64》在机制上似乎受到 “Croc” 启发……并且提前约一年上市。





\fullimage{snes_cartridge.png}
\par
\vspace{20pt}
MARIO 芯片采用简单设计:16 位 RISC 处理器,运行在 10.74 MHz,配 512 字节 i-cache。它拥有为数学优化的指令集与为像素绘制优化的帧缓冲。其运行方式是在帧缓冲中渲染,数据周期性通过 DMA 传回 SNES RAM。据称其性能可达 76,458 多边形/秒,这意味着《Star Fox》约 15 fps。\\


\par
见证《Star Fox》的巨大成功后,其他工作室也对该技术产生兴趣。芯片被修订(移除内部频率分频器)以 21.4 MHz 运行,并更名为 “GSU”\footnote{Graphics Support Unit}。第一代 GSU 驱动了四款游戏:Dirt Racer、Dirt Trax FX、Stunt Race FX 与 Vortex。\\
\par
第二代(GSU-2)仍以 21.4 MHz 运行,但额外引脚接入总线以增加可支持的 ROM 与帧缓冲容量。它用于三款游戏:\doom{}、Super Mario World 2: Yoshi's Island 与 Winter Gold。\\
\par
打开一张 \doom{} 卡带即可看到前述组件:\circled{1} 16 位 GSU-2,\circled{2} 512 KiB 帧缓冲(GSU 写入),\circled{3} 2MiB ROM(代码与资源),\circled{4} 十六进制反相器,\circled{5} 复制保护 CIC 芯片。






\rawdrawing{snes_board}
\vspace{10pt}
\fq{“十倍”的说法完全是我过度承诺。我们当时根本不知道是否可能。\\ 
\par
但这让我们既敢于承诺,又能超额交付。我们最终不是 10 倍,而是提升了约 40 倍。有些方面——比如 3D 数学——大约是 100 倍。它不仅能做 3D 数学与向量图形,还能做精灵旋转与缩放——这正是 Nintendo 想要的,比如在《Super Mario World 2: Yoshi's Island》。}{Jez San}\\
\par

\trivia{有些热情玩家收集了 SNES 全部 721 款游戏。看到它们摆在书架上非常震撼。你通常可以在 20 英尺外就认出 \doom{} 卡带。标准灰色之外仅有三款游戏被允许采用特殊颜色:两款是红色——\doom{} 与 “Maximum Carnage”,而 “Killer Instinct” 是黑色。}
\pagebreak


\drawing{snes_cartridge}{SNES 721 游戏库。Zelda 之所以独特,是因为它就是 Zelda。}
\par
\rawdrawing{snes_cartridge2}











\subsection{\doom{} 在 Super Nintendo 上}
\doom{} 登陆 SNES 要归功于一位天才且意志坚定的人:Randy Linden。他热爱这款游戏,决定把它移植到大众主机,让更多玩家体验。Randy 从未获得 PC 或主机版的源代码与资源,他是从零开始的。\\
\par
为获取资源,他利用 Matthew Fell 的《非官方 Doom 规格》,其中详细描述了 \cw{.wad} 的 lump 布局。他从 \cw{DOOM.WAD} 提取了精灵、纹理、音乐、音效与地图。但引擎则完全是另一码事。\\
\par

\fq{DOOM 是一款真正开创性的作品,我想让没有 PC 的玩家也能玩到。SNES 上的 DOOM 是另一个我相信可以完成的编程挑战。\\
\par
我独立开始了这个项目,并在有完整可运行原型后向 Sculptured Software 演示。Sculptured 的一些人帮助完成了游戏,以便赶上假日档期。\\
\par
开发之所以困难,是因为当时没有 SuperFX 芯片的开发系统。我必须先写一整套工具——汇编器、链接器与调试器——才能开始写游戏本身。\\
\par
开发硬件是一张“改装过的”Star Fox 卡带(因为它包含 SuperFX 芯片),以及一对改装手柄插在 SNES 两个端口,并连接到 Amiga 的并口。我们用串行协议在两者之间通信,用于下载代码、设置断点、检查内存等。\\
\par
我希望能有更多关卡,但游戏已经用尽当时最大的 ROM 容量,几乎满到只剩 16 字节,所以确实没有空间了!不过我仍然加入了 SuperScope、鼠标与 XBand 调制解调器支持——没错,你可以联网对战!}
{Randy Linden(接受 gamingreinvented.com 采访)}\\
\par
\trivia{Randy Linden 后来参与了更令人惊叹的逆向项目。1999 年,他参与制作了商业 PlayStation 模拟器 “Bleem!”——这很大胆,因为当时主机仍在生产与销售。}\\
\par
该版本最值得称道的是,Randy 在引擎能力与限制下,以不同于其他主机移植的方式做出取舍。\\
\par
\cfullimage{consoles/snes/snes_e1m1.png}{SNES 上 E1M1 起始画面}
\par
% \vspace{-10pt}
在图 \ref{consoles/snes/snes_e1m1.png} 中,尽管只有 600 KiB RAM,蓝色地面依然保留(但以纯色呈现)。注意几何并未改动\footnote{由于 Randy 无法使用 DoomED 或 doombsp,修改几何非常困难。},E1M1 起始房间的台阶仍保持原样(对比 PC 版第 \pageref{mashed_potatoes1.png} 页与 Jaguar 版第 \pageref{doom_jaguar3.png} 页)。\\
\par Reality 引擎(Randy 的命名)能够处理 PC 版地图几何,但似乎受限于填充率或纹理采样,因此天花板与地板纹理被完全移除。







\cfullimage{consoles/snes/wide.png}{E1M1 外部毒液池}
\par
% \vspace{-10pt}
上图可见窗口并非真正全屏。这并非 \doom{} 独有,因为所有使用 Super FX 的游戏(如 Star Fox 与 Star Fox 2)都必须缩小活动区域。这可能是由于 SNES 带宽有限,无法进行全屏 DMA 传输\footnote{一些网站如 \cw{anthrofox.org} 推测 Super FX 无法渲染超过 192 行。}。\\
\par
在原生 256x224 分辨率中,仅绘制 216x176,而 3D 画布只有 216x144(另 32 行为状态栏)。垂直线翻倍后,Reality 引擎实际渲染为 108x144。即便如此低分辨率,平均帧率仍约 10 FPS,是一项惊人壮举。“低”帧率并未阻止玩家享受 \doom{}。Randy Linden 表示该版本销量很好。






\cfullimage{consoles/snes/enemy.png}{E1M3,注意地板上的抖动效果。}
\par
% \vspace{-10pt}
令人惊叹的是,Reality 能通过抖动技术实现墙体与地板的光照衰减,如图 \ref{consoles/snes/enemy.png} 中大块地板“阴影”所示。\\
\par
在为了 RAM 牺牲的特性列表中,精灵分辨率被大幅降低,甚至有时难以辨认(与高分辨率的玩家武器形成对比)。敌人姿态除面对玩家的一帧外全部移除,怪物互斗也被移除;没有声音传播(怪物只在视野接触时被唤醒),大部分音效被删,所有怪物都听起来像小鬼(imp)。
\\
\par
\trivia{Nintendo 起初禁止 SNES 游戏中的血腥内容。到 DOOM 发布时,ESRB 已成立。鉴于游戏里的血腥与遍地残肢,Doom SNES 合情合理地获得 M 级评级。}

   % Super FX2:
   % Released 1993\\
   % GSU: 23 MHz\\
   % VRAM: 32 KiB\\
   
   % SNES:\\
   % Released 1990\\
   % CPU: 5A22 3.58 MHz\\
   % Audio: SPC700\\
