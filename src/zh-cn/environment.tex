1991 年 Wolfenstein 3D 的开发与 1993 年 \doom{} 的开发之间,PC 环境已经发生了显著变化。之前基于 Intel 386、2~MiB RAM 与 VGA 显卡的“推荐配置”已经过时。\\
\par
“新”的顶配 PC 仍由六个子系统构成:\circled{1}~输入、\circled{2}~总线、\circled{3}~CPU、\circled{4}~RAM、\circled{5}~视频输出、\circled{6}~音频输出,共同形成一条流水线。它们都变得更快或容量更大。\\
\par
\vspace{2mm}
\drawing{pc_components}{IBM PC 的六大组件。}
\par
在详细描述各组件之前,需要做一个重要澄清。本章以对比 \doom{} 可用硬件与 Wolfenstein 3D 可用硬件为结构,因此会给人一种“性能大增”的感觉,但这是误导。\\
\par
尽管下面列出了一长串改进,务必记住:IBM PC 仍然不是适合游戏的机器。它受限于最初的目标——办公。它被设计用来文字处理、计算表格,偶尔显示静态图表——并非用于实时、70Hz\footnote{VGA 最常见的游戏刷新率(320x200)为 70Hz,而非如今常见的 60Hz。} 动画。\\ 
\par 
回头看,很难相信工作室会专注为一台“显然”不如主机的机器制作游戏。问题清单很长:
\begin{itemize}
\item CPU 无法执行浮点运算,且没有协处理器\footnote{比如 Amiga 的 Paula、Denise 和 Agnus。}。
\item 图形系统陈旧,似乎无法双缓冲,且纵横比与显示器不同,导致画面变形。
\item 事实上只能发出刺耳“哔哔”声的声音系统;即便用户购买了声卡,生态也高度碎片化。
\item 顶配机器售价接近 3,000 美元。对比而言,SNES 与 SEGA Genesis 均为 199 美元,而可提供街机体验的 Neo-Geo 为 649.99 美元\footnote{按 2018 年通胀换算:PC 为 10,476 美元,SNES/Genesis 为 377 美元,Neo-Geo 为 1,134 美元。}。
\end{itemize}
\par
PC 最多只能说“不吸引人”,也似乎不太可能产生好游戏,尤其是与那些为 60Hz 动画而生并具备精灵引擎的低价系统相比。\\
\par
显然,鉴于你手中这本书的题目,只需一些软件技巧,PC 的硬件就能做出远超设计初衷的事情。PC 不擅长某些类型的游戏,但在需要帧缓冲的某些类型上可以大放异彩。在一个没有互联网且文档稀少的世界里,这绝非易事。\\
\par

对页的图 \ref{ibm_ps1_top} 展示了 90 年代许多电脑杂志中常见的广告。请注意其中的 IBM PS/1 搭载 Intel 486 CPU,强调办公用途以及运行静态办公应用的能力。对 1993 年而言,一台售价相当于美国年收入中位数 5\% 的机器,只能以生产力和利润来正当化\footnote{statista.com 列出 1993 年美国收入中位数为 52,335 美元。Byte Magazine 1993 年春季广告显示 486 DX2-66 VESA PC 的价格为 2,575 美元。}。\\
\par

 % (\$2,000\footnote{\$13,000 in 2017.}). Also worth mentioning, the "huge" standard 14" CRT which must have weighted close to 20 pounds!\\
\par
\begin{figure}[H] \centering
\fullimage{ibm_ps1_top}
\caption{约 1993 年的 IBM PS/1 广告。注意那台可笑地小的标准 14" CRT 显示器,分辨率最高可到 800x600。}
\label{ibm_ps1_top}
\end{figure}













\cleartoleftpage
 
\cfullimage{PX486P3/b-486-px486p3_romain}{QDI Computer, Inc. 的 PX486P3 主板。}
要概览当时可用硬件,一个实用方法是打开一台 PC,看一看将所有组件连接起来的部件。1994 年最畅销的主板是 QDI Computer, Inc. 的 PX486P3\footnote{一家加拿大公司 \scaledimage{0.03}{Canada}!}。\\

\par
最显眼的新东西当然是计算机的心脏:Intel i486 CPU \circled{1}。再细看会发现更多特性,而它们对 \doom{} 的架构至关重要。\\
\par 
黑色连接器显示的是传统 ISA 总线扩展槽。一个 8 位 \circled{2} 和三个双槽 16 位 \circled{3} 可容纳四张 ISA 卡。还有三组新的连接器,旁边多了一条棕色插槽 \circled{4}。这些是 VLB 插槽\footnote{亦称 VL-Bus,也即 VESA Local Bus。},其速度可达 ISA 的 10 倍。

\drawing{px486p3}{PX486P3 的组件示意图。}
\par
左上 \circled{6} 处,系统主内存的容量、速度和复杂度都增加了。由于制造成本大幅下降,标准 DRAM\footnote{Dynamic RAM.} 变为 4 MiB\footnote{\doom{} 无法在仅 2 MiB 的 PC 上运行。}。\\
\par
最后,右上 \circled{5} 处,一种新型 RAM 出现在这些新 PC 中。八颗 SRAM\footnote{Static RAM.} 黑色芯片提供了总计 256 KiB 的 L2 “缓存”。在防止 CPU 数据与指令饥饿的新系统中,SRAM 速度更快(访问时间 20ns,比 DRAM 快 10 倍),但也有双重缺点:制造成本高得多、密度低于 DRAM,从而限制了其用途。



% \begin{enumerate}
% \item RAM prices had dropped significantly. The standard 2 MiB was not a whooping 4 MiB. 
% \item Bandwidth hungry GUI and had lead motherboard manufacturers to come up with a faster bus called VLB.
% \item The sound generator ecosystems was even more fragmented than before with more and more manufacturer producing sound cards.
% \item Not visible on the drawing, the operating system shortcomings were being addressed by independent developers via something called "DOS eXtenders".
% \item Unsurprisingly the impossible to program VGA was still the standard but manufacturers were now competing to produce the faster DACs and \fixme{"RAMDAC"}?.
% \item CACHE
% \end{enumerate}
\pagebreak
\trivia{当时一些最令人惊叹的游戏工作室押注于 RAM 价格将继续下降。1994 年最出色的一批作品(包括 \doom{})要求至少 4 MiB 内存。Strike Commander、Ultima 8 和 Comanche Maximum Overkill 就是典型例子。
\par
\begin{center}
\scaledimage{0.32}{box_strike_commander.png}  \scaledimage{0.32}{box_ultima8.png} \scaledimage{0.32}{box_comanche.png}
\end{center}
}
讽刺的是,这一时期最终与 1994 年严重的 RAM 短缺重合,使价格再次上升。传说这次上涨源于台湾一家树脂工厂失火。更可能的原因是微软宣布 Windows 95,后者推荐至少 8 MiB RAM 的机器。

\begin{center}
\begin{tikzpicture}

\begin{axis}[
    width=1.0\textwidth,
    height=0.55\textwidth,
    title={1990-2000 年 1 MiB RAM 的平均价格\protect\footnotemark},
    xlabel={年份},
    xtick pos=left,
    ytick pos=left,
    ylabel={美元价格},
    yticklabel={${\$\pgfmathprintnumber{\tick}}$},
    xticklabel style={ rotate=30,},
    xticklabel style={/pgf/number format/1000 sep=},
    xmin=1990, xmax=2000,
    ymin=0, ymax=104,
    %xtick={0,20,40,60,80,100},
    %ytick={0,20,40,60,80,100,120},
    legend pos=north west,
    ymajorgrids=true,
    grid style=dashed,
]
 
\addplot[
    color=black,
    mark=*,
    ]
    coordinates {
    (1990,59)
      (1991,43)
      (1992,34)
      (1993,27)
      (1994,35)
    (1995,32)
      (1996,5)
      (1997,4)
      (1998,0.9)
      (1999,0.98)
      (2000,1.22)
    
    };
 
\end{axis}
\end{tikzpicture}
\end{center}
\footnotetext{来源:John C. McCallum 调查。}
