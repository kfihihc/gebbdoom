PC 环境在 1991 年开发 Wolfenstein 3D 与 1993 年开发 \doom{} 之间发生了显著变化。之前基于 Intel 386、2~MiB 内存与 VGA 显卡的“推荐配置”已成为过去。\\
\par
“新的”顶级 PC 仍由六个子系统组成:\circled{1}~输入、\circled{2}~总线、\circled{3}~CPU、\circled{4}~RAM、\circled{5}~视频输出与 \circled{6}~音频输出,共同形成一条流水线。每个子系统都变得更快或容量更大。\\
\par
\vspace{2mm}
\drawing{pc_components}{IBM PC 的六大组件。}
\par
在详细介绍各组件之前,需要一个重要说明。本章以“增量”的方式对比 \doom{} 与 Wolfenstein 3D 时代的硬件,因此会给人一种“力量充沛”的错觉。\\
 \par
尽管接下来会列出令人印象深刻的改进清单,但要记住 IBM PC 仍不是适合电子游戏的机器。它们的设计目标是办公,因此充满限制:用于文字处理、表格计算,也许偶尔显示静态图表——从未打算做能在 70Hz\footnote{VGA 最常见的游戏刷新率(320x200)是 70Hz,不同于如今普遍的 60Hz。} 实时动画的系统。\\ 
\par 
回头看,很难相信工作室会专注于一台显然比主机更弱的机器来制作游戏。问题清单相当长:
\begin{itemize}
\item CPU 无法执行浮点运算,也没有协处理器\footnote{如 Amiga 的 Paula、Denise 与 Agnus。}。
\item 古老的图形系统似乎无法双缓冲,且与显示器纵横比不匹配,导致图像失真。
\item 事实上的声音系统只能发出刺耳的“哔哔”声;即便用户购买声卡,生态也极其碎片化。
\item 顶级机型价格接近 3,000 美元。作为对比,SNES 与 SEGA Genesis 均为 199 美元,而提供街机体验的 Neo-Geo 为 649.99 美元\footnote{按 2018 年通胀调整后,PC 约为 10,476 美元,SNES/Genesis 为 377 美元,Neo-Geo 为 1,134 美元。}。
\end{itemize}
\par
PC 充其量并不吸引人,似乎也更难产生好游戏,尤其是相比更便宜、为 60Hz 动画与精灵引擎而设计的系统。\\
 \par
当然,正如你手中书名所暗示的,借助一些软件技巧,PC 硬件能做到远超其设计目标的事情。PC 不擅长某些类型的游戏,但在需要 framebuffer 的类型上可以非常出色。在没有互联网且文档稀少的时代,这远非易事。\\
\par

对页图 \ref{ibm_ps1_top} 复刻了 90 年代电脑杂志中随处可见的一种广告。请注意,广告中的 IBM PS/1 配备 Intel 486 CPU,强调办公与运行静态办公应用的能力。生产力与利润是唯一能为高价机器辩护的理由——它相当于 1993 年美国家庭年收入中位数的 5\%\footnote{statista.com 记录 1993 年美国家庭收入中位数为 52,335 美元;Byte Magazine 1993 年春季广告显示 486 DX2-66 VESA PC 价格为 2,575 美元。}。\\
\par

 % (\$2,000\footnote{\$13,000 in 2017.}). Also worth mentioning, the "huge" standard 14" CRT which must have weighted close to 20 pounds!\\
\par
\begin{figure}[H] \centering
\fullimage{ibm_ps1_top}
\caption{约 1993 年的 IBM PS/1 广告。注意那个小得离谱的 14" CRT 标配显示器,分辨率最高只有 800x600。}
\label{ibm_ps1_top}
\end{figure}














\cleartoleftpage
 
\cfullimage{PX486P3/b-486-px486p3_romain}{QDI Computer, Inc. 的 PX486P3 主板。}
获取当时硬件概览的实用方式是打开一台 PC,看看把一切连接在一起的组件。1994 年最畅销的主板是 QDI Computer, Inc 的 PX486P3\footnote{一家加拿大公司 \scaledimage{0.03}{Canada}!}。\\

\par
最显眼的新特性当然是电脑的心脏——Intel i486 CPU \circled{1}。仔细观察还能发现许多对 \doom{} 架构至关重要的特性。\\
\par 
黑色连接器显示传统 ISA 总线扩展口。一个 8 位 \circled{2} 与三个双槽 16 位 \circled{3} 共支持四张 ISA 卡。还有三组新型连接器,带一个棕色附加插槽 \circled{4}。这些是 VLB 插槽\footnote{又称 VL-Bus、VESA Local Bus。},总线速度可比 ISA 快 10 倍。

\drawing{px486p3}{PX486P3 的组件示意图。}
\par
左上角 \circled{6} 是系统主存,其容量、速度与复杂度都提升了。得益于制造成本大幅下降,标配 DRAM\footnote{Dynamic RAM.} 达到 4 MiB\footnote{\doom{} 无法在仅 2 MiB 的 PC 上运行。}。\\
\par
最后,右上角 \circled{5} 出现了一种新型 RAM。八颗黑色 SRAM\footnote{Static RAM.} 芯片共计 256 KiB,作为 L2 “缓存”。它用于防止 CPU 数据与指令饥饿的新系统中。SRAM 的速度更快(访问时间 20ns,是 DRAM 的 10 倍),但缺点是制造成本更高且密度更低,限制了使用范围。



% \begin{enumerate}
% \item RAM prices had dropped significantly. The standard 2 MiB was not a whooping 4 MiB. 
% \item Bandwidth hungry GUI and had lead motherboard manufacturers to come up with a faster bus called VLB.
% \item The sound generator ecosystems was even more fragmented than before with more and more manufacturer producing sound cards.
% \item Not visible on the drawing, the operating system shortcomings were being addressed by independent developers via something called "DOS eXtenders".
% \item Unsurprisingly the impossible to program VGA was still the standard but manufacturers were now competing to produce the faster DACs and \fixme{"RAMDAC"}?.
% \item CACHE
% \end{enumerate}
\pagebreak
\trivia{当时一些最令人惊叹的游戏工作室押注了 RAM 价格下降。1994 年最出色的游戏(包括 \doom)要求至少安装 4 MiB。Strike Commander、Ultima 8 与 Comanche Maximum Overkill 是代表。
\par
\begin{center}
\scaledimage{0.32}{box_strike_commander.png}  \scaledimage{0.32}{box_ultima8.png} \scaledimage{0.32}{box_comanche.png}
\end{center}
}
讽刺的是,这一时期恰好遇上 1994 年内存大短缺,价格反而回升。传说原因是台湾一家树脂工厂失火。更可能的原因是微软宣布 Windows 95,该系统推荐至少 8 MiB RAM 的机器。

\begin{center}
\begin{tikzpicture}

\begin{axis}[
    width=1.0\textwidth,
    height=0.55\textwidth,
    title={1990-2000 年 1 MiB RAM 的平均价格\protect\footnotemark},
    xlabel={年份},
    xtick pos=left,
    ytick pos=left,
    ylabel={美元价格},
    yticklabel={${\$\pgfmathprintnumber{\tick}}$},
    xticklabel style={ rotate=30,},
    xticklabel style={/pgf/number format/1000 sep=},
    xmin=1990, xmax=2000,
    ymin=0, ymax=104,
    %xtick={0,20,40,60,80,100},
    %ytick={0,20,40,60,80,100,120},
    legend pos=north west,
    ymajorgrids=true,
    grid style=dashed,
]
 
\addplot[
    color=black,
    mark=*,
    ]
    coordinates {
    (1990,59)
      (1991,43)
      (1992,34)
      (1993,27)
      (1994,35)
    (1995,32)
      (1996,5)
      (1997,4)
      (1998,0.9)
      (1999,0.98)
      (2000,1.22)
    
    };
 
\end{axis}
\end{tikzpicture}
\end{center}
\footnotetext{来源:John C. McCallum 调研。}






