\begin{wrapfigure}[5]{r}{0.35\textwidth}{
\centering \scaledimage{0.35}{jaguar_logo.png}}
\end{wrapfigure}
Jaguar 的开发始于 1990 年,当时 Atari 委托剑桥的 Flare Technology 同时设计两套新游戏系统:一套第四代 32 位系统“Panther”,以及大胆的 64 位系统“Jaguar”\footnote{Atari 以大型猫科命名其主机。除 Jaguar 与 Panther 外,1989 年的掌机名为“Lynx”。}。
\par 三年后,Jaguar 项目进度领先,Atari 决定放弃 Panther,并于 1993 年 11 月发布 64 位 Jaguar。\\
\par
\fullimage{consoles/Jaguar.png}
\par
Flare Technology 的 Martin Brennan、Ben Cheese 与 John Mathieson 做出了一些强烈主张的设计决策。除 18 键手柄外,该主机拥有不少于五颗处理器需要协调。\\
\par
音频方面配有一颗 32 位 27MHz RISC CPU,绰号 “Jerry”。视频方面则有三颗处理器,均集成在 32 位 27MHz RISC 芯片 “Tom” 中,包括 GPU、blitter 与 object processor。协调全局的是一颗 16/32 位 13MHz 68000,并配有 2MiB RAM。将一切连接起来的,是让市场部门兴奋不已的 64 位数据总线。\\
\par
“Do the Math!” 这条大胆的 64 位广告标语引发了潜在用户的质疑。无论 John Mathieson 在访谈中如何解释,这台机器看起来都像是已经令人怀疑的 Atari 市场部在误导消费者。Atari 怎么可能造出比 16 位 Super Nintendo 与 Sega Genesis 强四倍的主机,这一点并不清楚。
\drawing{jaguar_arch}{Jaguar 的架构。注意不均匀的总线。}
\par
谨慎的消费者与难以置信的开发者对峙。五处理器架构很强大,却对习惯于 8 位 NES 或 Sega Master System 单处理器的开发者来说极其反常。为 Jaguar 编程是一门艺术,很少有人愿意花时间掌握。\\
\par
首发游戏数量有限,阻碍了玩家规模的形成。低销量让开发者更不愿投入,进而进一步影响销量。在其三年寿命里,Atari 仅售出约 10 万台。


\fullimage{jagua_motherboard.png}\\
\par
机器内部:\circled{1} Motorola 68000,\circled{2} 2MiB RAM,\circled{3} JERRY,\circled{4} TOM,\circled{5} 操作系统 ROM,\circled{6} 卡带插槽,\circled{7} 手柄接口,\circled{8} 电源适配器接口,\circled{9} DSP 接口,\circled{A} 显示输出(复合、分量与 S-Video),以及 \circled{B} 频道切换、电视端口。\\
\par
\fq{Jaguar 具有 64 位内存接口,以便从廉价 DRAM 获取高带宽。……系统需要 64 位时,它就是 64 位,因此从 DRAM 获取数据并构建显示的 Object Processor 是 64 位;负责 3D 渲染、清屏与像素搬移的 blitter 也是 64 位。系统不需要 64 位的部分就不是 64 位。游戏主机没有 64 位地址空间的意义!3D 计算与音频处理一般不使用 64 位数字,因此 64 位处理器没有优势。}{John Mathieson}



\rawdrawing{jaguar_motherboard}
\par
John Mathieson 接受了许多访谈,提供了更多硬件工程师所面临的约束,从成本压力到 Atari 要求使用 Motorola CPU 等。硬件端的草地并不比软件端更绿。\\
\par
\fq{Atari 很希望使用 68K 系列器件,我们仔细研究了多个成员。我们确实做过 68030 版本的早期 Beta 开发系统,一度打算用 68020。但后来发现成本太高。我们也考虑过完全不用 [Motorola 680x0 芯片]。我始终觉得需要一个“正常”的处理器,能让开发者刚开始时感觉安心。68K 便宜且能很好地完成这份工作。}{John Mathieson}






\subsection{Jaguar 编程}
释放这只野兽意味着要让五颗处理器并行工作\footnote{来源:“Jaguar Technical Reference Manual: Tom \& Jerry”。}。理论上复杂,实践中更复杂。\\
\par
\fq{68000 可以说是“CPU”,因为它负责引导机器并启动其它一切;然而它并不是 Jaguar 的力量中心。……68000 就像一位不干实事的经理,只是指挥别人干活。\\
   \par
     我认为它的存在只是为了读取手柄。}{John Mathieson}

\subsubsection{理论}
\par
Motorola 68000 作为管理者,负责对外部世界的交互并管理其他处理器的资源。它处于最高控制层,对系统拥有完全控制。\\
\par
Object Processor 连接电视,负责生成显示行。它读取通常由像素构成的对象列表(可重叠),完成传统精灵引擎的所有功能。其 16 位每像素的 CRY(Cyan-Red-Yellow)颜色模型在当时主机中非常另类。一个字节是将 sRGB 立方体压扁成方形后的 (X,Y) 坐标,给出颜色;另一个字节给出亮度,总计 65,536 色。\\
\par
\fullimage{CRY.png}\\


\fullimage{CRY_persp.png}
\par
Graphics Processor 是高吞吐、强 ALU 的主力,拥有并行乘法器、桶形移位器与除法单元,以及常规算术功能。其内部 4KiB SRAM 用于保存数据与本地程序指令。\\
\par
Blitter 负责快速 RAM 块移动与填充,可生成 Gouraud 着色的 Z 缓冲多边形像素条,也可基于 Z 测试跳过像素。它能够进行位图旋转、画线、字符绘制等多种效果,并负责把本地数据与指令加载到 Tom \& Jerry 的 SRAM。\\
\par
Jerry(DSP)是 Graphics Processor 的孪生兄弟。更大的本地 SRAM(8 KiB)与更小的 32 位连接到 64 位主总线,使其适合生成音乐与音效。不过,程序员也可让它执行其他任务,包括图形。其灵活性之大体现在:连接两台 Jaguar 的 Jag-Link 直接插在背后的 “DSP 端口” 上,由 Jerry 负责网络。\\
\par
所有 RAM(包括 RISC CPU 内的 SRAM)都可被任何组件访问,依赖灵活的内存控制器。执行过程中任何处理器都可成为 DMA 总线主控。尽管 68000 是“总管”,在 DMA 主控冲突时它却优先级最低,DSP 反而是“王”,以避免糟糕的音频毛刺。
\par



\subsubsection{实践}
硬件存在不少 bug,尤其是内存控制器层面,使得多任务难以依赖,调试更是困难重重。\\
\par 
乍看不明显,但 Motorola 68000 与 Tom/Jerry 采用不同架构与指令集。预期工作流是使用 C 语言编程 Motorola,而 GPU/DSP 的 RISC 路径则更繁琐:程序员必须先学习新指令集,再写汇编代码,用汇编器生成机器码,最后构建完整流水线,将机器码存储并交付到 Tom \& Jerry 的本地 SRAM。\\
\par







\subsection{\doom{} 在 Jaguar 上}
John Carmack 很早就对 Jaguar 感兴趣,并亲自完成移植,Dave Taylor 负责音效与 MIDI 音乐。这并非 John 首次接触该机。\\
\par
\fq{我心血来潮移植了 Wolfenstein 3D。我在想 Jaguar 的硬件还能用在 Doom 之外的哪些游戏里,Wolfenstein 看起来是个很好的利用。我某天下午开始写,烧了 15 张 CD,第二天早上其他人来上班时,我已经有了一个可运行的 Jaguar Wolfenstein 版本。我们把它寄给 Atari,他们同意让 Doom 稍微延期,好让 Wolfenstein 先行发布。}{John Carmack for EDGE Magazine, June 1994}\\
\par
对于 \doom,二人组在签署移植协议两周后就跑起来了,尽管帧率糟糕\footnote{来源:EDGE Magazine, 1994 年 6 月。}。\\
\par
\trivia{为简化 Tom \& Jerry 的 RISC 指令生成,John Carmack 编写了自己的 \cw{lcc} 编译器后端。输出还被手工进一步优化\footnote{这不是 id Software 最后一次使用优秀的 \cw{lcc}。1999 年它用于 Quake 3 生成 VM 字节码。}}\\
\par
在一台 RAM 只有 PC 版本 50\% 的机器上运行非常困难。Cyberdemon 与 Spiderdemon 被删。精灵与纹理分辨率降低。地图被大量修改以减少纹理、线段与可见平面。请看图 \ref{doom_jaguar3.png} 中的 E1M1,并与 PC 版(第 \pageref{mashed_potatoes1.png} 页)比较:蓝色地面纹理消失,台阶从两阶变为一阶。\\
\par
3D 渲染器必须重写,以适配在 RISC 上运行的小型汇编块。运行时根据引擎“阶段”在 SRAM 中切换九个覆盖块。\\
\par
\cfullimage{doom_jaguar3.png}{}
\par
\vspace{-12pt}
Jaguar 版 \doom{} 的源代码于 2003 年 5 月由 Songbird Productions 的 Carl Forhan 发布。查看其中可发现内存占用缩减的细节,例如 visplane 从 128 缩减为 64。\\
\par
\ccode{jaguar_visplanes.c}
\par
一个重要决定是取消游戏内音乐播放。这是因为 Tom 无法独立承载游戏引擎性能。为解决此问题,Jerry(DSP)被用于碰撞检测。幸运的是,Jerry 仍有余力播放音效,引擎最终稳定在 20 FPS。\\
\par
3D 渲染分辨率为 160x180,列重复后达到 320x180,并在底部留出 60 像素状态栏,总分辨率 320x240。在很多方面,画面甚至优于 PC。凭借定制 16 位 CRY 模式,Jaguar 实际上拥有 65,536 色,且无色带。




\fullimage{doom_jaguar2.png}
\par
进一步查看源代码还能发现一些来自 Sega 32X 版本的遗留(\cw{MARS} 是 Sega 32X 的代号)。我们还可以得知开发是在 NeXTSTEP 的 “Simulator” 模式下完成的。\\
\par
\ccode{jaguar_src.c}
\par

\fq{内存、总线、blitter 与视频处理器是 64 位的,但处理器(68k 与两颗定制 RISC)是 32 位。\\
\par
blitter 可以做基本的水平与垂直跨度纹理映射,但因为没有缓存,每个像素都会造成两次 RAM 页缺失,只用了 64 位总线的 1/4。两个 64 位缓冲区本可以轻松让纹理映射性能提升三倍。可惜了。\\
\par
它可以用 Z 缓冲的着色三角形更好地利用 64 位总线,但这并不会带来更吸引人的游戏。\\
\par
它提供了一个有用的颜色空间选项,让你可以基于单一通道做光照效果,而不是 RGB。\\
\par
视频合成引擎是主机最具创新性的部分。Wolf3D 中所有角色都只用后台 scalar 完成,而不是 blitting。尽管如此,极限与失败案例的经验让我有足够弹药去反对微软(幸好被取消的)talisman 项目。\\
\par
小型 RISC 引擎是不错的处理器。我惊讶他们没用现成设计,但基本可用。它们有一些设计隐患(写后写)未修复,但真正的问题是:它们使用 scratchpad 内存而非缓存,且无法从主内存执行代码\footnote{在主内存中执行 \codeword{jmp} 会失败(见 Jaguar 技术参考手册第 137 页)!}。我不得不将 DOOM 渲染器拆成九个依序加载的覆盖块来运行(事后看来,我会在三……个不同的方案中做得更好)。\\
\par
68k 很慢,这是系统的主要问题。你的选择是要么轻松点,所有东西跑在 68k 上,速度很慢;要么在 RISC 上加班加点,通过大量叠加并行汇编块来提升速度。\\
\par
这也是 PlayStation 在开发上如此强势的原因——它的编程方式像单一串行处理器搭配一个快速加速器。若 Jaguar 去掉 68k,为 RISC 处理器提供动态缓存,并在 blitter 上增加一点缓冲,它完全可以与索尼有一战之力。} {John Carmack}
