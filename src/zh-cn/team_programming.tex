\vspace{-10pt}
\section{编程}
从 DOS 上的 Borland C++ 编辑器迁移到 NeXTSTEP 的 TextEdit,是一种权衡。一方面,语法高亮等便捷功能没了;另一方面,应用从不崩溃,宝贵的工作时间不会丢失。TextEdit 还提供标记(\cw{//}),可用于“折叠”代码,改善导航和可读性。\\
\par
 更高的分辨率(1120 x 832)让 MegaDisplay 可以纵向显示更多代码,并能并排显示最多三个 DOS 80 列窗口。注意 Borland 的 IDE(默认模式\footnote{Borland 的 C++ IDE 可以切换到 80 列模式显示 50 行,但可读性会大幅下降。})只能显示 21 行代码,而 TextEdit 可以显示 57 行。\\
\par
\fullimage{development.png}\\

\vspace{-4mm}
\cfullimage{TextEditApp.png}{TextEdit.app 可通过特殊标记折叠分节}


在图 \ref{TextEditApp.png} 中,TextEdit 的标记与折叠显示了 \cw{d\_main.c} 的 1181 行中的将近 700 行。这一功能让开发者能够以系统为单位思考,而非以函数为单位。

\subsection{Interface Builder、OOP 与 Objective-C}
工具列表中不能不提 \NeXTns{} 的王牌应用——Interface Builder。它被许多人视为 NeXTSTEP 的杀手级应用。\\
\par
“IB” 最初由 Jean-Marie Hullot 在 1984 年用 Lisp 编写,并在 1986 年以 “SOS Interface” 名称商业化\footnote{来源:“A Brief History of Human Computer Interaction Technology”。}。Hullot 后来加入 NeXT, Inc.,并与团队一起用 Objective-C 打造了类似但更专注的工具。\\
\par
他制作的 NeXTSTEP 版本将 GUI 开发成本降低了 5-10 倍\footnote{来源:“NeXT vs Sun: A world of a difference”,1991 年宣传视频。}。\\
\par
\fullimage{ib.png}



\vspace{-4mm}
借助 IB,GUI 编写中所有繁琐工作都能用鼠标飞快完成。“乏味”的部分——编写代码——只在业务逻辑层真正需要时才必须做。创建 GUI 分为两步:先绘制元素,再将 UI 元素连接到对象模型。\\
\par
第一步正如 Steve Jobs 所形容的那样“疯狂简单”,因为构建界面只需从控件面板拖拽控件到画布上即可。检查器能显示元素属性,从滑块的最小/最大值到文本框的默认值都可轻松调整。\\
\par
第二步——将可视元素连接到业务逻辑对象——同样用鼠标完成,把可视化的方框连接到目标/动作。目标可能是复选框的布尔属性,也可能是按钮的一个方法。
%\par
%The developer could focus its time on implementing only the business logic.\\

\subsubsection{面向对象编程}
除了革命性的设计外,IB 还与一种程序员友好的语言——Objective-C——的 OOP(面向对象编程)理念相得益彰。\\
\par
\fq{在我从业 20 年里,从未见过像面向对象编程这样深刻的革命。你可以以 5 到 10 倍的速度开发软件,而且软件更可靠、更易维护、功能更强大……所有软件终将采用这种对象技术。毫无疑问。}{Steve Jobs,《Rolling Stone》,1994 年 6 月 16 日。}\\
\par
OOP 的封装、继承与多态,让程序员能处理更高复杂度。程序可被概念化为一组可能嵌套的子系统。思维模型不必是复杂的整体块,可以拆分成更小、更易概括的黑箱系统。\\

\par
\vspace{-10pt}
\subsubsection{Objective-C}
Objective-C 与 \verb!C++! 差不多同一时期开发。不过如共同发明者 Brad Cox 回忆,他的作品与 Bjarne Stroustrup 的想法在哲学上截然不同。C++ 以性能为先,Objective-C 则更重视程序员生产力。\\
\par
\fq{
	回到 1980 年,当我们的语言都在构建中时,我去拜访了 Bjarne Stroustrup。我们对语言设计有截然不同的看法。关键在于机器效率与程序员效率的相对重要性。最终我们同意彼此不同。
} {Brad Cox(采访自《The NeXT Bible》)}\\
\par


\par
\vspace{-5pt}
随 NeXT 电脑出货的版本包含一个重要库,名为 Foundation Kit。其组件之一 \cw{NSObject} 通过 \cw{retain} 与 \cw{release} 的引用计数机制,让开发者摆脱易错的内存管理负担。另一组瑞士军刀式容器——\cw{NSArray}、\cw{NSDictionary}、\cw{NSSet} 和 \cw{NSData}——进一步让开发者聚焦核心功能而非基础设施。\\
\par
\trivia{Foundation 的对象最初使用 NX 前缀,后来在 OpenSTEP 中改为 NS,意为 NeXT 与 Sun Microsystems 的联盟。它们至今仍是 macOS 与 iOS 的核心,NS 前缀从未被移除。}\\
\par
Objective-C 架构的核心是通过调度方法 \cw{objc\_msgSend} 路由消息,使程序更能容错。最令人惊叹的能力(至少对 C++ 开发者来说)是向 \cw{nullptr} 发送消息。当开发者用以下语法向对象发送消息时:\\ \par
\objccode{obj.message}
幕后实际发生的是调用分发器。\\
\par
\objccode{objc_msgSend.m}
这段高度优化\footnote{来源:“Dissecting objc\_msgSend on ARM64”,Mike Ash。}、手写的汇编会在 NeXT 启动时被调用数百万次。尽管 ObjC 对象具有复杂可变特性(方法可在运行时添加,改变其鸭子类型),\cw{objc\_msgSend} 仍能在运行时沿继承链查找正确目标。更重要的是,它还能检测 \cw{nullptr} 并返回 \cw{0},而不是让进程崩溃。\\
\par

总而言之,NeXT 开发的四大支柱(Unix、IB、OOP 和 Obj-C)让 DoomED、doombsp 与 wadlink 的开发速度远超 DOS 工具\footnote{多年后,John Carmack 在推广静态代码分析时曾抱怨 ObjC 等动态语言带来的“随意编程”并非好事。}。
