\section{文件系统}
\doom{} 与操作系统的文件系统几乎没有交互。在典型的游戏过程中,引擎 \cw{DOOM.EXE} 只需要打开 \cw{DOOM.WAD} 来访问资源。因此,引擎处理的不是“文件”,而是称为 lump 的东西——它是 \cw{.wad} 档案以及其缓存系统的原子单位。\\ 
 \par
 尽管 \cw{DOOM.EXE} 相对稳定,资源档案却不断膨胀,每款新游戏都比前一款更复杂,如下列 \cw{.wad} 所示。\\
 \par
 \begin{figure}[H]
\centering  
\begin{tabularx}{\textwidth}{ L{1.3} L{0.7} R{1} R{1}}
  \toprule
  \textbf{游戏} &  \textbf{档案名} & \textbf{\# Lumps} & \textbf{字节大小}\\

  \toprule 
  Doom Shareware          & \cw{DOOM1.WAD}    & 1264 & 4,196,020 \\
  Doom Registered         & \cw{DOOM.WAD}     & 2194 &11,159,840 \\
  Doom II: Hell on Earth  & \cw{DOOM2.WAD}    & 2918 &14,604,584\\
  Ultimate Doom           & \cw{UDOOM.WAD}.   & 2306 &12,408,292\\
  The Plutonia Experiment & \cw{PLUTONIA.WAD} & 2984 &17,420,824\\
  TNT: Evilution          &  \cw{TNT.WAD}     & 3101 &18,195,736\\
  French Doom II          & \cw{DOOM2F.WAD}   & 2913 &14,607,420\\
   \toprule
    Heretic Shareware     & \cw{HERETIC1.WAD} & 1374 &5,120,920\\
    Heretic Registered    & \cw{HERETIC.WAD}  & 2633 &14,189,976\\
   \toprule
  Hexen Demo         & \cw{HEXENDEMO.WAD}& 2856 &10,644,136\\
  Hexen Registered        & \cw{HEXEN.WAD}    & 4270 &20,083,672\\
  Hexen Deathkings of DC  & \cw{HEXDD.WAD}    &  326 & 4,440,584\\
   \toprule
\end{tabularx}
\caption{\doom{} 各版本中的 WAD 文件\protect\footnotemark}
\end{figure}
\par
\footnotetext{来源:doomgod.com “Internal War Allocation Daemons”}
\par
\trivia{无需使用 \cw{I\_*} 抽象层来访问操作系统文件系统。幸运的是,当时所有系统都提供了 \cw{open}、\cw{lseek}、\cw{read} 和 \cw{close} 等标准函数。}\\
\par

lump 通过一个最多 8 个字符的唯一名称标识(恰好符合 DOS 文件名长度限制)。\\
\par
\ccode{lumpinfo_t.c}{}
\par
lump 类型超过三十种。与地图和音乐关联的 lump 遵循命名约定,因此可以分组在一起。并非所有 lump 都有内容,有些只是用来标记 lump 组的起止。
\pagebreak

\begin{figure}[H]
\centering  
\begin{tabularx}{\textwidth}{ L{0.2}  L{0.8}}
  \toprule
  \textbf{Lump 名称} &  \textbf{用途} \\
   
  \toprule 
  \cw{PLAYPAL} & 运行时使用的 14 个调色板。详见第 \pageref{label_palettes} 页。 \\
  \cw{COLORMAP} & 用于模拟 256 种颜色各 32 种亮度的转换表。详见第 \pageref{diminishedlightning} 页。 \\
  \cw{DEMO?} &  id Software 成员录制的游戏会话。在游戏启动时作为“街机风格”演示播放。\\
  \toprule
  \cw{EXMY} & 零大小的 lump,作为地图 lump 系列起始标记。\cw{X} 为章节号,\cw{Y} 为地图编号。之后还使用了 \cw{MAPXY} 变体达到同样目的。\\
  \cw{THINGS} & 当前地图中的所有怪物、武器、弹药与精灵。\\
  \cw{LINEDEFS} & \cw{SECTORS} 引用的所有线。\\
  \cw{SIDEDEFS} & \cw{LINEDEFS} 引用的所有“面”。一条线可以有 1 或 2 个面。\\
  \cw{VERTEXES} & 当前地图中的所有顶点。\\
  \cw{NODES} & 允许高效排序线段的二叉树。 \\
  \cw{SSECTORS} &  子扇区,\cw{NODES} 二叉树的叶子。  \\
  \cw{SEGS} &  \cw{SSECTORS} lump 指向的线段。\\
  \cw{SECTORS} &  \cw{SSECTORS} 引用。指定天花板/地板高度、纹理和光照属性。\\
  
  \cw{BLOCKMAP} & 将地图切分为 128x128 块的碰撞检测加速结构。提供对地图任一点附近 \cw{LINEDEFS} 的快速访问。详见第 \pageref{blockmapdetails} 页。 \\
  \cw{REJECT} &  视线加速数据结构。\\
  \toprule
  \cw{DP.*} &  PC Speaker 格式的音效。\\
  \cw{DS.*} &  PCM 单声道,8-bit 11kHz 的音效(支持 22kHz,但仅用于 \doomii{} 的超级霰弹枪)。\\
  \cw{D\_.*} & MUS 格式的音乐(一种略微修改过的 MIDI 格式)。\\
  \toprule
  \cw{ENDOOM} & 文本模式退出画面,用来吸引玩家购买完整版。 \\
  \cw{DMXGUS} & 将 MIDI 乐器与 Gravis Ultra Sound 采样文件匹配的转换表。\\
  \cw{GENMIDI} &  使用 OPL 音频芯片播放 MIDI 音乐的乐器库。\\
  \cw{PNAMES} &  列出所有用于墙面补丁的 lump 名称。\\
  \cw{TEXTURE1} &  \cw{SIDEDEFS} 引用的所有墙纹理 lump 的字典。用于在运行时加速访问与分配。\\  
  \cw{F\_START} &  零大小的 lump,标记平面纹理的开始。\\  
  \cw{F\_END} &   零大小的 lump,标记平面纹理的结束。\\  
  \cw{S\_START} & 零大小的 lump,标记物品/怪物“精灵”段的开始。 \\
  \cw{S\_END} & 零大小的 lump,标记物品/怪物“精灵”段的结束。 \\
  \cw{P\_START} & 零大小的 lump,标记墙面纹理的开始。\\
  \cw{P\_END} & 零大小的 lump,标记墙面纹理的结束。\\
  \cw{.*} &  还有许多其他类型,如字体、\cw{TITLEPIC}、\cw{HELP} 屏幕、过场屏幕、\cw{VICTORY} 屏幕等…… \\  
   \toprule
\end{tabularx}
%\caption{WAD lump types}
\end{figure}
\par
\pagebreak






\subsection{Lumps} \label{wad_detailled}
Lump 系统是引擎中最不炫的一部分,但其实现与提供给玩家的能力却非常酷。\\
\par
启动时,它会查看提供的每个 WAD 档案,把找到的每个 lump 索引到一个巨大的 \cw{lumpinfo\_t} 数组中,狡猾地命名为 \cw{lumpinfo}。
如果通过命令行参数 \cw{-file} 提供额外档案,属于官方 id Software WAD(\cw{DOOM.WAD}、\cw{DOOM2.WAD} 等)的 lump 会先被加入 \cw{lumpinfo} 数组。\\
\par
在下例中,\doom{} 通过命令行启动:\\
\par
\fakedosoutput{dos_doom_wad}
\par
在此示意图中,\cw{DOOM.WAD} 仅用三个 lump 表示,另外两个 WAD 档案各只有一个 lump。注意 \cw{DOOM.WAD} 的条目排在最前,而且一个 lump 名称可以在索引中出现多次(这里有两个名为 \cw{MUSIC1} 的 lump)。\\
\par
\drawing{lumpsMaster}{Lump system index}
\par
对 lump 系统的请求以 \cw{char[8]} 名称发送。第一步是把 lump 名称关联到一个 \cw{int} 类型的 lump 索引 ID。在 \cw{W\_CheckNumForName} 函数中,lump 数组被顺序搜索,为了加速比较,它使用了一个很酷的技巧:不再逐个比较 8 个字符,而是比较两个 32 位整数。\\
\par
\ccode{W_CheckNumForName.c}{}\\
\par
在上面的代码中,你会注意到索引从末尾开始搜索。这是故意的,以便 modder 可以在自己的 WAD 档案中提供资源来覆盖 id Software 的 lump。\\
\par
这个系统的妙处在于,原始的 \cw{DOOM.WAD} 永远不必被修改或打补丁。任何资源都可通过简单的命令行覆盖——从地图、音乐、音效到图形\footnote{唯一的例外是 AI 和地图名称,它们在可执行文件中被硬编码。}。\\
\par
\trivia{为了区分官方 WAD 与玩家自制 WAD,WAD 档案开头的魔数不同。\cw{IWAD} 保留给 id Software,而自制 WAD 被要求使用 \cw{PWAD}。}\\
\par
一旦找到 lump 的位置,就会向内存分配器请求一个内存块。lump 内容从硬盘复制到 RAM 并返回给调用者。 \\
\par
\cw{lumpinfo} 数组由一个 \cw{lump} 缓存系统镜像。当请求某个 lump 时,会先用索引 ID 在 \cw{lumpcache} 数组中查找(在 \cw{W\_CacheLumpNum} 中)。非空指针表示该 lump 已经在一个 zone 块中。lump 缓存槽会把自己设置为该内存块的 \cw{user},意味着一旦块被释放,缓存会自动失效。这也解释了默认的 \cw{owner} 值为何设为 2——它表示该内存块被占用但未缓存(因此释放时没有缓存需要失效)。在图 \ref{lumpcache} 中,lump \cw{0} 与 \cw{2} 不在缓存中,将请求访问 WAD。\\
\par
\scaleddrawing{0.9}{lumpcache}{Lump \cw{1} 在缓存中。Lump \cw{0} 和 \cw{2} 不在缓存中。}
\par
\ccode{W_CacheLumpName.c}


id Software 曾向社区中的几个人解释过资源文件格式。不到一个月,“Unofficial DOOM specs” 就在网上发布,它全面描述了 WAD 格式。有了已知格式和注入新 lump 的方法,mod 社区迅速繁荣。\\
\par

有些粉丝几乎替换了原作的所有方面。这些 mod 被称为“Total Conversions”(TC)。其中最著名的是 Aliens Total Conversion。\\
\par
它于 1994 年 12 月发布,Justin Fisher 用了一年时间完成。好玩的是,它回到了 id Software 一度考虑过、但后来改成恶魔的《异形》电影主题。\\
\par
 \fullimage{alientc.png}
 \par
 \vspace{10pt}
 许多音效,如门声、武器声与爆炸声,都直接来自电影。演员台词(“let's rock!”)和尖叫声被数字化处理。所有恶魔都被替换成异形、卵、抱脸虫,甚至异形女王。Pulse Rifle、Grenade Launcher 和 Smart-Gun 都在,连电锯也被替换成 “Caterpillar P-5000 Work Loader”。地图设计也没有被忽视。Aliens TC 在第一关完全没有敌人,成功复刻了电影那种偏执且恐怖的氛围!
