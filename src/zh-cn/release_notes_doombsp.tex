1994 年 5 月,\cw{doombsp} 的源代码被发布。和其他发布一样,John Carmack 写了一段简短的说明。\\
\par
\vspace{10pt} 
\hrule \par
\begin{verbatim}
我们用于 DOOM 的二叉空间划分器(binary space partitioner)源码
现在可在 ftp.uwp.edu: /incoming/id/doombsp.zip 获取。

需要注意的是,这份源码里包含一些 Objective-C 构造,
所以要移植到 DOS 还得花点工夫。唯一麻烦的部分是
替换集合对象,大多数代码都是纯 C。

这段代码是写出来并不断扩展的,而不是逐步演化出来的,
所以它可能不是世界上最干净的实现。请一定、一定、一定
不要向我请求支持。我已经被太多事情占满了时间。

我们的地图编辑器不能直接处理 WAD 文件。它会保存为
ASCII 文本表示,然后启动 doombsp 把它处理成 WAD 文件。
我附上了 E1M1 的输入与输出,你可以用来验证移植工作。

拆成两个程序让我们在 NEXTSTEP 上很好地分工,
但在 DOS 上做编辑器的人可能更希望把
bsp 代码直接集成到编辑器里。

如果你要制作供他人使用的新 DOOM 地图,
我们希望你生成的 WAD 文件在文件开头使用 PWAD 标识,
而不是普通的 IWAD。这样 DOOM 会告诉用户他们在玩修改版,
并且我们不会提供技术支持。

如果你要制作供他人分发的地图编辑器,
请联系 Jay Wilbur(jayw@idsoftware.com),
以获取使用我们商标等的许可协议。
这不是钱的问题,只是一些法律上的手续。

顺便说一下,这里面确实有一个 bug,会导致
最多 4 像素宽的一列被乱序绘制,
从而让更远处的地板与天花板平面
“向前流动”得比它应该的更远。
有时你可以在 E1M1 上看到:
在锯齿房间入口处朝着台阶上的小恶魔方向看,
右侧通道旁会有一条几像素宽的黏液“往下流”。
你得稍微用鼠标找一找那个位置。
如果有人查到了这个问题,请告诉我……

玩得开心!

John Carmack
Id Software

\end{verbatim}
\par \hrule
