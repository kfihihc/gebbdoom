\section{发行}
为发行 \doom,id Software 再次采用共享软件模式,即免费提供游戏的一小部分。第一章 “Knee-Deep in the Dead” 包含九张地图,可免费下载安装,并鼓励玩家尽可能复制并分享。为此,id Software 设法将游戏引擎和第一章削减并压缩到仅需两张 3$\nicefrac{1}{2}$ 英寸软盘。\\
\par 
满意的玩家可以向 id Software 付费,并通过邮寄收到剩余两章:“The Shores of Hell” 以及其续篇 “Inferno”。\\
\par
\cfullimage{endoom.png}{共享版末尾的“广告界面”,引导玩家如何获取更多章节。}
\par
但这一次,id 想把事情提升到另一个层次。不仅要通过玩家传播游戏,还想进入实体零售店。但他们不想承担实体发行所带来的装箱、库存管理等繁琐物流。\newpage





 事实证明,在 Jay Wilbur 的一个点子帮助下,这个看似不可能的任务还是有了解法。\\
\par
\fq{我们对零售商说:“我们不在乎你们是否从这个共享版赚到钱。” “卖吧,尽量大批量地卖。” 零售商简直不敢相信,从来没人告诉他们不用交版税。
但 Jay 很坚持。“免费拿 DOOM,利润你们留着。” 我的目标是发行。DOOM 会成为升级版的 Wolfenstein,我希望它遍地开花。
我要你们堆满 DOOM。事实上,我还希望你们做广告,因为你们会赚钱。把原本可能交给我的版税拿去宣传你们在卖 DOOM。}{Jay}\\
\par
John Romero 也分享了同样的记忆,并进一步说明他见证的创意。\\
\par
\fq{挑战是:“怎么让 Doom 进店?怎么把免费的东西放上货架?”\\
\par 
想法是:在 doom 的标题画面上写上“建议零售价 9 美元”,然后告诉那些已经进驻商店的公司:“你们把 DOOM 装进盒子放上货架就行,钱全归你们。我们什么都不要,只要你们装盒子卖。”\\
\par 
疯狂吧,但它奏效了。到处都是。如果你在 1994 年走进 CompUSA,会看到十几种 doom 的盒子,以为是不同游戏,其实全是同一个共享版。分销商只被允许卖共享版,所以他们拼命做最漂亮的盒子来击败竞争对手。}{John Romero}\\
\par

这种公式的成功超出他们最狂野的预期。共享版无处不在,甚至出现在最意想不到的包装中。\\
\par 尽管制造有困难,著名的《\doom{} Strategy guides》把两张软盘装在书后封底的信封里。杂志也抓住了机会,即便不得不用塑封包裹来容纳软盘。




%Ideally, to make distribution as easy as possible, the game would have fitted on one 3\nicefrac{1}{2}-inch floppy disk. Even though 650 KiB floppy reader were fading out in favor of 1,440 KiB floppies, because of the volume of assets, DooM shareware still used two disks.\\
\cfullimage{floppies.png}{\doom{} 共享版软盘}
\par
\vspace{-10pt}
图 \ref{floppies.png} 显示两张 3$\nicefrac{1}{2}$" 软盘,装着 \doom{} 共享版,并随《Survivor's Strategies and Secrets》一书附带。出版社没有支付版税。\\
\par
二进制打包方式也不同于上一款作品。Wolfenstein 3D 使用 \cw{WOLF3D.EXE} 引擎和大量 \cw{.WL6} 文件,而 \doom{} 只需要两个关键文件。安装后,除了若干 \cw{TXT} 文件与网络驱动外,游戏体验完全由引擎 \cw{DOOM.EXE} 与包含全部资产的 \cw{DOOM.WAD} 构成。\\
\vspace{3mm}

\rawscaleddrawing{0.9}{graph_wad2}
\vspace{2mm}
注册版占用 11,869,745 字节,其中 \cw{DOOM.EXE} 为 709,905 字节,\cw{DOOM.WAD} 为 11,159,840 字节。
\pagebreak

\trivia{游戏也在 Internet 上发布。1993 年 12 月 10 日,他们试图在 \cw{ftp://ftp.wisc.edu/} 上播种,但无法连接,因为玩家为了第一时间拿到游戏而一直在线。}\\
\par
付费注册版与未注册免费版的 \cw{DOOM.EXE} 并无差别。引擎会扫描当前目录,识别 \cw{WAD} 档案的文件名,并据此分支。\\
\par
\rawscaleddrawing{0.9}{graph_wad}
% \subfile{graph_wad2} 




\subsection{WAD 档案:Where's All the Data?}
\label{wad_explained}
\cw{WAD} 格式的目标部分是替代操作系统的文件系统,但更重要的是拥抱模组社区。在 \cw{WAD} 中,每个资产都存储为一个 “lump”。\cw{WAD} 由三部分构成:文件头、lump 内容,以及末尾的目录。\\

\par
\ccode{wad_structure.c}
\par
\trivia{\cw{WAD} 这个扩展名由 Tom Hall 在一次奇妙的对话中命名。John Carmack 正在为档案格式找名字,问道 “How do you call a file Where's All the Data?” Tom 立即回答:“A WAD!” }\\
\par
\drawing{wad_arch}{一个包含两个 lump 的 wad 文件。}
\par
这个格式由两种工具操控。\cw{lumpy} 将一个数据块封装为 lump 再写入 \cw{WAD}。\cw{wadlink} 将多个 \cw{WAD} 合并/追加为一个 \cw{WAD}。该结构便于增删 lumps,因为新增一个 lump 只需移动末尾的小目录并更新头部偏移。\\
\par
\cw{DOOM.EXE} 有一个命令行参数,允许模组作者加载自定义 \cw{WAD} 来覆盖 \cw{DOOM.WAD} 的 lump 条目。这一机制几乎允许自定义所有内容。比如自定义 \cw{WAD} 中包含 \cw{E1M1} lump,可通过简单的 \cw{doom -file mylevel.wad} 命令使用(详见第 \pageref{wad_detailled} 页)。
