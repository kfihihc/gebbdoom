1997 年 12 月,备受期待的 \doom{} 引擎源代码终于向公众发布。
John Carmack 为此写了几句话。\\
\par
\vspace{10pt}
\hrule \par
\begin{verbatim}
终于,终于发布了。DOOM 的源代码开放给你们非营利使用。
你仍然需要真实的 DOOM 数据才能与这份代码配合工作。
如果你没有真正的 DOOM 拷贝,
在软件商店里应该还能找到。

非常感谢 Bernd Kreimeier 花时间清理项目并确保它真的能跑起来。
项目放几年不管就会腐朽,
重新收拾起来需要有人投入精力。

坏消息:这份代码只在 Linux 上能编译运行。
我们不能发布 DOS 代码,因为里面用到了受版权保护的声音库
(哇,那真是个错误——现在我都自己写声音代码了),
而且我也不清楚微软移植到 Windows 的版本后来怎样了。

不过这份代码的可移植性还是不错的,
应该很容易把它移植到几乎任何平台上。

这些代码是很久很久以前写的,
现在回头看有不少地方显得相当愚蠢
(比如用极坐标做裁剪),
但总体来说它仍然是个很有用的基础,
可以用来实验和构建新的东西。

基本的渲染概念——按带状区域使用固定光照的
水平/垂直恒定 Z 线——是完全正确的,
但如果重新审视,原始实现仍有极大的改进空间。
渲染流程从墙到地板再到精灵,
可以合并成一次从前到后的 BSP 树遍历来收集信息,
然后在回溯时绘制一个子扇区内的所有内容。
这需要把地板和天花板视作多边形,
而不是墙之间的空隙,
还需要把精灵广告牌裁剪成子扇区碎片,
但这是正确的做法。

我回头看最糟糕的地方之一是运动与视线检测。
它的代码很乱,而且有一些失败情况,
其实有一个简单得多(也更快)的方案就在我眼前。
我用 BSP 树来渲染,但当时没有意识到它也可以用于环境测试。
把视线测试替换成 bsp 线段裁剪会很容易。
用体积扫描来处理移动会更困难一些,
会触及 Quake/Quake2 中许多边缘倒角的挑战。

一些项目想法:

移植到你喜欢的操作系统。

添加一些渲染特性——透明、抬头/低头、斜坡等。

添加一些游戏特性——武器、跳跃、下蹲、飞行等。

创建一个基于包服务器的互联网游戏。

创建一个客户端/服务器架构的互联网游戏。

做一个 3D 加速版本。在现代硬件上(快速的 Pentium + 3DFX),
你可能甚至不需要太聪明——直接把整个关卡画出来,
也能有不错的速度。稍微努力一下,
应该很容易锁定在 60 fps(不过 DOOM 的 35 Hz 时基
还是有一些问题)。最大的麻烦可能是
纹理尺寸不是 2 的幂,以及墙由多层纹理组成。

我并不太确定会有多少人来折腾这些代码,
但如果有人做出重要项目,
我希望能看到一定程度的社区协作。
我知道早期项目大多会是各自为战的粗糙黑客作品,
但如果明年能看到一个协调发布的、改进且向后兼容的
DOOM 多平台版本,那就很酷了。

玩得开心。
\end{verbatim}
\par \hrule
