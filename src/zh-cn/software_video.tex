\section{视频管理器}
在继续我们的代码之旅、查看 \cw{D\_Display} 中的渲染之前,让我们先花几页来研究每帧存储与处理的图形堆栈。视频系统位于核心中,它维护并提供两类数据结构:一组五个帧缓冲和一个 dirtybox(用作“脏矩形”)。所有写操作都会更新帧缓冲与 dirtybox。\\
\par
\cccode{kernel_renderer.c}{五个帧缓冲。}
\par
\par
\cccode{kernel_renderer_dbox.c}{dirtybox。}
\par
\vspace{-10pt}
视频系统实现必须向核心提供四个函数。\cw{I\_InitGraphics} 用于在需要时初始化自身。\cw{I\_UpdateNoBlit} 在帧缓冲某部分被修改时调用。\cw{I\_FinishUpdate} 在帧缓冲完全合成并应显示到屏幕时调用。\cw{I\_WaitVBL} 会阻塞直到下一次垂直同步。\\
 \begin{figure}[H]
\centering
\begin{tabularx}{\textwidth}{ L{0.27} | L{0.73} }
  \specialrule{1pt}{0pt}{0pt}
  \textbf{方法} & \textbf{DOS 实现} \\
  \specialrule{1pt}{0pt}{0pt}
\cw{I\_InitGraphics} & 将 VGA 设为 Mode-Y(320x200 256 色,拉伸到 4:3)。\\
\cw{I\_UpdateNoBlit} & 把更新区域发送到 VGA 硬件。\\
\cw{I\_FinishUpdate} & 翻转缓冲(更新 CRTC 起始扫描地址)。\\
\cw{I\_WaitVBL} & 等待 V-Sync(用于在更新调色板前等待)。\\
   \specialrule{1pt}{0pt}{0pt}
\end{tabularx}
%\caption{视频系统接口}
\end{figure}
\par
把帧缓冲放在核心而不是视频系统里,是一个大胆的权衡:性能会受影响,因为数据在到达屏幕前要复制两次。但它极大提升了可移植性,因为帧缓冲被抽象化了。它还为从核心帧缓冲回读打开了大门。这种能力带来了新的效果,例如“spectre”恶魔使用的“predator”透明效果。\\
\par
 % The performance hit was not too back on VLB systems. If you refer back to the architecture on page \pageref{vlbarchitecture}, you will see that it allowed transfer of one framebuffer to the video system while the next frame was being generated in the cachelines. This parallel pipeline allowed the framerate to be 90\% of what it would have been with direct access to the VGA.\\
 \trivia{因为 \doom{} 每帧都会渲染全屏,所以不像 Wolfenstein 3D 那样需要“清空”帧缓冲。如果我们在渲染一帧时打断引擎去窥视帧缓冲 \#0,它看起来会像上一帧与尚未完成的新帧的组合。}\\
\par
\vspace{5pt}
\drawing{renderer_full_pipeline}{}
\pagebreak
%\par
%\trivia{Notice how the VGA framebuffer are not exactyl 320x200=0x3E80 in size but 0x4000 instead. This is done because updating the CRTC start address can be don atomically.}
\par
一帧的生命周期总是相同的,无论游戏正在渲染什么:
\begin{enumerate}
\item 所有写操作都在帧缓冲 \#0 中完成。每次引擎更新这个缓冲时,也会更新 \cw{dirtybox} 来标记被触及的区域。
\item 在一次较大的写操作后,引擎调用 \cw{I\_UpdateNoBlit},触发视频系统读取帧缓冲 \#0,并借助 \cw{dirtybox} 尽可能减少数据传输。在 DOS 上,VGA 子系统把内容复制到它的三个 VRAM 帧缓冲之一。每个 VGA 缓冲都有自己的 dirtybox(\cw{updatebox})来加速 blit,但这依然是昂贵操作。
\end{enumerate}
函数 \cw{I\_UpdateBox} 中负责把数据从 RAM 传到 VRAM 的内容可能会让人惊讶。毕竟在讨论过 32 位 CPU 和 32 位 VL-Bus 之后,却看到传输循环一次只做 16 位(\cw{short *dest;})。事实证明这是有意为之。\\\\
\par
\fq{我们的美术资产是按 8x8 块制作的,Tom 称它们为“ebes”。32 位循环会需要更多代码来处理宽度为奇数个 ebe 的情况。它也许能提速,但总线和显卡必须能处理完整的 32 位写入,而我不记得当时这很常见。许多 VL-Bus 卡仍然使用与 ISA 卡相同的基本芯片组,而且往往仍是 16 位,这意味着 32 位写入会花 2 倍的时间。}{John Carmack}\\\\
\par
\ccode{I_UpdateBox.c}
\par
