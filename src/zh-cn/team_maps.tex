

\section{地图}
地图以俯视视角设计。其限制是墙壁与地面垂直,地面与天花板水平,因此地图是以 2D 绘制的。设计师使用五类元素:\cw{VERTEX}\footnote{顶点坐标用有符号短整数表示 [-32768, 32767]。32 单位大约等于 1 米(或对必须使用英制的人来说是 3.28 英尺)。}、\cw{LINE}、\cw{SIDEDEF}、\cw{SECTOR} 和 \cw{THING}。\\
\par
\drawing{doom_map_basics}{\doom{} 地图以及通过 DoomED 制作的数据}

\par
\cw{SECTOR} 是一个被 \cw{LINE} 包围的封闭区域,具有指定的地面高度、地面贴图、天花板高度、天花板贴图以及光照等级。扇区可以是凹形,但线段不能相互交叉。\\
\par
\cw{LINE} 可以是实墙,也可以是连接两个 \cw{SECTOR} 的传送门。区别在于关联的 \cw{SIDEDEF} 数量。墙只有右侧一个 \cw{SIDEDEF},完全不透明。传送门有两个 \cw{SIDEDEF},通常可以部分透视。\\
\par
\cw{SIDEDEF} 描述 \cw{LINE} 的一侧。为同时贴墙体与传送门,它最多可有三张贴图。中间贴图用于墙体覆盖的完整区域。\cw{SIDEDEF} 还可以为连接不同天花板/地面高度的 \cw{SECTOR} 的传送门设置上贴图与下贴图。如果传送门通向更高的地面,就使用下贴图绘制“台阶”。如果 \cw{SECTOR} 连接到更低的天花板,则使用上贴图绘制“下台阶”。为对齐门和按钮,\cw{SIDEDEF} 贴图可有垂直/水平偏移。\\
\par
相比之下 \cw{THING} 更简单,只包含一个 2D 坐标 \cw{X,Y}、一个角度,以及控制类型的标识符。地图至少必须包含一个玩家出生点 \cw{THING}。\\

\cfullimage{doom_map.png}{图 \ref{doom_map_basics} 所示地图的渲染效果。}
\par

图 \ref{doom_map.png} 的场景很丑,但配色不一致希望能帮助区分不同元素。除 \cw{E-B} 以外的所有 \cw{LINE} 都是墙;\cw{E-B} 有两个 \cw{SIDEDEF},因此是传送门。所有墙使用 \cw{BRIK} 中间贴图,传送门上下均使用 \cw{GRAY}。\\
\par
\cw{SECTOR} \#0 使用 \cw{RED} 地面贴图与 \cw{WOOD} 天花板贴图。地面高度为 20,天花板高度为 40。\cw{SECTOR} \#1 使用 \cw{BLUE} 地面贴图与 \cw{GREEN} 天花板贴图。其地面高度为 0,天花板高度为 60。两个扇区的光照等级相同(10)。\\
\par
注意 \cw{E-B} 传送门,它没有中间贴图,而是上贴图与下贴图,用于绘制通向 \#0 扇区的上台阶与下台阶。\\
\par
还要注意 \cw{D-E} 墙体,其中间贴图的垂直偏移设置不正确,导致与 \cw{E-F} 墙衔接处出现垂直撕裂。\cw{B-C} 墙的垂直偏移设置正确,因此没有视觉瑕疵。所有墙体都未使用水平偏移,但图 \ref{doom_map_basics} 用 \cw{XOFF} 标注了其位置。\\ 
\pagebreak



\subsection{地图编辑器(DoomED)}
为了驾驭复杂的地图格式,他们制作了一款新工具替代 TED5\footnote{id Software 此前的地图编辑器。}。这款 Doom 地图编辑器名为 \textbf{DoomED}。这也是 \NeXT 解决方案影响最大之处。高分辨率显示器提供了充足的空间来呈现细节和众多控件。NeXTSTEP 的稳定性确保在编写 DoomED 或制作地图时不会丢失工作成果。
Objective-C 的设计也产生了巨大影响。该语言的消息分发系统优雅地处理了 \cw{nullptr}\footnote{Colin Wheeler 的 “Understanding the Objective-C Runtime”。} 解引用,使得在出现故障时功能失效但不会崩溃。\\
\par   
杀手级特性是 Interface Builder,它不仅提供完整的控件库,还允许快速创建新控件并立即与业务逻辑相连。\\
\par
\fullimage{doomed/DoomEd.png}
\par

2015 年 4 月源代码公开后,程序员得以窥探内部。编辑器代码量约为引擎的一半(doom:32kloc,DoomED:20kloc)。如果没有 \NeXTns 的强大支持,编辑器的开发时间至少要翻倍。



\tcode{cloc_doomed.txt}
\par
DoomED 被设计为“世界地图的 Adobe Illustrator”,设计师只需画线、选扇区、挑贴图。\\
\par
\vspace{10pt}
\trivia{DoomED 的图标像地狱男爵。启动时会播放小恶魔的咆哮声。}\\
\par
\fullimage{doomed/all_widgets.png}

DoomED 输出的数据不能直接给游戏引擎使用。它生成一种名为 \cw{DWD} 的文本格式。文件头为魔数,后面是线(含 sidedef)的列表和 thing 的列表。扇区从线的天花板/地面贴图、高度与光照属性推断得到。\\
\par
\tcode{map.txt}
\par
DWD 的设计并不追求空间效率,而是便于解析,因为它随后会由节点构建器 \cw{doombsp} 进行后处理。\\
\par
\cfullimage{props/tom.png}{Tom Hall 正在愉快地制作后来被称为 E2M7 的地图。显示器上的贴纸写着 “quality”。}
\par

\pagebreak





\section{地图预处理器(Node Builder)}
地图预处理并非 id 的新事物。自 1991 年《德军总部 3D》起,地图就会预处理以加速声音传播。到了 \doom,这项工作在复杂度与处理时间上都被提升到了新高度。\\
\par
核心问题是,在放宽 Wolfenstein 的正交网格约束、失去 DDA 算法\footnote{数字微分分析器广泛用于 VSD(可视表面判定)、碰撞检测与视线计算;在 \doom 中这一切都消失了。} 的情况下,仍要保持同样的渲染速度。解决方案是为每张地图生成多种加速数据结构,各自针对不同问题。\\
\par
 执行该任务的工具叫 \cw{doombsp}。它输入 \cw{.DWD} 地图,输出 \cw{.WAD}。地图不仅以更高效的形式表达(例如顶点只存一次并用索引引用),还会同时生成三种数据结构:二叉空间划分后的节点树以加速渲染;blockmap 加速碰撞检测;reject 表加速 AI 处理。\\
 \par
\trivia{地图预处理耗时显著。用 NextStation TurboColor 运行 \cw{doombsp} 处理 \cw{E1M1} 需 10 秒,E1M2 需 30 秒,E2M7 需要整整一分钟。共享版前九张地图需要 3 分 26 秒。注册版的 27 张地图需要 11 分钟。}\\% This was a problem for designers...yet it was little compared to what Quake's \cw{qbsp} would be three years later.}\\
\par
\tcode{cloc_doombsp.txt}
\par
 \trivia{\cw{DoomED.app}、\cw{doom} 与 \cw{doombsp} 紧密耦合。编辑器里一个按钮就能保存地图、调用节点构建器并启动游戏加载正在制作的地图。}


%
\begin{figure}[H]
\vspace*{3mm}
\centering
\includegraphics[width=\textwidth]{drawings/E1M1_lines.pdf}
\end{figure}
\par
通过 DoomED 生成的地图。传送门为红色,墙体为黑色。\\
\par
\begin{figure}[H]
\vspace*{2mm}
\centering
\includegraphics[width=\textwidth]{drawings/E1M1_fab.pdf}
\end{figure}
BSP 节点树,扇区被分割成凸子空间(sub-sector)。



\begin{figure}[H]
\centering
\includegraphics[width=\textwidth]{drawings/E1M1_blockmap.pdf}
\end{figure}
\par
Blockmap 切分,每个块为 128x128,用于加速碰撞检测。\\
\par
\begin{figure}[H]
\centering
\includegraphics[width=\textwidth]{drawings/E1M1_sides_with_player.pdf}
\end{figure}
\par
REJECT 数据结构用于加速敌人和怪物的视线计算。
\pagebreak


\doom{} 发布不久后的 1994 年 5 月,节点构建器的源代码公开。它最初是 NeXTSTEP 版本,但很快被移植到 DOS 并以 IDBSP 之名发布,让大批模组作者欣喜不已。\\
\par
\cfullimage{doombsp_compiling.png}{在 NeXTSTEP 上编译 doombsp。}
\par
% \vspace{-10pt}
对每个 \cw{.dwd},\cw{doombsp} 输出一组 lumps 并存入 \cw{.wad} 文件(见第 \pageref{wad_explained} 页)。\\

\par
 \begin{figure}[H]
 \setlength{\belowcaptionskip}{-10pt}
\centering  
\begin{tabularx}{\textwidth}{ L{0.15} L{0.75} }
  \toprule
  \textbf{Lump 名称} &  \textbf{说明} \\
  \toprule 
   
   \cw{EXMY} & 地图起始标记,其中 X 为章节、Y 为地图序号。其后所有 lumps 属于该地图“块”。\\
   \cw{MAPXY} & 与 EXMY 相同,但用于 Doom II。\\
   \cw{VERTEXES} & \cw{signed short} 的 X、Y 对数组。地图块中的所有坐标均为此数组索引。\\
   \cw{LINEDEFS} & 引用两个顶点的线段数组。直接对应 DoomED 中的线,同时指向一或两个 \cw{SIDEDEFS}(取决于是墙还是传送门)。 \\
   \cw{SIDEDEFS} & 定义上、下、中贴图,并包含水平/垂直偏移。\\
   \cw{SECTORS} & 由线围成的区域,包含天花板与地面贴图/高度及光照等级。\\
   \cw{THINGS} & 所有怪物、道具与出生点的位置与角度。\\
   \toprule
   \cw{NODES} & BSP,包含 segs、nodes 与 sub-sector 叶子。\\
   \cw{SEGS} & 由于 BSP 切分产生的线段片段(见第 \pageref{Binary Space Partitioning: Theory} 页)。\\
   \cw{SSECTORS} & 由 \cw{SEGS} 构成的凸空间集合。\\
   \toprule
   \cw{REJECT} & 扇区之间的可见性矩阵,用于加速视线计算。\\
   \toprule
   \cw{BLOCKMAP} & 以 128x128 网格分割地图 \cw{LINEDEFS},加速碰撞检测。\\
   \toprule
\end{tabularx}
\caption{“The Unofficial Doom Specs” v1.666 中记录的地图数据 lumps。}
\end{figure}
\pagebreak

通过二分切割地图并不容易。\cw{doombsp} 的启发式算法尝试在生成尽可能少的段数的同时,创建平衡的树并选择轴对齐的切割线。调试标志 \cw{-draw} 可监控流程。BSP 树与二分切割详见第 \pageref{Binary Space Partitioning: Theory} 页。 \\
\par
\cfullimage{doombsp_run.png}{以调试模式运行 doombsp 展示了切割线的选择过程。}
\par
