\section{调色板效果}
\label{label_palettes} \label{doom_palette}
\begin{wrapfigure}[8]{r}{0.25\textwidth}
\centering
\scaledimage{0.25}{palette.png}
\end{wrapfigure}

尽管 VGA 硬件有许多不足,但它有一个非常酷的特性:调色板。只需 768 字节就能让整个屏幕向某个颜色淡入。\doom{} 用淡入来实现三种效果——受伤、拾取物品以及辐射防护服。为实现这些效果,调色板被预先计算并存储在 \cw{PLAYPAL} lump 中。总共有 14 个调色板。调色板 \#0 是默认调色板,在游戏的大部分时间使用。\\
\par
\#1 到 \#8 的八个调色板用于强化受伤效果,让玩家知道自己受伤有多重。根据所受伤害量,显示会切换到七个调色板之一(伤害越高,起始调色板编号越高)。整屏变红通常是非常糟糕的信号。如果不再受伤,调色板会以每 1/2 秒回退一个等级的速度淡回正常(调色板 \#0)。\\
\par
\scaledimage{0.25}{palette_damage1.png}
\scaledimage{0.25}{palette_damage2.png}
\scaledimage{0.25}{palette_damage3.png}
\scaledimage{0.25}{palette_damage4.png}
\scaledimage{0.25}{palette_damage5.png}
\scaledimage{0.25}{palette_damage6.png}
\scaledimage{0.25}{palette_damage7.png}
\scaledimage{0.25}{palette_damage8.png}\\
\par
由于一个 off-by-one 的 bug,调色板 \#1 从未被使用。仔细看看 \cw{ST\_doPaletteStuff} 中选择伤害调色板的代码,它只能生成范围 [2,8] 的值。\\
\par
\ccode{ST_doPaletteStuff0.c}\\
\par
\ccode{ST_doPaletteStuff.c}\\
\par
调色板 \#9 到 \#12 会在拾取物品时短暂使用。\\
\par
\scaledimage{0.25}{palette_item1.png}
\scaledimage{0.25}{palette_item2.png}
\scaledimage{0.25}{palette_item3.png}
\scaledimage{0.25}{palette_item4.png}\\
\par
同样由于 off-by-one 的误算,调色板 \#9 也不会被使用。调色板选择器只能生成 [10,12] 范围的值,而本应需要 [9,12]。\\
\par
% \ccode{ST_doPaletteStuff2.c}\\
\par
最后一个调色板(\#13)不是用于淡入淡出效果,而是当玩家穿着辐射防护服时临时切换使用。\\ 
\par

\scaledimage{0.25}{palette_radiationsuit.png}
