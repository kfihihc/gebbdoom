\section{音频系统}
\label{dmx_section}
和引擎里的其他 I/O 系统一样,音频也被抽象在接口之后。为了满足 \doom 的核心需求,这个系统至少要实现二十个函数,涵盖音效、音乐以及计时器。以下是其中一些。\\
\par
 \begin{figure}[H]
\centering  
\begin{tabularx}{\textwidth}{ L{0.57}  L{1.43}}
  \toprule
  \textbf{方法} &  \textbf{用途}\\

  \toprule 
  I\_StartupSound & 初始化音频系统,检测音频硬件\\
  I\_SetChannels & 设置通道数与采样率\\
  \toprule 
   
I\_RegisterSong & 上传音乐 lump 并返回 ID。\\
I\_SetMusicVolume & 顾名思义。\\
I\_PlaySong & 播放音乐\\
I\_PauseSong & ……嗯,暂停音乐\\
I\_ResumeSong & 效果不明的神秘函数。\\
I\_StopSong & 也许这个表格根本不是个好主意。\\
I\_UnRegisterSong & 使用 \cw{I\_RegisterSong} 获取的 ID 释放音乐。\\




  \toprule 
I\_GetSfxLumpNum & 从 WAD 上传音频样本并返回 ID。\\
I\_SetSfxVolume & 如果你读到这里,你是真人英雄。\\
I\_StartSound & 开始播放 SFX 样本。\\
I\_StopSound & 停止播放 SFX 样本并释放。\\
I\_SoundIsPlaying & 测试 SFX 是否正在播放。\\
I\_UpdateSoundParams & 设置音高、左右位置与音量。\\

  \toprule 
  
I\_StartupTimer & 在 DOS 上,触发音频系统挂接 Intel 8259 PIC。其他平台为无操作。\\
I\_ShutdownTimer & 在 DOS 上移除挂接。其他平台无影响。\\

   \toprule
\end{tabularx}
\caption{\doom{} 的音频系统接口}
\end{figure}



在 NeXTSTEP 上,这个系统从未成为问题,因为只需要实现计时器功能。但在 PC 上,引擎必须有音效与音乐。一个困难是声卡市场的碎片化,相较于 Wolfenstein 3D 已指数级增长。前作只支持四种声卡就需要巨大努力,而两年后可用声卡超过十五种,且各有 bug、怪癖与技术差异。\\
\par
更糟的是,Jason Blochowiak 的离开让 id Software 既缺乏专业知识也缺乏热情。他们用钱解决问题,授权了 DMX 库。由 Paul J. Radek 编写的库以其授权价格提供一体化音频解决方案:支持所有主要声卡、提供便捷检测方式、支持多种声音与音乐格式,并可轻松集成到任何游戏引擎。DMX 非常合适,无疑为 id 节省了数月开发时间。





DMX 提供的抽象层工作量巨大。检测硬件的函数前的注释清楚说明了这项工作有多沉重。\\
\par
\ccode{whypcssucks.c}\\
\par
有了 DMX 支撑,系统支持了十种音频芯片组。\\
\par
\ccode{cardenum_t.c}\\
\par
在一个既不支持线程也不支持进程的操作系统上,似乎没有办法同时生成视频与音频输出。细心的读者会注意到硬件章节提到两个芯片组:i8259(可编程中断控制器:PIC)和 i8254(可编程间隔定时器:PIT)。它们可以协同配置,以中断引擎执行并调用 DMX 例程\footnote{PIC 与 PIT 的交互在《Game Engine Black Book: Wolfenstein 3D》中有详细描述。}。\\
\par
初始化时,DMX 把自己安装为中断处理器,由 PIC 与 PIT 以 140Hz 调用。唤醒后,DMX 的中断处理器负责将引擎提供的音乐与音效数据喂给音频设备。\\
\par
由于绑定在计时器上,DMX 也通过名为 \cw{ticcount} 的变量承担了引擎的“心跳”。\doom{} 中的一切都用该变量来节奏控制。
\par


\scaleddrawing{0.95}{sound_manager_architecture}{DMX 架构}

这种架构下有两个系统“伪并发”执行,但音频约束比视频更严格。中断触发时,DMX 只有几毫秒来填充声卡缓冲并回到休眠。如果 DMX 太慢,会延迟视频渲染或屏蔽其他中断。\\
\par
 这解释了为何音频资源分配在 zone 分配器中享有特殊待遇。音频系统无法承受内存未命中(也无法从中恢复,因为它无法访问 WAD 或内存分配器)。音频数据必须在需要时立刻可用。

\subsection{音频数据:格式与 Lumps}
WAD 档案包含数百个五种类型的 lump 供 DMX 使用。\cw{DP*} lump 用于 PC speaker 音效,\cw{DS*} lump 用于 PCM 音效,\cw{D\_*} lump 用于音乐曲目。还有一个 \cw{GENMIDI} lump,以及后面详述的 \cw{DMXGUS} lump。\\
\par
PC speaker 音频面向没有声卡的 PC。PC speaker 只能产生方波,原本用于发出开机诊断噪声。\doom{} 通过每 1/140 秒改变方波频率,让它生成不那么刺耳的声音\footnote{PC Speaker 的细节见《Game Engine Black Book: Wolfenstein 3D》。}。数据率为每 1/140 秒 1 字节,描述要设置的方波频率。\\
\par
\doom{} 使用的数字化音频样本是 8-bit、11025 Hz、单声道 PCM 流。PCM 是一种很直观的音频格式:流中的每个字节代表一个时间点的波形幅度,指示扬声器振膜应处于的位置。声卡把字节转换为驱动振膜位置的电压,以每秒 11025 次的频率执行。\\
\par
音乐数据存储为 \cw{MUS} 格式,这与标准 MIDI 类似但更紧凑。该格式支持 8 个乐器通道和第 9 个鼓通道。MUS 描述每个通道的一系列精确时序事件,指定哪个乐器以什么音高演奏音符。注意 MUS 只描述“做什么”和“何时做”,而不描述“如何做”。每个通道引用某个乐器的音符;乐器本身由一个名为“乐器库(instrument bank)”的数据结构描述。\\
\par
乐器库存储在 \cw{GENMIDI} lump 中,面向基于 OPL2 芯片、通过频率调制合成音乐的 SoundBlaster 兼容声卡。该 lump 描述如何在 MIDI 乐器集中播放每个乐器的音符。\cw{GENMIDI} 中共有 175 项:128 个标准通用 MIDI 乐器与 47 个打击乐效果。每一项描述如何设置一个通道以模拟某个乐器。一个通道由两个单元组成,其中一个作为载波,另一个作为调制器。对每个单元,音乐家可以选择起音、衰减、保持、释放、谐波类型与波形。乐器库是 90 年代音乐人的“秘方”,\doom{} 的通用 MIDI FM 补丁集以效果好而闻名\footnote{来源:Freedoom 文档中的《The dark and forgotten art of GENMIDI》。}。\\
\par
\cw{DMXGUS} lump 与 \cw{GENMIDI} 类似,但用于 Gravis Ultrasound 声卡。它让 GUS 播放 MUS 音符时不用 FM 合成器,而使用 GUS 驱动提供的 PCM 样本。它是一个简单的 lump,把 MIDI 乐器映射到 GUS 乐器,并根据板卡上的 RAM 大小(256KiB、512KiB、768KiB、1024KiB)制定特殊的内存分配规则。\pagebreak



处理如此复杂的情况对 DMX 来说并不容易。Paul Radek 在 \cw{usenet} 上的一篇帖子\footnote{论坛帖:“Gravis Ultrasound - Hardware Mixing Game List”。}透露了他不得不面对的一些疯狂问题。\\





\fq{所有音效现在都在软件中混音,而不是在 GUS 硬件上。为什么?有几个原因。首先,GF1 芯片的最小斜坡时间对于非常尖锐的音效来说太长了。其次,因为加载 MUSIC 补丁会占用所有 GUS 内存,我不得不在音效播放时把 8 个音效通过 DMA 送到卡上。这反而暴露了 GF1 芯片的一个 bug,Gravis 在我的代码开始折腾它之前并没发现。这个 bug 会导致总线冻结,从而冻结任何程序。解决方法是尽量减少 DMA 活动:在软件中混音,只传输 1 个通道到 GUS。但由于 GF1 不支持自动初始化 DMA,而且在卡上播放交错数据的唯一方法是设置两个 voice 指向同一个 patch 并调节频率跳过每隔一个样本,因此你无法从 GF1 获得采样平滑的好处。\\
\par
抱歉,但只能这样 :(\\
}
{Paul Radek, Digital Expressions, Inc.}\\
\par

该库在 \doom{} 开发过程中不断演进。有时 API 变更会引入 bug。v1.666 破坏了 Gravis UltraSound 支持;Audio Spectrum 的支持也被破坏,用户不得不退回到(很差的)SoundBlaster 模拟\footnote{来源:John Romero 在 \cw{alt.games.doom} 上的帖子。}。\\
\par
问题一部分来自 DMX 的 API 设计不佳,一部分来自 id Software 的匆忙调整。多数问题最终被修复,但仍有一个重大 bug 留给了玩家。引擎原本应随机改变声音音高以避免单调,为此使用了 DMX 函数 \cw{SFX\_PlayPatch}。\\
\par
\ccode{dmx_before.c}





在早期版本中它能正常工作,但 DMX API 随后做了不兼容修改。虽然不易察觉,但 \cw{SFX\_PlayPatch} 的参数被调换了。\\
\par
\ccode{dmx_after.c}\\
\par
调用点从未调整,游戏在没有随机音高功能的情况下发布,反而随机在左右声道之间平衡声音。\\
\par
\ccode{I_StartSound.c}\\
\par
事后 John Carmack 后悔使用 DMX,因为这在开源引擎时导致了问题(不清楚是 Paul Radek 不愿开源 DMX,还是 id Software 不愿与他协商)。\\
\par
\fq{
我们在 DOOM 开发中的最大错误是外包 DOS 声音驱动。因为我们要依赖这个黑盒功能,我没有在 NS 下模拟它。大错特错。今后的工作会完全在 NS 下开发,硬件层只剩 DMA 缓冲翻转。我们也可能在 NS 下运行 MIDI 音乐(在 Quake 中将根据游戏情况动态调整)。}{John Carmack}\\
\par
在 \cw{alt.games.doom} 上仍能找到一些存档的 Usenet 帖子,John Romero 的评论暗示 Radek 与 id Software 的关系在游戏临近发布时已出现问题\footnote{来源:John Romero 在 \cw{alt.games.doom} 上的帖子。}。由于语气并不友好,是否要挖出这些帖子留给读者自行决定。我不推荐这样做。
% \par
% \fq{Hey, everybody!  Thanks a lot for downloading DOOM II -- that's really "cool".   
% We love it when we send evaluation copies to magazine editors and some piece of  
% shit uploads the thing.  The DOOM II out there is the master copy we sent to GT  
% Interactive, our distributor.  If your GUS sound doesn't work, it's probably  
% because your IRQ is > 7.  Our sound code dork broke part of the sound support  
% while he was "expanding" it, so IRQs > 7 don't work, Pro Audio Spectrum owners  
% now have to run their SB-emulation driver (no more native support), etc.  The  
% sound guy is a real shithead.  All these fuckups are in v1.666 as well --  
% sorry.  We have NO time to wait and hope that sound-dork fixes these problems,  
% as he's demonstrated in the past year and a half that he's incompetent.}{John Romero (alt.games.doom)}



\section{声音传播}
敌人对声音的反应会极大影响玩家对 A.I. 智能程度的感受。id Software 确保了声音传播的真实性。当某个扇区开火时,会使用洪泛算法传播噪声。\\
\par
\ccode{P_RecursiveSound.c}\\

\begin{wrapfigure}[10]{r}{0.48\textwidth}
\centering
\includegraphics[width=.48\textwidth]{drawings/sound_blockers.pdf}
\end{wrapfigure}
  利用扇区/门户格式,从玩家所在扇区开始,通过门户(双面线)向相邻扇区洪泛。声音在遇到门或穿过两条“声音阻挡线”后停止。E3M9 关卡有一段包含 38 个怪物的区域,AI 成本被三条(红色)阻挡线降低,确保部分怪物保持休眠。\\
\par
注意 \cw{P\_RecursiveSound} 函数提到“唤醒所有怪物”,却从未遍历关卡中的物体列表。这是一个加速技巧,用于避免在每个扇区中昂贵地搜索要唤醒的所有怪物。怪物总是查看当前扇区的 \cw{soundtarget} 来选取目标。只需给 \cw{sec->soundtarget} 赋值,\cw{sec} 中的所有怪物就会自动获得相同目标。\\
\par
\rawdrawing{E1M1_audio_sections}
\par
\subsubsection{埋伏}
地图设计者希望怪物能潜伏起来,等玩家靠近时突然跳出。这通过给怪物设置 MF\_AMBUSH 标记实现。它不会让怪物失聪,而是让它们在看到玩家之前不会主动寻找玩家。


\subsubsection{超级埋伏}
在 E1M9 关卡中,声音传播被巧妙用于“超级埋伏”。设计者希望玩家在走过五芒星中心时,遭遇怪物传送蜂拥而至,仿佛来自地狱维度。\\
\par
在没有脚本语言的情况下,这几乎不可能实现。正如我们在 AI 章节会看到的,怪物只是状态机,只能做三件事:休眠、追逐目标、以及在碰到墙和物体时改变方向。\\
\par
\begin{wrapfigure}[24]{r}{0.52\textwidth}
\centering
\includegraphics[width=.52\textwidth]{drawings/E1M9_sides.pdf}
\end{wrapfigure}

\par

为了实现这一效果,他们在超级埋伏位置旁建了一个无法进入的“怪物池”房间,并塞满怪物(图中红圈)。在隐藏房间的东南角放置了一个传送器,并用墙保护起来。\\
\par
然后他们建了一个很细的管道让声音在两个房间之间流动,使“怪物池”房间中的怪物被唤醒并尝试抵达玩家。由于没有其他路径(管道在天花板高度,怪物太大无法通过),怪物会在原地绕圈,但会倾向于朝玩家位置(东南方向)移动。\\
\par
超级埋伏的最后一环是在“五芒星”中心周围设置四条触发线,降低传送器周围的墙体。他们放上诱人的诱饵(生命与弹药)确保玩家会去那里。\\
\par
当玩家跨过“星形”房间的触发线时,传送器周围的墙体下降,怪物涌向传送器并围攻玩家。 \\
\par
\trivia{代码注释暗示怪物的尖叫会唤醒其他怪物,但实际上并没有。这个特性为何被删不清楚,可能是太多 bug 或代价太高。}






John Romero 自己描述了他们称这些声音通道为“管道(pipes)”。\\
\par
\fq{我们在 Wolfenstein 3D 中用声音区域来提醒敌人你的存在。在 DOOM 中我们做了同样的事,但使用扇区作为音频传播的通道。这对让游戏更恐怖非常关键,因为声音可以泄漏到各处并警告恶魔。你可能会看到许多小扇区“管道”连接不同扇区,只是为了提醒你看不到的怪物扇区,因为我们把它们放在房间高处的角落里。所以我们非常关注声音的洪泛。}{John Romero, Scarydarkfast 第 29 页}\\
\par
