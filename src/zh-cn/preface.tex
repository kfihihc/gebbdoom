这是第二本 \textit{Game Engine Black Book}。它从第一本结束的地方继续,第一本在 1992 年 5 月 Wolfenstein 3D 发布处收尾。本书一直讲到 1993 年 12 月,id Software 在 90 年代的第二次突破——\doom{}。\\ %in the world of PC gaming
\par
像前作一样,本卷尝试非常详细地描述那个时代的硬件与软件。它打开一扇回到过去的窗,窥视 id Software 在花了 11 个月发布下一款作品时所用的工程手段与他们遇到的问题的解决方式。\\% which resulted in what is universally considered one of the best game of all times.\\
\par
在游戏发布 25 年后写一本书似乎有些奇怪。毕竟,谁会对那些在灭绝硬件与过时操作系统上运行的看似陈旧技术感兴趣?第一本 \textit{Black Book} 的成功表明,答案是很多人。无论读者迷恋历史、怀旧、工程,甚至哲学,似乎人人都能找到自己的切入点。\\ 

\par
\doom{} 的影响深远且持久,已成为现代史的一部分。它是无可争议的里程碑,娱乐了数百万人并催生了职业道路。因为源代码被公开,程序员们得以学习其游戏引擎架构。因为它易于修改且工具可用,无数怀揣梦想的游戏制作人第一次尝试设计关卡或绘制资产。直到今天,由于它如此标志性,黑客常用它来展示技能\footnote{2018 年 8 月在破解 BitFi 钱包后,黑客团队通过在设备上运行 \doom{} 来展示成果。}。从 MacBook Touchbar 到 ATM、CT 扫描仪、手表,甚至冰箱,几乎所有电子设备都曾运行过 \doom{}\footnote{“它能运行 \doom{} 吗?”已经成为电脑/游戏圈的常见玩笑,甚至有网站 “itrunsdoom.tumblr.com” 来回答这个问题。}。\\
\par

\doom{} 在商业与口碑上都取得了成功,重塑了 PC 游戏产业\footnote{甚至还“杀死”了 Amiga。来源:Brian Bagnall 的《Commodore: The Amiga Years》。}。在 1994 年它获得了多个奖项,包括 \textit{PC Gamer} 与 \textit{Computer Gaming World} 的 \textit{年度游戏},\textit{PC Magazine} 的 \textit{技术卓越奖},以及 \textit{互动艺术与科学学院} 的 \textit{最佳动作冒险游戏奖}。销量超过 200 万份、试玩版安装量估计达 2000 万份,在巅峰期这个现象每天带来接近 10 万美元的收入。在 “第一人称射击” 这一术语流行之前,人们用 “Doom 克隆” 来指代此类游戏。\\
\par

 \rawdrawing{Doom_clone_vs_first_person_shooter}
 \par

\doom{} 还具有极高的情感价值。它是一款初次接触便留下永恒印象的作品。那些在发布时或不久后玩过的人,至今仍记得第一次看到它运行时的情景。了解曾被视为魔法的事物内部机制,是一种令人振奋的感觉。\\
 \par
 \vspace{-12pt}
怀旧之外,也是本书最重要的相关性所在:\doom{} 的制作是发明者、工程师与建造者围绕共同愿景不断重演的故事。id Software 从所处的位置到想去的地方没有明确的路径——只有没人曾经到达过的确定性。他们日夜工作,睡在地板上,涉水过河,只为让梦想成真。\\
\par
\doom{} 的制作很好地概括了如何将一项宏大任务分解为成千上万件被做对的小事。这是一群梦想者将技能、投入与好运结合起来的故事,最终形成了技术、艺术与设计的惊艳组合。\\
\par



% You may disagree with the values of these "old" things. Some people prefer to sail with the wind, rarely looking back. But even to them this book could turn out to be a useful engineering map someday.\\
% \par
 为了讲述这段精彩的冒险,这本黑皮书必须回应两个看似正交的约束。一方面,它需要自成体系,不依赖补充信息或交叉引用。另一方面,又要避免用已在系列中讲过的内容让忠实读者感到乏味。折中的方式是:让读过 Wolfenstein 3D 的人能从本书获得更多价值,但又不要求他们必须读过前作。\\
 \par
像 VGA 硬件架构、DOS TSR、386 实模式、PC Speaker 声音合成、PIC 与 PIT、DDA 算法等原本值得重访的主题虽被提及,但没有展开细述,因为它们已在 \textit{Game Engine Black Book: Wolfenstein 3D} 中详细讲过。这样的取舍让我们达成目标:一本约 400 页、可以一手拿着喝茶阅读的书。\\
\par
在代码示例上我们做了少量调整。由于纸质版版面受限,代码有时必须略作修改以适配篇幅。还有时,为了循序渐进地引入复杂性并避免读者被淹没,函数中的部分代码会被移除。一些代码示例来自开源前的原始源代码,因此可能与 \cw{github.com} 上的版本不同。请放心,其语义与精神保持不变。\\
\par 
本书是一次练习的成果,灵感来自 Nicolas Boileau 据称说过的一句话:“我们充分理解的东西,就能清晰表达。”这也是我希望别人早已写好的那本书,这样我就可以直接买来读(我其实很懒)。\\
\par
希望你能读得开心!\\
\par
-- Fabien Sanglard\\
Occasional Link to the Past\\
\par
Sunnyvale, CA\\
\monthyeardate\today\\
