\section{公共关系}
在 Facebook、Instagram 与 Twitter 出现之前,在社交媒体和去中心化互联网之前,工作室几乎没有办法直接与潜在客户联系。\\
\par 大多数时候,报纸和杂志会转述信息,但这些信息更新不频繁、速度慢、准确性差,而且往往让读者疑问更多、答案更少。


id Software 很快注意到他们的 NeXT 硬件所带的 Unix 系统包含一个名为 \cw{finger} 的工具。\cw{finger} 完全基于文本,允许远程探索 UNIX 系统。一个 \cw{fingerd} 守护进程监听 TCP 79 端口,随时响应请求。对某个域名执行 \cw{finger} 会返回该机器上注册的所有账户目录。\\
\par
\tcode{finger_idsoftware}
\par
\trivia{当它被发明时,"finger" 这个词在 70 年代带有“打小报告”的含义。这种想象中的“指责性手指”很好地提示了该 UNIX 命令的语义。}\\
\par

id Software 的每位员工都可以在其 \NeXT 工作站的主目录中创建一个 \cw{.plan} 文本文件。任何拥有互联网连接的人都可以查看每个 \cw{.plan} 的内容,只需要在域名前加上用户名即可。这形成了从开发者到消费者的单向直接通道,也是一种独特的信息传递方式,类似于今天的博客。\\
\par
当这个系统刚开始运行时,很少有人知道它,更少的人有条件去 finger id。它没有更新能力,也没有通知机制。有些人,比如 John Carmack,会每天更新 \cw{.plan}。它最初是 bug 修复的项目列表,后来演变为博客式内容,例如 John Carmack 著名的 “OpenGL vs Direct3D”\footnote{见第 \pageref{openglvsdirectd} 页}。现存最早的 plan 记录写于 1996 年 2 月 18 日、Quake 开发期间。

\tcode{finger_johnc.txt}

