\begin{wrapfigure}[4]{r}{0.4\textwidth}{
\centering \scaledimage{0.4}{sega_x32_logo.png}}
\end{wrapfigure}
1994 年 1 月,Sega 处境微妙。Genesis 这台 16 位现金奶牛在日本节节败退。到 1993 年销量已排第三,落后于 Nintendo 的 Super Famicom 与 NEC 的 PC Engine。更糟的是,Sega 还要面对 1993 年进入主机市场的两位新对手:Atari Jaguar 与 Panasonic 3DO。Sega 日本总部(SOJ)认为公司应把所有资源投入 32 位 Saturn 项目。\\
\par
虽然 SOJ 在 Saturn 上推进迅速,但它被担心还需要一段时间才能完成。在美国,Genesis 仍卖得很好(截至 1993 年末已达 3200 万台),Sega of America(SOA)渴望抓住财务机会,推出 Genesis “增强器”。\\
\par
1994 年拉斯维加斯 CES 上,SOJ CEO Hayao Nakayama 召集 SOA 高管 Joe Miller(研发负责人)、Marty Franz(SOA 技术总监)与 Scot Bayless(高级制作人)进行电话会议\footnote{来源:Retrogamer \#77。本节所有引述亦来自 Retrogamer 采访。}。\\
\par 
他们被批准启动 “Mars” 项目,目标是在九个月内发布 Genesis Booster。令人难以置信的是,他们按期完成。Sega 32X 于 1994 年 11 月上市。\\
\par
\fullimage{consoles/32X.png}%{Sega Genesis (a.ka. MegaDrive) with a 32X booster on top of it.}
\par
%The 32X is inserted into the Genesis like a standard game cartridge.
%\pagebreak
32X 以卡带方式插入,32X 游戏再插在顶部。游戏可访问 Genesis 的全部硬件,包括 7.6 MHz Motorola 68000 与 3.58 MHz Zilog Z80。\\

\fq{电话会议结束后,Marty Franz 抓起酒店便签,画了两颗 Hitachi SH2 处理器,各带一个帧缓冲。这基本上就是 32X 的起点。}{Scot Bayless}\\
\par

\drawing{x32_arch}{那张便签可能长这样。}

\par
\fq{图形子系统的设计极其简洁,堪称当时程序员的梦想。它以两颗中央处理器为核心,喂给各自独立的帧缓冲,每像素深度是当时任何东西的两倍。这是一个非常棒的平台,可以以工作站市场之外没人尝试的方式进行 3D。}{Scot Bayless}\\
\par

除了双 SH-2,32X 还配备来自 QSound 的强大音频芯片,支持脉冲宽度调制,并增加了额外声道,甚至具备多维音效能力,可让普通立体声信号近似现实生活中的 3D 音效。还配有名为 “VDP” 的图形芯片,负责双缓冲避免撕裂,并可快速清除帧缓冲。
\par

\pngdrawing{x32_fullarch}{开发者文档总结的 32X 系统。}

设计上与 Saturn(也使用双 SuperH CPU)相似,但理念不同。\\
\par
\fq{Saturn 本质上是 2D 系统,通过移动精灵四个角来模拟 3D 投影。它的优势在于硬件渲染,但这种渲染方式也带来诸多问题,像素覆盖率很高,许多硬件优势被内存访问停顿抵消。32X 则完全用软件渲染,但提供两颗快速 RISC 芯片和大帧缓冲,给程序员完整控制。}{Scot Bayless} \\
\par
工程师全力投入,开发者关系团队也努力准备首发游戏阵容,32X 在 1994 年末卖出 665,000 台。这一良好开端随后却走向悲剧。

一个故事,最适合由 RetroGamer 记者 Damien McFerran 来总结。\\
\par
\fq{怎样把半个十年的口碑与商业成功冲进马桶?\\
\par
很简单:你发布一个像 Sega 32X 这样的设备。}{Retrogamer \#77 记者}\\
\par

真正拖垮 32X 的是 Saturn。1994 年里,SOJ 继续推进 32 位系统,进展足以让 Sega 决定提前在 1994 年 11 月于日本发布——这正是 32X 在美国上市的月份。\\

\par

\fq{毫不意外,消息很快传到西方,美国与欧洲消费者立刻问出显而易见的问题:“既然 Saturn 几个月后就要到,为什么还要买 32X?”遗憾的是,Sega 给出的最好答案是 32X 是“过渡设备”——作为 Mega Drive 通往 Saturn 的桥梁。\\
\par
这让我们在消费者面前显得既贪婪又愚蠢,这是我一年前无法想象的事情。我们曾是酷孩子。}{Scot Bayless}\\
 \par 这一糟糕时机使 32X 几乎出生即死。不仅 Saturn 近在咫尺,PlayStation 也在 32X 发布一个月后的 1994 年 12 月 3 日上市。到 1995 年底,32X 库存以每台 \$19.95 的价格清仓。\\
\par
回望这段历史,会有一种苦涩感:在此之前,Sega 曾是 Nintendo 的强劲对手,凭借五年经营建立起“酷”形象\footnote{1993 年,Sega 是 MTV 最大广告主(来源:“RetroGamer \#77”)。}。\\
\par
从那时起,公司似乎接连做错。Sega 最后一台主机 Dreamcast 于 1998 年发布,广受欢迎,但销量不足以挽救硬件业务。Sega 最终退出市场,转而专注游戏开发。





回顾他在 32X 赶工发布过程中的经历,Scot Bayless 分享了许多洞见。\\
\par
\fq{32X 的游戏队列像被塞进盒子里一样,必须尽快推出,这意味着在任何可想象的地方都得大幅删减。从一开始,这些游戏的设计就因时间紧迫而极度保守。等它们发售时更保守:几乎没有展示硬件能力。}{Scot Bayless}\\
\par
更深层的问题源自 Sega, Inc. 的文化,并解释了后续失误。\\
\par
\fq{
32X 是两个方面的绝佳案例研究:\\
\par
第一是信息传达:营销的首要任务是建立价值主张。即便硬件赶工、软件也匆忙,如果 Sega 能说服人们 32X 值得拥有,它或许还有机会成功。但我们没做到;我们从未以可信的方式解释清楚 32X 的独特价值。结果正如你所料:Sony 抢走了我们的午餐。\\
\par
第二是诚实——不是法律意义上的,也不是公众意义上的,而是内部的诚实。我 1998 年加入微软时,第一天参加高管入职培训,负责接待的 VP 说:‘我们唯一要求你们的是说出真实想法。’这种态度让微软保持了 20 多年的活力与成功。Sega 相反,缺少这种“残酷的诚实”。没人想伤害他人的感情。即便大家都知道 32X 与 Saturn 远落后于时代,也没人愿意站出来说。硬件之外,同一时期 Sega 还发布了一些最奇怪的游戏——漏洞百出、完全无法打动人。所有人却还在笑着说:‘天啊,我们多棒!’当时我说不清楚,但内心知道哪里出了问题,我们正在迷失方向。
}{Scot Bayless}





\subsection{\doom{} 在 32X 上}
如果将 \doom{} 移植到 Jaguar 是一次壮举,那么在更弱的系统上再来一次则几乎是奇迹。John Carmack 再次全力投入。\\
\par
\fq{我花了数周和 id Software 的 John Carmack 一起工作,他甚至住在位于 Redwood City 的 Sega of America 大楼里,试图把 Doom 移植过去。那家伙拼命工作,但仍不得不删掉三分之一的关卡以赶进度。\\ 
\par
现在想想,令我惊讶的是,在这种情况下没人愿意说“等等,我们在干什么?为什么不停止?”Sega 本应在 1994 年春天就取消 32X,但我们没有。我们冲上山顶,却发现爬错了山。\\
\par
回看这段经历,我会说那真的是 Sega 作为硬件公司的声誉走向终结的起点。}{Scot Bayless}\\
\par
为了让游戏塞进 32X 仅有的 512KiB RAM,比 Jaguar 版还要削减更多内容。又一个敌人 Spectre 被移除。怪物的额外姿态被删,除面对玩家的姿势外全删除。由于无法彼此对视,怪物互斗也被取消。没有存档;玩家改为手动选择起始关卡。卡带只能容纳 17 张大量修改的地图。因为没有一张地图包含 BFG9000,这把武器也不可用(但可用作弊码获得)。\\
\par
性能也是大问题。即便有双 SuperH,机器也无法以原始分辨率渲染。\\
\par
\fq{我很喜欢 32X——它基本上是两颗不错的 32 位处理器(SH2)加一个帧缓冲,所以编程方式像 PC,而且在 PC 普及之前就拥有 SMP。它仍比 386 还弱,所以分辨率很低。}{John Carmack}\\
\par
\cfullimage{consoles/x32_screenshot.png}{E1M1 传奇入口大厅}

在图 \ref{consoles/x32_screenshot.png} 中,可以看到像 Jaguar 一样,E1M1 的蓝色地板纹理被替换为棕色以节省 RAM。同样,为了减少可见平面的数量,楼梯台阶也减少了。\\
\par
游戏以 320x224 分辨率运行,但 CPU 压力太大,可视窗口被缩减到 128x144(列翻倍达到 256x144),剩余 100 垂直像素用于状态栏与棕色边框/背景。如此折中仍能达到 15-20 FPS\footnote{来源:Digital Foundry YouTube 频道,“DF Retro: Doom - Every Console Port Tested and Analysed!”。},体验还算不错。\\
\par
\trivia{Sega 用太阳系行星命名项目。除 Saturn 与 Mars 外,还有两个已知项目:Neptune 是计划在 1995 年秋发布的 Genesis 与 32X 合一主机,因担心稀释 Saturn 的营销且价格过近而取消。Jupiter 被传闻为无 CD 驱动的 Saturn。}



\fullimage{32xe1m1.png}\\
\par
地图复杂度被大幅降低。上图中,E1M1 主房间被剥离了许多纹理(对比 PC 版第 \pageref{complex_scene_plain_light.png} 页)。下图中,E1M3 的“坑”被填平。\\
\par
\fullimage{32xe1m2.png}
