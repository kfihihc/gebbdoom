1992 年 5 月,id Software 已是 PC 游戏行业冉冉升起的新星。Wolfenstein 3D 确立了第一人称射击类型,《命运之矛》续作的销量也在飙升\footnote{到 1993 年底,Wolfenstein 3D 与 Spear of Destiny 的合计销量已超过 20 万份。到 1994 年底,这一数字增长到 30 万份。}。耗费多年打造的游戏引擎与相关工具远超竞争对手。他们拥有高效的游戏生产流水线,并具备以精美关卡与资产充分利用它的才能。几乎没人能接近挑战他们……但还能持续多久?他们本可以继续榨取技术红利,但硬件的演进会把他们 \textit{“逼入绝境”}。\\
\par

\par
\fq{由于摩尔定律的特性,极其聪明的图形程序员在某一时点能做的任何事情,几年后都可以被一个水平尚可的程序员复制。}{John Carmack}. \\
\par
竞争对手开始带着自己的游戏而来。自以名为 “Adaptive Tile Refresh” 的技术突破为起点,id Software 的核心价值一直是创新。他们已经推出了 Wolfenstein 3D 的续作,是时候继续前进了。正确的做法(也是最冒险的\footnote{《Things You Should Never Do》(Netscape 6 开发)by Joel Spolsky.})是扔掉他们辛苦构建的一切,从白纸开始下一款游戏。资产、关卡、工具与游戏引擎——一切都要全新与创新。\\
\par
在开始之前,id Software 必须决定要瞄准怎样的硬件,以及使用怎样的工具。对消费者市场的概要评估显示,自他们上一次爆款以来,PC 已显著进化:
\begin{itemize}
\item 英特尔最新的 486 CPU 于 1989 年发布,终于变得可以负担。它提供了上一代两倍的处理能力,越来越多的用户不再选择“老旧”、慢一倍的 Intel 386。
\item Microsoft Windows 3.1 及其饥渴的图形界面促使硬件厂商提供更强的图形适配器。渲染仍需软件完成,但芯片组更快、容量更大。
\item 受总线瓶颈困扰,厂商联合推出新标准。PC 常配备一条比旧有 ISA 快十倍的总线,称为 VESA 本地总线(VLB/VL-Bus)。
\item 内存价格显著下降。曾经的标准 2 MiB 预计将变成 4 MiB。
\item 音频生态更加碎片化,市场上有许多 SoundBlaster “兼容”声卡克隆,以及 Gravis Ultrasound 的波表合成等新技术。\\
\end{itemize}
 \par 
不仅硬件进化,软件也不同了。Watcom 等更好的编译器可以生成更快的代码。对耗时的手工汇编的需求减少,它正逐渐成为过去\footnote{英特尔后来凭借其超标量处理器 Pentium 让这一趋势回潮,使 Quake 的开发高度依赖汇编,但这是另一段故事。}。DOS 扩展器让机器摆脱了 16 位编程及其著名的 1 MiB 地址空间限制。\\
 \par
在开发硬件方面,新选择出现了。强大的工作站如今对专业人士而言既可用也可负担。尤其是一家公司——史蒂夫·乔布斯离开苹果后创立的公司——将强劲硬件与高效开发工具结合起来。\NeXT 生产了运行基于 UNIX 的 NeXTSTEP 操作系统的令人印象深刻的机器。\\% NextStep were overlooked yet solid productivity boosters.\\
 \par
在这股新事物的旋风中,很容易走错方向。然而 id Software 似乎做出了所有正确选择。他们是如何从零开始,在短短 11 个月内做出一款史上最佳游戏之一的?这是本书尝试回答的问题。\\
 \par
为此,前两章将深入审视当时的硬件——不仅包括 \doom{} 运行的 IBM PC,也包括 id Software 选择作为生产流水线基础的 \NeXT 机器。第三章聚焦于团队以及他们编写的工具,用以连接硬件与软件。在这些能力与限制的背景下,最后几章会深入探讨游戏引擎,希望能帮助读者理解为什么事情会被这样设计。\\
\par
现在装上你的霰弹枪,带上几包医疗包,我们开始深入吧!\\
\par
% \vspace{2cm}
\par
{
\setlength{\abovecaptionskip}{15pt}
\cfullimage{demonandshotguns2.png}{"\doom{} 意味着两件事:恶魔与霰弹枪!" -- John Carmack}
}
