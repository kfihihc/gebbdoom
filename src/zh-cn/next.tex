\section{历史}

\begin{wrapfigure}[8]{r}{0.20\textwidth}
\centering
\includegraphics[width=.20\textwidth]{drawings/NeXT_logo.pdf}
\end{wrapfigure}
\par
NeXT 的历史开始于(也可说结束于)Apple。1985 年 5 月,Macintosh 平庸的销量让公司前景黯淡。联合创始人兼当时 Mac 部门总经理 Steve Jobs 希望降价并加强营销以提振 Mac 销量。时任 CEO John Sculley 则希望放弃 Mac,将资源聚焦于当时唯一赚钱的产品 Apple II。\\
\par
董事会投票站在了 Sculley 一边,Steve Jobs 被剥夺了所有职责。几个月后,1985 年 9 月 13 日,他辞职并着手下一项计划。\\
\par
NeXT, Inc. 于 1986 年 2 月成立,启动资金为 Jobs 自掏腰包的 700 万美元。Mac 部门许多成员离开 Apple 加入新公司,其中包括 Joanna Hoffman、Guy “Bud” Tribble(软件部门负责人)、George Crow、Rich Page、Susan Barnes、Susan Kare 和 Dan'l Lewin。\\ 
\par
在 NeXT,Jobs 回到了他在 1985 年 8 月就为 Apple 设想的一个项目。那时他在高校巡回推广 Mac,结识了化学诺奖得主 Paul Berg。Paul 对在湿实验室里教学生重组 DNA 的成本\footnote{10 万美元。}感到挫败,觉得模拟教学会便宜得多。一个面向高校与学生的 3M\footnote{一兆字节内存(MegaByte)、一百万像素显示(MegaPixel)与百万次浮点性能(MegaFLOP)。} 工作站市场似乎存在\footnote{《The Second Coming of Steve Jobs》。}。NeXT 决心打造一台强大又足够便宜、让大学生买得起的机器。\\
\par
\fq{我希望斯坦福有个孩子能在宿舍里治愈癌症。}{Steve Jobs, 1987}\\
\par
Steve Jobs 不计成本。花 10 万美元请 Paul Rand 设计了标志。在加州 Fremont 建造了一座自动化工厂,具备自动贴片主板装配能力\footnote{来源:“The Machine to Build The Machines” 迷你纪录片。},月产可达 1 万台。设计公司 Frogdesign(在 Apple IIc 上已证明实力)也被聘用。目标是 1986 年底出货。\\
\par
这台机器必须完美,遵循 Alan Kay 的理念:同时打造硬件与其运行的软件。\\
\par
\fq{对软件真正认真的人应该自己做硬件。}{Alan Kay, 1980}\\
\par

借助在 Apple 尤其是 Macintosh 上的经验,公司定义了 NeXT Computer 的三大支柱:GUI、网络与面向对象编程。\\
\par
\fq{我去了施乐 PARC。他们很友善,让我看了他们的研究成果。他们展示了三件事,但我被第一件彻底迷住,以至于几乎没看到另外两件。他们给我展示了面向对象编程——我看了,但没真正看见。另一件是联网的计算机系统……他们有上百台 Alto 通过邮件等方式联网,我也没真正看见。我被他们展示的第一件事——图形用户界面——完全吸引住了。我觉得那是我一生中见过的最棒的东西。 }{Steve Jobs, 1995}\\





\par
\section{NeXT 计算机}
第一台机器在 1989 年发货,三年辛苦工作后终于交付。未能达到最初的发布目标并不重要,Jobs 在被记者问到延期时 famously 回应:“晚?这台电脑领先时代五年!”\\
\par
机器基于 25 MHz 的 Motorola 68030、8 MiB RAM,并配备 DSP 与 FPU 等强大的协处理器,硬件性能出色,且外观惊艳。那是一个大多数电脑外壳还由米色塑料制成的时代,而这台用镁合金喷漆打造的一英尺完美立方体格外优雅。\\
\par
\cfullimage{NEXT_Cube}{NeXT 计算机}


显示器本身也是艺术品。17 英寸 MegaDisplay 可提供 1120 x 832 像素的高分辨率\footnote{当时,14 英寸显示器的 640x480 分辨率已属 PC 高端标准。},像素密度 92 DPI。Cube 的 256 KiB VRAM 允许每像素四级灰度。上市初期供应链紧张,订购 NeXT Computer 的客户会收到两份包裹——一份来自 Fremont 的主机,另一份来自 Sony 的 MegaDisplay。





\begin{figure}[H]
\centering
\cscaledimage{1}{68030_blueprint.png}{Motorola 68030}
\end{figure}
\par
NeXT 计算机诸多创新之一是采用 256 MiB 磁光盘驱动器——介于硬盘与软盘之间的混合媒介,试图同时覆盖两种用途。Steve Jobs 说,它应能让用户“把整个世界装进背包”。\\
% Released in 1987
\par
在这台机器的核心,32 位 68030 是 Motorola 68000 系列的最新产品。这个选择很可能受 NeXT 硬件工程师在 Apple Macintosh 与 Lisa(均使用 68000)的经验影响。\\
\par
68030 运行在 25MHz,能够执行接近 5 MIPS。它没有内建 FPU,因此在主板上还放置了 Motorola 68882。%\footnote{当然,Motorola 文档声称几乎是两倍,达到 12 Mips!}



\begin{figure}[H]
\centering
\cscaledimage{1}{68030_layout.png}{Motorola 68030 示意图\protect\footnotemark }
\end{figure}
\footnotetext{来源:《The NeXT Book》, Bruce F. Webster。}
\par
\vspace{-3mm}
上图中,68030 的 273,000 个晶体管由以下部分构成:\circled{1} 内存管理单元、\circled{2} $\mu$ROM、\circled{3} nROM、\circled{4} 控制单元、\circled{5} 指令流水线、\circled{6} 程序计数器执行单元、\circled{7} 地址执行单元、\circled{8} 数据执行单元、\circled{9} 256 字节 i-cache、\circled{A} 256 字节 d-cache、\circled{B} 时钟发生器。\\         
\par
尚不清楚这两级缓存究竟带来了多少性能提升。每个只有 256 字节,意味着缓存命中率很低(英特尔就因此在 386 上弃用了片上缓存)。有趣的是,设计者同时使用了微码与纳码。提供了 16 个通用寄存器,这对 CISC 架构并不常见\footnote{Intel 基于 CISC 的 486 只有 8 个。}(如此多的寄存器更常见于 RISC)。
 


\begin{wrapfigure}[9]{r}{0.33\textwidth}
\centering
\scaledimage{0.33}{next/next-cube-system.png}
\end{wrapfigure}

与线缆杂乱的 PC 相反,NeXT 计算机形成一条链:鼠标连键盘,键盘连显示器,显示器再连 Cube。\\


\par


如果说 NeXT Computer 的规格初看令人赞叹,价格却成了严重问题。市场研究显示学生和研究人员希望工作站价位在 3000 美元。NeXT Computer 起价超过理想价位两倍,为 6500 美元。更糟的是,基础配置的光驱虽适合备份却太慢,不适合运行系统。不仅噪音大且不稳定,访问时间 90 ms,比硬盘慢 10 倍,导致操作系统几乎无法使用。这使得“可选”的 3500 美元 330 MiB SCSI 硬盘成为绝对必要,将最终价格推到 10000 美元!一台甚至无法输出彩色的电脑,却要这个价。\\
\par




\section{产品线}
NeXT Computer 销量不佳,原型机被停产,产品线得以更新。1991 年 NeXT 发布三款新产品\footnote{提前四个月在 1990 年 9 月 18 日宣布。}:NeXTcube 是 NeXT Computer 的直接继任者。更小、更扁平的 NeXTcube 变体叫 NeXTstation,具备内建彩色显示但没有扩展槽。最后还有一块图形与视频处理扩展板 NeXTdimension。\\
\par
\newcolumntype{L}[1]{>{\hsize=#1\hsize\raggedright\arraybackslash}X}%
\newcolumntype{R}[1]{>{\hsize=#1\hsize\raggedleft\arraybackslash}X}%
 \begin{figure}[H]
\centering  
\begin{tabularx}{\textwidth}{ L{1.7}  R{0.3}  R{1.4}  R{0.8}  R{0.8}}
  \toprule
  \textbf{名称} &  \textbf{年份} & \textbf{CPU} & \textbf{价格} & \textbf{折算 2018}   \\
  \toprule 
   NeXT Computer           & 1989 & 68030 25 MHz & \$6,500 & \$12,938 \\
\toprule 
   NeXTstation             & 1991 & 68040 25 MHz & \$4,995 & \$9,157 \\
   NeXTcube                & 1991 & 68040 25 MHz & \$12 395 & \$21,171 \\
   NeXTdimension           & 1991 & i860  33 MHz & \$3,995 & \$7,552 \\
   NeXTstation Color       & 1991 & 68040 25 MHz & \$7,995 & \$14,656 \\
\toprule 
   NeXTcube Turbo          & 1992 & 68040 33 MHz & \$10,000 & \$18,121 \\
   NeXTstation Turbo       & 1992 & 68040 33 MHz & \$5995 & \$11,932 \\
   
   NeXTstation TurboColor & 1992 & 68040 33 MHz & \$8995 & \$17,904 \\
   \toprule
\end{tabularx}
\caption{\protect\NeXT 1989-1993 年产品线\protect\footnotemark。}
\end{figure}
\par
\footnotetext{来源:kevra.org(竞品硬件对比)、https://simson.net/ref/NeXT/specifications.htm、以及《The Second Coming of Steve Jobs》。}
\par
1992 年,NeXT 通过 Turbo 版本强化了整个产品线,并推出后来在 id Software 内被奉为“黄金标准”的 NeXTstation TurboColor。






% The only weakness of the standard and turbo versions was the display.
% The standard and turbo versions' only weakness remained the video.

\section{NeXTcube}
从外观看,NeXTcube 的 12 英寸立方主机与前代 NeXT Computer 几乎相同。但内部则截然不同。\\
\par
CPU 升级为 Motorola 68040 25MHz,吞吐量是 68030 的三倍,达到 15 MIPS\footnote{来源:“Fast New Systems from NeXT”,B.Y.T.E 1990 年 11 月。}。内存容量翻倍,出厂 16 MiB,可扩展至 64 MiB。磁光盘被弃用,转而使用必选硬盘、软驱,以及可选 CD-ROM 驱动器。硬盘容量可选 400 MiB、1.4GiB 或 2.8GiB 的 SCSI 盘。软盘容量是 PC 的两倍,达到 2.88 MiB。\\
\par
\begin{minipage}{\textwidth}
\scaledimage{0.5}{next/next-crt-top.png} \scaledimage{0.5}{next/next-cube-top.png}\\
\end{minipage}

\par
1992 年发布的 NeXTcube Turbo 几乎相同,仅把 68040 频率提升到 33 MHz,并将最大内存扩至 128 MiB。\\
\par
标准版与 Turbo 版的唯一短板是显示。配备 256 KiB VRAM 的机器只能输出 4 色(白、黑与两级灰度)。要让 NeXTcube 输出彩色,用户必须购买 NeXTdimension 板卡。\\
\par

\trivia{NeXTcube Turbo 的扩展槽可安装 Nitro 板卡,将 68040 33MHz 替换为 68040 40 MHz。已知仅有 10 块 Nitro 板卡存在,极其稀有、被藏家追捧。}
\pagebreak


\cfullimage{next/NeXTcube_motherboard.png}{NeXTcube 主板}
\par
打开机器可以看到 NeXT 在细节上的坚持。上图的 NeXTcube 主板显示了性能与美学并未被彼此牺牲。表面贴装技术让器件比常规更紧密地排列。\\
\par
藏在散热片下的 CPU 拥有 120 万晶体管,性能大幅提升。哈佛架构(指令与数据分离存储与传输)、回写能力、8 KiB 缓存(4 KiB 数据与 4 KiB 指令)以及集成 FPU,使吞吐量相对 68030 提升三倍。





\vspace{50 mm}
\drawing{NeXTCube_motherboard}{NeXTcube 主板示意图}
\par
NeXTcube 主板的芯片与组件:\\
\par 
\circled{1} NeXTBus 连接器,
\circled{2} VLSI NeXTBus 接口芯片,
\circled{3} CPU Motorola 68040,
\circled{4} 256~KiB~VRAM,
\circled{5} DRAM 控制器 CS38PG017CG01,
\circled{6} 集成通道处理器(DMA 控制器 Fujitsu MB610313),
\circled{7} 光存储处理器(Fujitsu MB600310),
\circled{8} 16 个 SIMM 插槽,每条最大 4MiB,总计 64 MiB,
\circled{9} DSP-56001RC20,
\circled{A} 电池,
\circled{B} NeXT BIOS PROM,
\circled{C} DSP 768K 插槽,
\circled{D} 硬盘与软驱接口。
\circled{E} 多个连接器(自上而下):56001 DSP、串口 A\&B、SCSI2、打印机、以太网 RJ45\&CoaxBNC、DB19 显示器。 
\circled{F} Intel n82077 软驱控制器。
\circled{G} DSP SRAM(8KiB)MCM56824A。
\circled{H} SCSI 控制器(NCR 53C90A)
\circled{I} 100.000MHz 振荡器 K1149AA





\section{NeXTstation}

由于成本是产品线的主要问题,NeXT 试图推出更便宜的产品。它的设计接近 NeXTcube,但移除了非必要元素,从而打造一台成本低三分之一的全合一机器。\\
\par
NeXTstation 的定价与外观使其直接对标 SPARCstation。不再是完美立方体,这种机箱被昵称为“板砖”(Steve Jobs 也禁用的称呼是“披萨盒”),获得了用户欢迎,并成为 \NeXTns{} 最成功的计算机。\\
\par 
\cfullimage{cube_vs_slab.png}{1991 年 NeXTWorld 杂志中的 NeXTstation 广告。}
\par
\vspace{-10pt}
设计师从 NeXTcube 与 NeXTdimension 中选取元素打造这台一体化、不可扩展的机器。三条 NeXTBus 扩展槽被移除,CD-ROM 也被去掉。右侧新增 2.88MiB 软驱。最显著差异来自彩色版本:它拥有 2 MiB VRAM,可输出 16 位 RGB 色彩。为满足带宽需求,主板重新设计并加入 Bt463 RAMDAC。\\
