\begin{wrapfigure}[14]{r}{0.25\textwidth}{
\centering \scaledimage{0.25}{3do_logo.png}}
\end{wrapfigure}
3DO 公司由前 Electronic Arts 与前 Apple 员工 Trip Hawkins 于 1991 年创立。公司没有能力亲自生产硬件,因此目标不是打造一台机器,而是制定一个标准。3DO 提供其机器规格的授权。作为回报,潜在制造商支付费用并获得蓝图,从而大幅降低研发成本。公司的商业模式是对每台主机与每款游戏收取版税。这个想法并不疯狂,因为 JVC 曾通过 VHS 录像带系统成功实现了类似模式。\\

Sony 曾短暂考虑将其用于 PSX 项目。日本公司甚至参观了圣马特奥办公室的原型机,但最终拒绝。其他几家公司确实购买了 3DO 的生产权(Samsung、Toshiba 与 AT\&T),却从未真正制造。\\
 \par
最终在 1993 年 10 月,Panasonic、Goldstar 与 Sanyo 分别发布了各自的机器:Panasonic 3DO GZ-1、Goldstar 3DO 与 Sanyo TRY 3DO。后来 Creative 还发布了一块可插入 PC 的 ISA 卡。\\
\par
\cfullimage{consoles/3DO.png}{Panasonic FZ-1 对 3DO 规格的实现}
\par
\vspace{-5pt}
\trivia{这台机器的规格最初在 1989 年由 Dave Needle 与 RJ Mical 在餐馆的餐巾纸上写下\footnote{来源:RetroGamer \#122《Ahead of its time》。}。}












宣布于 1992 年 CES 展会上,3DO 的概念与规格立刻引发轰动。它是第五代 32 位时代的第一台主机,市面上没有任何硬件可与之匹敌。\\
\par


\fq{3DO 在引擎盖下使用 ARM60 RISC
处理器,并拥有两颗强大的定制图形芯片与一个动画处理器。
它还配备了 3Mb 内存与多任务操作系统。独特之处在于,
开发者为操作系统而非硬件编写游戏,从而保证向后兼容。 }{RetroGamer \#122}\\

\par
3DO 的野心不止游戏。凭借其 CD 格式,它希望取代录像机,并通过互联网实现电影流媒体。\\
\par
但事情最终急转直下。1993 年 2 月,《WIRED》杂志刊登了一篇整版文章《3DO:潮流还是泡沫?》,提出了许多可行性担忧。\\
\par
其中一个问题是商业模式。为了盈利,3DO 必须在每台主机与每款游戏上赚钱。这与 Sega 与 Nintendo 的模式相反——他们会亏本卖主机,再靠游戏盈利。这使 3DO 的建议零售价飙升到 699 美元,成为市场上最贵的主机\footnote{除了 Neo-Geo,其价格相同且对大多数玩家而言始终只是梦想。}。相比之下,PlayStation 的首发价只有 299 美元,不到一半。机器从一开始就被贴上“富家子弟”的标签\footnote{价格在发布不久后下调,但名声已无法挽回。}。\\
\par
另一个致命点是游戏库。首发游戏仅六款,质量不高,除了《Crash 'N Burn》。直到最后一刻的固件与开发套件改动,阻碍了工作室赶制首发作品。\\
\par
讽刺的是,3DO 也被其媒介 CD-ROM 所害。容量比旧时代大了 150 多倍(650 MiB 对 4MiB),游戏工作室尝试制作冗长的像素化过场与“几乎互动”的电影,结果一点也不好玩。\\
\par
更多仓促且质量低下的游戏,再加上 1994 年末 PSX 的发布,彻底摧毁了回天乏术的希望。
到 1995 年,销量不足 70 万台\footnote{来源 Next Generation 杂志,1995 年 2 月。},标准失去势头。不久之后,3DO 标准与其缔造者 3DO 公司双双陨落。





\cfullimage{3D0_motherboard.png}{3DO Panasonic 型号 “FZ-10 R.E.A.L” 主板}
\par
Panasonic、Goldstar 与 Sanyo 都设计了各自的主板,但所有 3DO 的功能相同。打开最受欢迎的 “FZ-10” 机型,能看到多达 17 颗芯片!\\
\par
\bu{芯片}\\
\par
\circled{1} 50 MHz CEL Engine CLIO,
\circled{2} 50 MHz CEL Engine Madam,
\circled{3} 2 MiB RAM,
\circled{4} 1 MiB VRAM(帧缓冲),
\circled{5} 12.5 MHz ARM60 主 CPU,
\circled{6} Corner engine(50MHz 数学协处理器),
\circled{7} 1 MiB OS ROM,
\circled{8} 数字视频编码器(25MHz VDLP),
\circled{9} 32 KiB 电池备份 SRAM,
\circled{A} ARM CPU 32 KiB SRAM,
\circled{B} DSP(16 位,25MHz),
\circled{C} CD 信号处理 LSI,
\circled{D} CD-ROM 控制器 MN1882410,
\circled{E} CD-ROM 固件,全部通过 50 MiB/s 与 36 条 DMA 通道连接。\\
\par
\bu{连接}\\
\par
\circled{F} 手柄接口,
\circled{G} 扩展端口,
\circled{H} 复合/S-Video 输出端口,
\circled{I} RF 调制器及其音频输出插孔。 


\fullimage{3do_motherboard}
\par
3DO 开发者被禁止直接访问硬件,除非为 ARM 处理器编写手工汇编。\\
\par
\fq{我们从未得到 3DO 硬件的寄存器文档。通过逆向,我们能访问 I/O 端口,但 3DO 公司告诉我们如果发现绕过 OS,游戏会被拒绝。}{
Rebecca Heineman}\\
\par
M2 项目——最初是 3DO 的加速扩展——后来演变为可能的 3DO 2,计划配备双 PowerPC 602 处理器以及更新的 3D 与视频渲染技术,但最终未完成。\\
%\footnote{"3DO Press Release" August 1994.} 
\par
\trivia{为了预想中的 M2 机器,开发者在游戏中隐藏作弊码以提升画面。在 \doom{} 中,序列 “Up, Right, L, Up, Right, Right, R, A, Left” 可将可视窗口放大到全屏。}


\subsection{3DO 编程}
图形编程通过 3DO 的 “CEL 引擎” 完成,由 Clio 与 Madam 芯片驱动。CEL 是“精灵(sprite)”的花哨叫法。每个 CEL 在屏幕空间中绘制,并伴随三组向量 HD、VD 与 HDD。它们可实现缩放、旋转、剪切,甚至在手册中称为“透视”的操作。CEL 引擎还能同时处理多个 CEL。\\
\par
HD 与 VD 显然用于设置水平/垂直向量,而差分向量 HDD 需要进一步解释。手册中这样描述:\\

\fq{
当固定的 HDX 与 HDY 设置一个 cel 的水平偏移,VDX 与 VDY 设置其垂直偏移时,结果总是严格的平行四边形——所有行边互相平行,所有列边互相平行。虽然可以改变平行四边形的尺寸与角度,但无法得到行边会汇聚或发散的透视效果。要添加透视,投影器使用 HDDX 与 HDDY 这一对偏移。
HDDX 与 HDDY 在每一行边开始时改变 HDX 与 HDY 值。当计算一条行边时,投影器会在 HDX 与 HDY 上各加一个 HDDX 与 HDDY 的偏移,然后用新的 HDX 与 HDY 值计算下一条行边。由于 HDDX 与 HDDY 可以改变行斜率与像素间距,它们可以让行边逐渐汇聚或发散\footnotemark。}{3DO Programmer Guide}\\
\par
\circled{1} 正常,\circled{2} 旋转,\circled{3} 透视(不正确),\circled{4} 倾斜并放大。 \\
\par
\footnotetext{这与 Atari Lynx 用于精灵扭曲的机制相同——也确实应当如此,因为 RJ Mical 与 Dave Needle 也设计了 Lynx。}
% \trivia{Despite its name, the "perspective" is not correct.}
\begin{figure}[H] \centering
\scaledimage{0.8}{3do_vectors.png}
\end{figure}

\scaledimage{0.90}{CEL_examples.png}


\subsection{\doom{} 于 3DO}
从规格来看,3DO 本可以成为 \doom{} 最佳的主机版本。Jaguar 版口碑不错,因此凭借更多内存与更强图形硬件,理应让玩家与发行商都满意。然而,现实荒诞:最终发布的版本成了对原作的屠杀,被公认为最差的主机版。\\  
\par
% How could they screw this up? Easily in fact, just give a lone developer ten weeks to do it and ship it.\\
% \par
1995 年 1 月,Art Data Interactive 以 25 万美元(有些文章说 50 万)并承诺在 1995 年圣诞节前发布为条件,拿下了 \doom~的 3DO 授权。公司内部觉得这是“印钞许可证”。他们向媒体承诺诸多特性,包括新武器、新怪物、新地图,以及由真人演员拍摄的全动态影像(FMV)以铺陈剧情。\\
\par
\cfullimage{doom_3do_fmv.png}{FMV 拍摄现场(照片由 Rebecca Heineman 发布)}
\par


项目很快外包给一家真正有游戏开发经验的公司。Art Data Interactive 很快意识到移植成本远高于预期:要花 100 万美元并历时一年,他们同意了。\\
\par
到 1995 年 7 月,与外包方的关系已恶化到无法修复\footnote{来源:“The unfortunate tale of 3DO DOOM” by Matt Gander.}。而 1995 年 10 月的承诺发布日期迫在眉睫,ADI 转而联系一家曾将 Wolfenstein 3D 出色移植到 3DO 的承包商。接受该项目的可怜人是 Logicware 的 Rebecca Ann Heineman。\\
\par
接下项目后,Rebecca 要求 ADI 提供她以为在签约时已拿到的源代码。结果什么都没有。最终她收到一张软盘,里面只有 \doom~商业版的 \cw{DOOM.EXE} 与 \cw{DOOM.WAD}!Rebecca 不得不向 ADI 解释什么是源代码,以及她无法从二进制开始工作。经过数周的挣扎,她终于联系上 John Carmack,并获得 Jaguar 版源代码。\\
\par
在距离发版仅剩 10 周的情况下,Rebecca 英勇奋战并按期交付。但最终成品受到严重损伤\footnote{Burgertime 7/12/2015: DOOM 3DO.}。\\
\par
\cfullimage{doom_3do.png}{3DO 上的 Doom}
\par
CEL 引擎用于绘制墙体(像 PSX 一样用一像素宽的柱),但一个 bug 迫使 Rebecca 用软件渲染平面。她没有时间写音乐播放的音频驱动,于是录下 PC 版并寄给 ADI 的 CEO(恰好也是吉他手)。公司请了一支乐队重新录制音乐,翻录曲目直接从 CD 播放。\\
\par
性能极差。看到结果后,id Software 要求把可视窗口从全屏缩小到 1/3。即便如此,帧率在大多数时候仍是个位数。\\
\par
ADI 以 15 万美元的授权与制造费用向 3DO 订购 50,000 份成品。公司收回成本的唯一希望是卖光每一份。面对 25 万用户的装机基数,以及 AAA 游戏通常只有 10,000 到 20,000 份销量,这是一次巨大赌博。遗憾而合理地,玩家讨厌它,媒体痛批它,ADI 不久后宣布破产。 \\n\par
\fq{虽然还是 Doom,但这是一次彻底失败的移植。}{Ed Lomas for CVG  - 评分 60\%}\\
\par
% BELOW IS A PLACEHOLDER, MAKE A DRWAWING OF ARCH INSTEAD.\\
% \fullimage{doom_3do.png}
\vspace{20pt}
\rawdrawing{3do}

% \par
% Lots of good stuff here: https://github.com/Olde-Skuul/doom3do
% \par
%  CornerEngine - RISC processor, capable of vertex calculation and transformation, and 4*4 matrix math : multiplication by another 4*4 matrix, multiplication by vector, dot product, rotation.

查看上一页的 3DO 硬件结构图,只会让人更加惋惜 Rebecca 没有更多时间完成这个项目。\\
\par

\fq{我在接到这份工作时被误导了移植进度。对方告诉我已有一个版本,包含新关卡、武器与特性,只需要“打磨”和优化即可上市。经过多次索要后我才发现根本没有这个版本,而 Art Data Interactive 错误地以为把游戏从一个平台移植到另一个平台只需要编译代码,加入武器就像放进美术资源那么简单。\\
% \par
% Uh... No...\\
\par
我在 1995 年 8 月接下这份工作,10 月中旬就要交付金版。我的 3DO 朋友恳求我让 DOOM 登陆他们的平台,而圣诞节将至,我几乎住在办公室里,只在必要时小憩,最终完成了移植。\\
% \par
% Shortcuts made...\\
\par
我没有时间移植音乐驱动,于是 Art Data 请了一支乐队重新录制音乐,这样我只需要调用流式音频函数播放即可。结果证明这是个好决定,因为虽然画面欠佳,音乐却获得一致好评。\\
\par
3DO 的操作系统围绕运行一个应用并清除而设计,存在大量内存泄露 bug。因此当我想在启动时加载 Logicware 与 id Software 的标志时,3DO 会泄露内存。为了解决它,我制作了两个应用,一个绘制 3DO 标志,一个显示 Logicware 标志。它们执行后被清除,主游戏即可无内存损失地运行。\\
\par
数据里还有一个 EA 标志的动画,因为当时 EA 一度要发行该游戏,但最终交易失败。\\
\par
垂直墙体由 CEL 引擎绘制成条带。然而 CEL 引擎无法处理 3D 透视,因此地板与天花板由软件渲染。我没时间将这部分翻译到 CEL 引擎,因为我的实现会导致纹理撕裂。\\
\par
我不得不自己写一套 string.h ANSI C 库,因为 3DO 的编译器自带的库有 bug!string.h???怎么能搞砸成这样!他们居然做到了!我花了一天用 ARM6 汇编实现了我需要的所有函数。}{Rebecca Ann Heineman}
