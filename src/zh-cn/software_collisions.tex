\section{碰撞检测} \label{blockmapdetails}
碰撞检测是引擎中重要的活动之一。每个移动对象(玩家、怪物或投射物)在移动前都必须进行碰撞检测。视线也依赖于高效的碰撞系统。造成直接命中伤害的敌人同样需要检查是否有清晰的射线。\\
\par
碰撞本可以通过 BSP 来检测。然而直到 Bruce Naylor 访问 id Software 后,John Carmack 才意识到这一点。那时 \doom{} 已经发布,并带有名为 blockmap 的碰撞检测数据结构。\\
\par
\vspace{10pt}
\drawing{E1M1_lines}{E1M1 的扇区与线段}
\par
每张地图都有一个 blockmap,它通过在 NeXTstation 上运行 \cw{doombsp} 预处理生成。它被保存为一个大胆命名为 \cw{BLOCKMAP} 的 lump,在运行时用于减少需要测试相交的线段数量。\\
\par
\cw{doombsp} 的工作很简单:将地图划分为 128x128 的轴对齐块。对于每个块,会列出所有穿过它的线段。最后,根据 blockmap 坐标(以 128x128 为单位)构建索引,指向对应的线段列表。注意一条线可能出现在多个块中。在地图 E1M1 中,结果如图 \ref{E1M1_blockmap} 所示。

\drawing{E1M1_blockmap}{通过 blockmap 索引的 E1M1 线段。注意空块未绘制}

所有地图遍历由抽象方法 \cw{P\_PathTraverse} 完成,它接收两点坐标(定义要检测碰撞的线)以及一个在检测到命中时调用的函数指针(也就是如何用 C 模拟 OOP)。\\
\par
\ccode{P_PathTraverse.c}
\par
函数 \cw{P\_AimLineAttack}(用于拳击与电锯攻击)以 flag = \cw{PT\_ADDLINES|PT\_ADDTHINGS} 调用 \cw{P\_PathTraverse},从而在遍历时只考虑线段与物体。任意地图坐标的块坐标可通过除以 128 获得(优化为 $\gg$ 7)。要检测与物体的碰撞,每当物体改变位置,其块坐标都会更新。\\
\par
\ccode{P_AimLineAttack.c}
